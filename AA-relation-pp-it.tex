\chapter{How do I think Process Philosophy and IT are related (what is the relation PP and IT)}
\section{Provisional section}



IT lijkt in zekere zin op het process model. Als we de hardware buiten beschouwing laten dan bestaat IT vooral uit processen die onderling met elkaar in relatie staan. De structuur bestaat uit de code regels.

Is het denkbaar dat processen spontaan ontstaan? Zou het mogelijk zijn om de pp te simuleren?
Nu waarschijlijk nog niet.
De wereld zonder it veranderd steeds als gevolg van nieuwe processen.
\todo{rdb de wereld met it dus niet?} Als ik de geschiedenis bekijk dan hebben mensen, ook een process neem ik aan, een grote impact op de ontwikkeling doordat zij, meer dan dieren bv, gericht ontwikkelen, dus processen creeren. Een process creer gericht andere processen.
Dat doet een process (verzameling van processen) zoals de mens in zijn eigen belang. Dus en process zal in het belang van zijn eiegen process een bepaalde invloed proberen uit te oefenen op het creeren van andere processen. Als de mens, of een dier of een soort, wegvalt dan is de kans dat nieuwe processen ontstaan die hem/haar/het gebaat zouden hebben op zijn minst afgenomen.


\section{Conclusion}

