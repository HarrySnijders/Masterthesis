%%% Local Variables: 
%%% mode: latex
%%% TeX-master: t
%%% End: 
%\documentclass[a4paper]{report}
\documentclass[a4paper]{Thesis}
% for report \usepackage{suthesis-2e}
\usepackage[colorlinks,citecolor=black]{hyperref}
\usepackage{apacite}
\usepackage[english]{babel}
\usepackage[pdftex]{graphicx}
\usepackage{subfig}
\usepackage{verbatim}
%ToDoNotes
\usepackage{todonotes}
%\usepackage{cancel}
\usepackage{soul}



\begin{document}
	
\begin{titlepage} 
\begin{center}
		
% Upper part of the page. The '~' is needed because \\ % only works if a paragraph has started. \includegraphics[width=0.15\textwidth]{./logo}~\\[1cm]
		
%\textsc{\Large Creativity and ICT technology}\\[1.5cm]
		
%\textsc{\Large Final year project}\\[0.5cm]
		
% Title 
\HRule \\[0.4cm] { \huge \bfseries Creativity and ICT \\[0.4cm] }
\HRule \\[0.4cm] { \large \bfseries A process philosophical view on the influence of ICT on technological ideation \\[0.4cm] }
\HRule \\[0.4cm] { \large \emph{Author:} Harry Snijders \\[0.4cm] }

\vfill		
% Author and supervisor 
%\noindent 
\begin{minipage}[b]{\textwidth} 
	\begin{flushleft} 
		\large Thesis for the degree of Master of Science \\[0.4cm]

\begin{tabular}{ll}
	
		\large \emph{Studentnumber:}	 & S1143972	\\
		\large \emph{Place:} 			 & Oldenzaal	\\
		\large \emph{Date:} 			 & \today \\
		\large \emph{Institution:} 		 & University of Twente \\
		\large \emph{Department:} 		 & Department of Philosophy  \\
		\large \emph{Programme:} 		 & Philosophy of Science, Technology and Society \\
		\large \emph{Specialization:} 	 & Technology and the Human Being \\
		\large \emph{First supervisors:} & Dr. Johnny Hartz Søraker and Dr. Aimee van Wynsberghe \\
		\large \emph{Second supervisor:} & Prof. Dr. Philip Brey \\
\end{tabular} 		

	\end{flushleft} 
\end{minipage}
		
\end{center} \end{titlepage}
%content 
\bigskip
\bigskip
\tableofcontents
%\listoffigures

%PREFACE AND ACKNOWLEDGEMENT

%\begin{abstract}
%\end{abstract}
%\bigskip
%\bigskip

%\providecommand{\keywords}[1]{\textbf{\textit{keywords---}} #1}
%\keywords{}

%\bigskip

%\begin{flushright} 
% But in the real world it is more important that a proposition be interesting than that it be true.
% (A.N. Whitehead)
%\end{flushright} 

%------------------------------------------------------------------------------
%------------------------------------------------------------------------------
\chapter{Introduction}

Versie 27-juli-2015.\\

De \textit{provisional section} is een hulp tijdens het schrijven. Daar begint het eigenlijke hoofdstuk. Voor deze sectie staan algemene opmerkingen en ideen voor het hoodstuk.\\


Ik maak gebruk van het TODO pakket van latex om herinneringen toe te voegen voor mijzelfdie me te binnen schieten tijdens schrijven bijvoorbeeld.\\


Deze week heb ik de introductie in pp gelezen van Rescher en een artikel van Cloots over de vraag naar het ultime. Vervolgens heb ik de structuur van de thesis nog eens goed bekeken en ben overgestapt op een draft versie waar allelei in staat ook samenvattingen, en een writing versie die de uiteindelijke thesis moet worden.

En toen eindelijk maar aan het schrijven. Ik heb er toch voor gekozen om met de introductie te beginnen om nog eens scherp te krijgen wat ik nou eigenlijk aan het onerzoeken ben. Ik verwacht dat ik aan de verschillende secties in dit hoofdstuk delen ga toevoegen als ik met de verschillende subvragen bezig ga.
Voor maandag is alleen de introduction van belang. Het is een eerste draft, ik ben vooral benieuwd naar of je mijn redenering kunt volgen en overtuigend vindt waar dat van toepassing is, of er iets mist en of ik structurele taalfouten maak.\\

%A fundamental difference between a natural science like physics and an artificial science such as computer science relates to the age old philosophical distinction between is and ought.
%Natural science onderzoekt wat er is, computer science onderzoekt hoe het zou moeten. Misschien is ditte gebruiken.

Working definition:
Creativity is the production of novelty with a structure that did not exist before.
All existing entities in reality have a structure, whether it is a idea or an artifact.


\section{Provisional section}
\todo{waar mogelijk nog referenties toevoegen}
\todo{toevoegen een verwijzing dat creativiteit in filosofie een ondergeschoven kindje is terwijl de psychology het belang sinds 1950 de komst van J.P. Guilford als voorzitter van de APA}

(Grotendeels overgenomen van het proposal)\\

Our society is saturated with all kinds of technological objects that shape our every day lives in many ways. With modern ICT artifacts like smartphones and computers we stay in contact with other people in quite different ways then our ancestors did without these artifacts.  But artifacts, like smartphones and computers, are more then just tools for communication, they influence our perception of the world and influence our behaviour \cite{Verbeek2005,verbeek2011moralizing}. Artifacts are not neutral technological objects that function only as a tool to serve an end.

In the past, technology pessimist thought that the development of society and culture followed the development of technology and therefore technological development was hard or impossible to influence. 
But contemporary scholars argue that technological development did not follow a straight path as the most efficient way to the end it was designed for. Technological development, they argued, was depending on social and cultural factors \cite{Bijker1989}.

If technology development is depending on social and cultural factors and therefore on humans (individual, as a group or as an society al large) then humans are able to steer the development of technology and are responsible for the effects of the technological on themselves and the environment. \todo{bron vermelding krijg ik van Rdb} Therefore it becomes important to understand the powers at work in he development process and possibilities to control or at least influence this process.  
No matter how important social and cultural factors are, at the end, for the idea of new technologies, or the design  of new artifacts, creativity is needed because in the way I see it, creativity is lies at the hart of the technological novelty.

Creativity is a word used for description of many different concepts in the domain of novelty but there are some characteristics most scholars in the field degree on that is that creativity is related to the generation of ideas or artifacts that are both original and valuable. With original is meant that the novelty produced did not exist in a similar way in the world before. A newer version of a smartphone with minor changes will not be considered very original. The first smartphones however could be seen as an original new artifact. 
The value of novelty is less obvious. It is often used, Boden does for example, to exclude novelty produced by fools from creative novelty. But what to say about novelty that is used against people? In the case of technology, the artifact produced should have value for its users.
\todo{hier mist nog wat, misschien verder ingaan op de waarde van novelty, zie ook Berry Gauts, wil ik ook nog wat zeggen over het surprise effect, dat iets nieuws niet voor de hand mag liggen anders is wordt het niet als creatief beschouwd of doe ik dat pas later. Zie o.a. Boden, Fergusson}

%Different models exist representing the creative process, the ones I have seen so far are the computational model \cite{Boden2004} and the psychological GenePlore model \cite{finke1996imagery}.

Creativity is possible on different levels \cite{Ferguson2010}, groups or individuals can be creative. Also different kinds of agents can be creative such as systems, animals or humans (Ibid.). I will however, only consider technology that is made by humans and finds its origin  in human creativity. I will also focus on the creativity of a single human,

\section{Creativity}
Creativity is a very wide spread concept and a highly appreciated value. It is used to classify a person or an artifact or an idea.  It is also used to denote the process of creation where novelty is produced. Creativity can be a an experience, a agent might think he is being creative in his practice.
\todo{bronnen vermelden}

%new and value
%Scholars have different ideas about what creativity is but most agree that creative ideas and artifacts are those that are new and have a value. Boden differs two types of new. New can be that it either has not existed before in human history, called historical new or it is new to the creative agent, called psychological new \cite{Boden2004}. The value clause is added to exclude random novelty from being considered creative. For example a random sequence of characters might be the first one of its kind but it has no value and is therefore not considered as a creative product.


\paragraph{Novelty, agents and ideas}
\todo{stukje over TA en upstream?}
% alleen geinteresseerd in nieuwe ideen
I am interested in the origin of new technology, created by men. Whether the agent that comes up with new technology is considered a creative agent or not seems not to be of great importance. However an agent that is potential seen as creative agent might generate other ideas than an agent that never will be considered creative because the ideas she generates are to smalls steps forward and to much alike other ideas. I will focus on the generation of novelty that has an potential of being considered as creative but will leave the agent out.
\todo{citeer het artikel over de genius en alledaagse creativiteit om problemen op te lossen}

The outcome of an creative process might be an artifact or an idea. in the case of technology, the production of the artifact is accompanied with extra limitations,  technology has to work to have value.
 It has, among others, a tool function. 
\todo{wat is technology eigenlijk en waar dient het voor, is daar iets over te vinden?}
Software for example has to be designed according strict rules else it will not execute on the hardware it is designed for. An technological artifact is very much depending on its specific environment and the current technological possibilities. 
Like all artifacts, a technological artifact has to be manufactured and because it is more or less complex this requires a design of some type first.
\todo{see Dasgupta}
A design is an answer to an idea, where the idea can be a problem to solve or a new functionality required.
The environment is often included in the design process. For example in the case software where the hardware it has to run often puts limitations to the software to develop and is therefore included in the design.
It is very easy to imagine that an idea for a new technology is not yet possible to realize, to make a design and build the artifacs to realize the idea. 
Like for example he idea of a submarine with crew that is shrunken to a microscopic size and injected into the blood stream of a nearly assassinated diplomat to save his life (IMDB, the fantastic voyage). It was before 1966 that the idea is posed but it still is not possible to shrink objects. 
Coining the idea was possible, maybe the story is based on earlier ideas about micro objects that could be injected into the body for repair jobs,  but building the technology is not yet possible. 

\todo{moet ik hier nog iets zeggen over kunst en science, omdat veel literatuur die ik gezien heb juist daar over gaat}
%These roles are much less strict in the domain or art where Many artifacts in the domain of art and literature,

Since I am interested in the origin of new technology and not the realisation of new technology, and the fact that in the domain of technology an artifact is always preceded by an idea and a design to accomplish that idea, I will not have to consider the creation of new artifacts for understanding the origin of new technology but it is sufficient to focus on the genesis of new ideas.

\todo{nog iets over innovatief gebruik van niewe technology en user innovation? Of het verschil tussen innovatie en orginaiteit?}


%hoe om te gaan met waarde?
%twee fasen process, ideeen genereren fase twee ideeen beoordelen.
\todo{misschien opnemen Creativity SCOC factor. Social Construction of Creativity. Het idee dat of iets als creatief beoordeeld wordt wordt bepaald door de maatschappij
	}

\paragraph{Novelty and value}
I have said something about novelty, the agent, artifacts and ideas but what about value? How important is it for the TI that the ideas produced have a value. 

Creativity is related to a value and for deciding if something has a value some kind of judgement is necessary. In order to be valued as an creative action or outcome, the subject of creativity has to be compared to its environment before it can be granted with the predicate creative. It does not matter who performs the valuation, the idea has to be known first before it can be the subject of valuation.This is even the case if the idea does not leave the human brain.
In fact some psychological theories of creative processes in the brain are based on this two phase principle like the Geneplore model \cite{finke1996imagery} and the BVSR (Blind Variation Reductive Selection) model \cite{simonton2003human}.
In the GenePlore model novelty is generated in one or more cycles of the following two phases process. In the first phase new ideas are generated and in the second phase these ideas are validated for their usefulness. It is possible te enter a new first phase and generate new ideas based on the outcome of the second phase, these on their turn are validated and so on. 
BVSR is based on the evolution theory. The first phase exist of generating new ideas based on the blind variation of existing ideas. In the second phase the ideas are validated and selected according their value.
\todo{wat is het verschil tussen GenPlore en BVRS}

First there is the novelty produced and then this novelty is the subject of a validation process using different criteria to compute its value.
This can only be accomplished after the idea is posed. It does not to relate to how novelty is created but to how the novelty is appreciated. 
The outcome of the valuation will be that the novelty is considered being not creative or creative on a sliding scale depending on how different the idea is in the light of already existing ideas.
\todo{boden surprise? Examples?}
Im interest lies at the first phase. But again, as I said with the agent, I will focus on the generation of new ideas that have a potential of being merited as creative ideas.

\paragraph{Technological ideation}
%technological ideation
Above I showed why my interest lies not at the agent, the artifact and te validation part of creativity. I am first of al interested in the coming into existence of the new idea.  
Referencing for this focus to creativity might lead to confusion because creativity is for denoting so many different aspects of novelty. I am interested in the very beginning of new technology, the idea for a new technology, and therefore understand creativity as technological ideation (TI). 
With understanding Technological Ideation I hope to find the source where new technology comes from and what the influence of ICT is on this source.
Because the common notion related to technological ideation is creativity, I will still use creativity to find a link in existing literature. 
\todo{In eerste instantie focus ik op de invloed van softare op  TI in het algemeen. Het kan echter nodig blijken om TI toe te spitsen op een beperkerter domein, dan kies ik mogelijk wederom voor software ontwikkeling}


\section{An other world view}
The common world view in the western world is that of a world constructed out of matter an Aristotelian heritage. Matter consist of small, standard, building blocks that last for ever. 
Creativity, the generation of novelty, conflicts with this idea of endurance and is therefore more difficult to explain in a world that exists of fixed and ever lasting building blocks.
\todo{see Freidrich Rapp ch 5 in witeheads metaphysics of creativity about how modern sience is not capable of dealing with principles like creativity because of the scientific method}
In a process view of the world, change and creativity is a basic principle. The world is no longer constructed out of matter but consists of processes. Micro processes, related in myriad ways with other micro processes, build up to macro processes, that because of their recurred characteristic construct the world as we experience it. In this world view every thing is explained as a process, also we humans. 

Change is a basic principle in process philosophy. There is not one version of process philosophy, different scholars have different description which contribute to the overall idea of process philosophy.
On of the goals of process philosophy is to overcome the limitations of the substance paradigm and cause a paradigm view.
American philosophy turned out as a fertile source for process philosophy with scholars as C.S. Peirce, John Dewey and William James. The best known process philosopher from the 20th century with the most comprehensive description of Process philosophy was A.N. Whitehead. For Whitehead creativity is the absolute principle of existence. It is his process philosophy I like to use for this case.
\todo{Bergson, Hartshorn}
\todo{twijfels zijn er ook, zijn atomistisch idee van basis processen staat me wat tegen, liever ga ik uit van het oneindig opdelen  van processen, maar dat zien we verderop in het hoofdstuk over proces philosophie}

\todo{moet hier nog meer bij, bv welke auteurs, dat er verschilende stromingen zijn}
\todo{misschien refereren naar de presentatie van Mieke Boon over anders kijken. Met PP kijk ik anders dan met de gangbare vaste stof metaphysica}


\section{What type of technology}
Technology is a very broad domain with very different disciplines. 
I could either research from a top down position the process of technological ideation (TI) or perform the research bottom up.
Top down means that I do not choose for a specific technology but take the field of technology as a whole. Besides that this could turn a out to be a to big endeavour, I also think that I could oversee details that are important for my case. Therefore I choose a bottom up  approach and choose for limited technological field.
I choose as a technology ICT and more specific the domain of software. I have two reasons for this choices. First because I am a software engineer of origin which grants me with specific knowledge in the field of software development. Second because software, how it operates with its concept of processes and the products it brings to the fore like for example virtual reality reassembles the world view of process philosophy.


\section{Problem statement}
%-------------------------------------------------------------
\todo{dit is de, grotendeels oorspronkelijke, omschrijving uit het thesis proposal, deze houd ik al schrijvende aan de verschillende secties bij en gebruik ik tijdens het schrijven als een rode draad. }
\todo{niet te veel naar kijken, vooral aan het eind bijwerken}

Philosophers are increasingly interested and looking at the relations between humans and technology. 
In the field of ethicists new technology are assessed for their influence on the environment and their impact on just distribution of goods and fair distribution of the good life,  anthropological philosophers study how humans perceive the world and how humans are shaped both by technology and social philosophers study the influence of the society on technological development.
There is however remarkable little attention from philosophers what lies at the very hart of new technology, creativity \cite{gaut2010philosophy}. How does a human get her first idea for a new technology, and what than is the impact of technology on her creativity?

A contemporary field of technology where innovations cycles have become very small and new technologies are created at a very high rate is ICT (information and communication technology).

I propose therefore to ask the following question:
\textit{(Q) What is the impact of ICT software technology on creativity?}
In order to answer this question we have to understand what human creativity is. 

Therefore I propose following sub questions:
\textit{(Q1) What is (human) creativity} and  
\textit{(Q1) What is IT-software} and  
\textit{(Q1) What is Proess Philosophy} and  
\todo{aanvullen, dit zijn waarschijlijk de verschillende hoofdstukken}

\section{Conclusion}

\section{Outline of the thesis}


%------------------------------------------------------------------------------
%------------------------------------------------------------------------------
\chapter{What is process Philosophy}

\paragraph{What is the pronlem with the common western philosophical view of reality?}
	Knowledge comes from science
	Scientifical method very fruitfull but fails to describe phenomena like creativity
		Logic, descartes object-subject, see Rapp in ch 5
		labratory, reductionist, isolating and testing, not as a combination (interwined verbeek)
		paradigma, limits seeing
		
\paragraph{What is the process philosophical view of reality}
See Reschner introduction en SEP art \cite{Rescher-2012-sep}
Seibt SEP entry \cite{Seibt-2013-sep}


Process philosophy is a view of reality that emphasizes becoming and changing over static being. The principle of process philosophy ranges back to the Greek.
In the Western tradition it is the Greek theoretician Heraclitus of Ephesus (born ca. 560 B.C.E.) who is commonly recognized as the founder of the process approach \cite{Seibt-2013-sep}. 
 but there are two contemporary philosophers that are most associated with the term process philosophy namely Alfred North Whitehead (1861-1947) and Charles Hartshorne (1897-2000).



\paragraph{How does process philosophy help to understand creativity?}

\paragragph(What is creativity)
 beschrijving?
Algemene omschrijving PP
Wat is een process
Wat is creativiteit




\section{fuzzy logic}
Enclopedia of creativity chaper AI
see also fuzzy set


\section{ctr4process}
http://www.ctr4process.org/about/what-process-thought

Basic Doctrines

Process metaphysics, in general, seeks to elucidate the developmental nature of reality, emphasizing becoming rather than static existence or being. It also stresses the inter-relatedness of all entities. Process describes reality as ultimately made up of experiential events rather than enduring inert substances. The particular character of every event, and consequently the world, is the result of a selective process where the relevant past is creatively brought together to become that new event. Reality is conceived as a process of creative advance in which many past events are integrated in the events of the present, and in turn are taken up by future events. The universe proceeds as "the many become one, and are increased by one" in a sequence of integrations at every level and moment of existence. Process thought thus replaces the traditional Western "substance metaphysic" with an "event metaphysic." Terms that further characterize process thought are inter-relatedness, unity-in-diversity, non-dualism, panentheism, mutual transformation, person-in-community, and panexperientialism. The following links are helpful short essays written by scholars that describe and summarize process thought.

Process Philosophy

A Synopsis of Process Thought, by Sheela Pawar
Process Philosophy: Stanford Encyclopedia of Philosophy, by Nicholas Rescher
Process Philosophy: The Internet Encyclopedia of Philosophy, by J.R. Hustwit
Process Philosophy: Wikipedia
Process Thought: It's Value and Meaning to Me, by Charles Birch


\section{Sheela Pawar}
A Synopsis of Process Thought


\section{SEP rescher is opgevolgt door seibt}
http://plato.stanford.edu/archives/sum2012/entries/process-philosophy/

\section{Section Overview}

-1 de big bang, waar komt deze vandaan?
	http://www.big-bang-theory.com/
		Geen duidelijkheid over het begin, in het begin was er niks, het universum en tijd is uit de big bang voorgekomen.
		Geen ontploffing maar een expansie, senl op blazende ballon
	
Whitehead
	Process

Whitehead concept of creativity and modern science
 	



Vragen beantwoorden:
	waarom pp? als model van de cosmos voor deze thesis?
	wat is PP
	
	Wat is een process, micro proces macro proces
	Wat is de drijvende kracht die nieuwe processen doet ontstaan
	waar komen nieuwe processen vandaan
	Aan welke eisen moet een process voldoen
	Hoe ontstaan nieuwe relaties tussen processen
	wat verstaat men onder creativiteit
	wat is het concept van een (nieuw) idee

Begin in deze sectie met een algemene verklaring wat PP is, gebasseerd op een SEP artikel van Johanna Seibt. Er zijn veel verschillende stromingen,waar de wat oudere een totaal oplossing ptoberen te brengen zijn de huidige vooral specialisaties die bijdragen aan een groter geheel.
Zet PP af tegen het bestaande westerse paradigma van de materie-philosopie en licht toe waarom pp zo geschikt is voor de verklaring van creativiteit. Omdat creativiteit over vernieuwing gaat en dat het basis principe van pp is.

Geef een overzicht van N>3 (waarom 3?) verschillende perspectieven van verschillende auteurs op creativiteit. Whitehead is zeer bekend en heeft een zeer uitgebreide pp geschreven, zijn bekendste werk is \cite{whitehead1929process}. Hij ziet creativiteit als het universele principe, maar schijnt niet erg duidelijk te zijn om uit te leggen hoe nieuwe relaties tussen processen tot stand komen bovendien speelt God een belangrijke rol als een soort van primaire invloed op alle processen.

Ik heb gelezen dat Whitehead een atomistic PP is. Hij begint met een soort van basis blokken waaruit processen zijn opgebouwd. Deze ziens wijze speelt mij niet erg aan. Ik volg liever Rescher die stelt dat processen uit processen zijn opgebouwd en dit tot in het oneindige doorloopt. Met vaste bouwstenen wordt het wijzigings principe van PP een beetje onderuit gehaald, immers zijn zijn overanderbare basis blokken. Een God zoals bij Whitehead is in deze visie niet nodig..ik heb ook liever dat deze niet nodg is omdat ik niet in een God geloof dieaan de basis van alles ligt.

Voorlopige auteurs zijn geselecteerd uit het artikel van Seibt.

Zijn er grensen aan welke processen kunnen ontstaan? Is het bijvoorbeeld mogelijk dat een planeet ontstaat in de form van een perfecte 3d driehoek?
Deze vraag is relevant om na te gaan of er grenzen zijn te stellen aan de ideen die kunnen ontstaan

Is het mogelijk dat er processen ontstaan die een planeet/wereld voortbrengen waar geen zwaartekracht heerst?

m.a.w. zijn er basis regels waar processen zich aan moeten houden of zijn de mogelijkheden onbegrenst? 
Rdb..te grote vraag, hoort bij een nieuw project.

Worden de mogelijkheden door de bestaande processen beperkt? Onstaat er een soort van Focus, hoe meer processen er zijn hoe groter de waarschijnlijkheid op een bepaalde uitkomst voor volgende processen? Of anders gezegd met nieuwe processen ontstaan er mogelijkheden voor nieuwe relaties maar misschien beperken processen elkaar ook wel.
Is bijvoorbeeld zwaarte kracht de uitkomst van een process of een bais regel waar processen zich aan moeten houden?
IS de big-bang the schepper van de eerste processen of het gevolg van een zeer lang of eewig lopend process. God was hier een handige verklaring geweest.
Wat zeggen de verschillende auteurs hier over?

Is bijvoorbeeld een mens een cluster van processen of een process?

Zijn er processen die niet kunnen bestaan? Worden dan halve processen gemaakt die direct ophouden te bestaan analoog bijvoorbeeld instabiele materie?
Wat is dan het toetsings criterium? Voorbeeld stel het God-IT proces genereerd een nieuw algortime, als dit niet compileert dan is het als het ware dood geboren en wordt verwijderd/vervalt. Sommige algoritmes creeren een software proces dat lang duurt, mischien met vele herhalingen (zie hier alsof het blijvende materie is), andere een process dat kort is en geen herhalingen bevat.

PP is opgebouwd op andere logica. Welke? Is het mogelijk deze na te bootsen in IT?

Er ontstaan veel nieuwe succesvolle processen. Is dat het gevolg van een goede begeleiding of van het grote aantal pogingen. Voor sommige verschijnsels zijn heel veel micro procesen nodig zoals bijvoorbeeld de werking van de zon, zwaartekracht, bewustzijn etc.

\section{fuzzy logic}
Enclopedia of creativity chaper AI
see also fuzzy set

IV. FUZZY LOGIC
Creative thinking is often required to solve illformed
or poorly formed problems. These are problems
where there is a great deal of uncertainty and difficulty
in making statements with complete precision.
Fuzzy logic has a role to play helping us to get to grips
with such problems. Fuzzy logic is multivalued (as opposed
to binary) logic. While classical logic holds that
everything can be expressed in binary terms—0 or 1,
black or white, yes or no, etc.—fuzzy logic permits
values between 0 and 1, shades of gray, and even partial
membership in a set. Moreover, when the approximate
reasoning of fuzzy logic is used with an expert
system, logical inferences can be drawn from imprecise
relationships.
Neural network technology can be used to produce
a fuzzy logic system which does not provide precise
answers and outcomes to every problem but which will
give reasonably correct estimations. A fuzzy logic system
attempts to categorize patterns according to other
patterns which it has ‘‘learned’’ and makes use of this
learning to suggest answers. This allows more fuzzy input
to be used in the neural network and greatly decreases
the learning time of such networks.
Fuzzy logic has many domestic applications. Home
appliances are common applications and fuzzy logic is
a part of the AI that helps to control such products as
enhanced washing machines, vacuum cleaners, and
air conditioners. some clothes washing machines automatically
adjust for load size and dirtiness of the
clothes. Some vacuum cleaners adjust their suction
power according to the volume of dust and the nature
of the floor. Fuzzy logic is also used to control passenger
elevators, cameras, automobile subsystems, and
smart weapons. Fuzzy logic seems to have potential for
development as far as creative thinking is concerned.
In conjunction with neural networks and expert systems,
fuzzy logic offers a means of improving the
power of such tools.

X. AI AND CREATIVITY
The debate as to whether a machine can think for
itself has some bearing on the role that AI can play in
the sphere of creativity.We can take the general debate
a step further and ask the question of whether a computer
can be creative or whether creativity is separable
from the human mind. Computers clearly can be programmed
to produce a credible, grammatically correct
English sentence given a set of rules and a database of
words. However, the ability to judge the creative value
of such a sentence, and if necessary to modify and
improve its creative value, seems to demand an extensive
base of experience and complex logic that is
so far unique to human cognition. Given these limitations
expert systems would seem to be most relevant to
the notion of providing support for idea processing
rather than its total automation. We might then ask
whether expertise on creativity can be captured and effectively
utilized within an interactive, user-controlled
computer support system.
Arguably, people are intrinsically much more creative
than even the best computer. Human experts can
reorganize information and use it to synthesize new
knowledge. An expert system, in contrast, is apt to
behave in a somewhat uninspired, routine manner.
Human experts handle unanticipated events by using
imaginative and novel approaches to problem solving,
including drawing analogies to situations in completely
different problem domains. Programs have not had
much success at doing this. All humans possess commonsense
knowledge which represents a very broad
spectrum of general knowledge about the world and
how it functions. This commonsense knowledge is immense
and represents a considerable challenge to program
designers. On first thought it might seem that
there does not appear to be any feasible way of programming
it into a computer, though attempts have
been made by the CYC project at MCC in Austin in this
direction. It is commonsense knowledge which makes
humans aware of what they do not know as well as
what they do know. This essential difference allows the
human to avoid wasting time searching for solutions
that are impossible and to concentrate only on finding
feasible solutions. A human would know, for example,
that you cannot put a camel through the eye of a needle
but a computer might search endlessly and in vain for
a solution unless it was programmed initially to appreciate
that this problem was an impossible one to solve.
Finally, human experts can appreciate the overall aspects
of a problem and conceptualize how it relates to
the central issue. Expert systems, however, tend to focus
on the problem itself and do not take account of
issues which are relevant but separate from the
problem.
Given our present knowledge about the processes
used to develop and enhance creativity it might be con-
cluded that there is no deep experience on the subject
but that there are a number of useful methods and
guidelines. A system that can guide the user in the application
of such methods might not seem to be worthy
of being referred to as an expert system—but this of
course may be a matter of opinion.
Both experts and expert systems must possess a large
repertoire of complex knowledge and be able to utilize
and operationalize it within a problem situation. Expert
systems should offer advice which can be acted
upon and not merely prescriptions for how users can
arrive at their own conclusions. Moreover, an expert
system should also be able to explain its own reasoning
as to how it reached its conclusions and advice to enable
the user to assess the value of the advice proffered.
An expert system capable of acting as a creative consultant
would have to be able to produce novel problem
definitions and be able to respond to human reactions
to these definitions with further meaningful
comments, explanations, or modifications. The implication
of this would be that such a system should possess
an experience base as extensive as that of a highly
experienced human adult. Moreover, a potentially useful
knowledge base would not have to be limited to
any particular domain since idea-generating methods
themselves are premised on the notion that creativity
requires the breaking down of overly constraining categories
of knowledge and finding previously undefined
associations.
While automation or expert systems seem to offer a
basis of assisting the creative process, at the present time
the nature of the creative process hinders the effective
use of the pure forms of both these approaches. To
give some idea of the potential problems involved, the
CYC (from enCYClopedia) Project at MCC in Austin, a
10-year project, was begun in 1984 and aimed to enumerate
tens of millions of commonsense facts that will
ultimately ‘‘add up’’ to logical intelligence. The goal is
a system that can understand and speak ordinary language,
and detect violations of common sense as readily
as humans can. The total number of ‘‘rules’’ required for
this was subsequently revised upward by a factor of ten
(to 20–40 million), and extended the time needed by
another 10 years.
Case-based reasoning seems to offer one of the best
short-term prospects for producing suitable vehicles
to assist in creative problem solving. The database in
this instance might contain documented problem solving
case histories across many different domains. Such
a database might be accessed in a way that it provides
insights for problem solving through analogical
reasoning.


\section{Section Overview}

-1 de big bang, waar komt deze vandaan?
http://www.big-bang-theory.com/
Geen duidelijkheid over het begin, in het begin was er niks, het universum en tijd is uit de big bang voorgekomen.
Geen ontploffing maar een expansie, senl op blazende ballon

Whitehead
Process

Whitehead concept of creativity and modern science




Vragen beantwoorden:
waarom pp? als model van de cosmos voor deze thesis?
wat is PP

Wat is een process, micro proces macro proces
Wat is de drijvende kracht die nieuwe processen doet ontstaan
waar komen nieuwe processen vandaan
Aan welke eisen moet een process voldoen
Hoe ontstaan nieuwe relaties tussen processen
wat verstaat men onder creativiteit
wat is het concept van een (nieuw) idee

Begin in deze sectie met een algemene verklaring wat PP is, gebasseerd op een SEP artikel van Johanna Seibt. Er zijn veel verschillende stromingen,waar de wat oudere een totaal oplossing ptoberen te brengen zijn de huidige vooral specialisaties die bijdragen aan een groter geheel.
Zet PP af tegen het bestaande westerse paradigma van de materie-philosopie en licht toe waarom pp zo geschikt is voor de verklaring van creativiteit. Omdat creativiteit over vernieuwing gaat en dat het basis principe van pp is.

Geef een overzicht van N>3 (waarom 3?) verschillende perspectieven van verschillende auteurs op creativiteit. Whitehead is zeer bekend en heeft een zeer uitgebreide pp geschreven, zijn bekendste werk is \cite{whitehead1929process}. Hij ziet creativiteit als het universele principe, maar schijnt niet erg duidelijk te zijn om uit te leggen hoe nieuwe relaties tussen processen tot stand komen bovendien speelt God een belangrijke rol als een soort van primaire invloed op alle processen.

Ik heb gelezen dat Whitehead een atomistic PP is. Hij begint met een soort van basis blokken waaruit processen zijn opgebouwd. Deze ziens wijze speelt mij niet erg aan. Ik volg liever Rescher die stelt dat processen uit processen zijn opgebouwd en dit tot in het oneindige doorloopt. Met vaste bouwstenen wordt het wijzigings principe van PP een beetje onderuit gehaald, immers zijn zijn overanderbare basis blokken. Een God zoals bij Whitehead is in deze visie niet nodig..ik heb ook liever dat deze niet nodg is omdat ik niet in een God geloof dieaan de basis van alles ligt.

Voorlopige auteurs zijn geselecteerd uit het artikel van Seibt.

Zijn er grensen aan welke processen kunnen ontstaan? Is het bijvoorbeeld mogelijk dat een planeet ontstaat in de form van een perfecte 3d driehoek?
Deze vraag is relevant om na te gaan of er grenzen zijn te stellen aan de ideen die kunnen ontstaan

Is het mogelijk dat er processen ontstaan die een planeet/wereld voortbrengen waar geen zwaartekracht heerst?

m.a.w. zijn er basis regels waar processen zich aan moeten houden of zijn de mogelijkheden onbegrenst? 
Rdb..te grote vraag, hoort bij een nieuw project.

Worden de mogelijkheden door de bestaande processen beperkt? Onstaat er een soort van Focus, hoe meer processen er zijn hoe groter de waarschijnlijkheid op een bepaalde uitkomst voor volgende processen? Of anders gezegd met nieuwe processen ontstaan er mogelijkheden voor nieuwe relaties maar misschien beperken processen elkaar ook wel.
Is bijvoorbeeld zwaarte kracht de uitkomst van een process of een bais regel waar processen zich aan moeten houden?
IS de big-bang the schepper van de eerste processen of het gevolg van een zeer lang of eewig lopend process. God was hier een handige verklaring geweest.
Wat zeggen de verschillende auteurs hier over?

Is bijvoorbeeld een mens een cluster van processen of een process?

Zijn er processen die niet kunnen bestaan? Worden dan halve processen gemaakt die direct ophouden te bestaan analoog bijvoorbeeld instabiele materie?
Wat is dan het toetsings criterium? Voorbeeld stel het God-IT proces genereerd een nieuw algortime, als dit niet compileert dan is het als het ware dood geboren en wordt verwijderd/vervalt. Sommige algoritmes creeren een software proces dat lang duurt, mischien met vele herhalingen (zie hier alsof het blijvende materie is), andere een process dat kort is en geen herhalingen bevat.

PP is opgebouwd op andere logica. Welke? Is het mogelijk deze na te bootsen in IT?

Er ontstaan veel nieuwe succesvolle processen. Is dat het gevolg van een goede begeleiding of van het grote aantal pogingen. Voor sommige verschijnsels zijn heel veel micro procesen nodig zoals bijvoorbeeld de werking van de zon, zwaartekracht, bewustzijn etc.


\section{Content}
\section{Conclusion}

%------------------------------------------------------------------------------
%------------------------------------------------------------------------------
\chapter{The concept of creativity}



-Laat de verschillende concepten in filosofie zien.
-Laat de verschillende verklaringen, uitleg over de werking in de psychology zien.
-Leg uit wat het begrip creativiteit betekend in de PP en schets een mogelijke werking. Gebruik daarvoor de verschillende PP auteurs, zie boek rescher.

\section{Provisional section}

\section{Conclusion}

%------------------------------------------------------------------------------
%------------------------------------------------------------------------------
\chapter{The concept of IT-software}
Wat bedoel ik met it-technology:

- programming environments for making new programs
- tools that support engineering
- tools that support ideation specified to technology ideationsubsection{}
- visualisation, prototyping
- build a model of PP that creates new processes in de creatie way?

\section{Provisional section}
\section{Conclusion}

%------------------------------------------------------------------------------
%------------------------------------------------------------------------------
\chapter{What is the role of Process Philosophy(What is the relation between PP and creativity)}

In deze sectie: Wat is de relatie tussen PP en creativiteit/TI.

\section{Provisional section}
\section{Conclusion}

%------------------------------------------------------------------------------
%------------------------------------------------------------------------------
\chapter{How do I think Process Philosophy and IT are related (what is the relation PP and IT)}
\section{Provisional section}



IT lijkt in zekere zin op het process model. Als we de hardware buiten beschouwing laten dan bestaat IT vooral uit processen die onderling met elkaar in relatie staan. De structuur bestaat uit de code regels.

Is het denkbaar dat processen spontaan ontstaan? Zou het mogelijk zijn om de pp te simuleren?
Nu waarschijlijk nog niet.
De wereld zonder it veranderd steeds als gevolg van nieuwe processen.
\todo{rdb de wereld met it dus niet?} Als ik de geschiedenis bekijk dan hebben mensen, ook een process neem ik aan, een grote impact op de ontwikkeling doordat zij, meer dan dieren bv, gericht ontwikkelen, dus processen creeren. Een process creer gericht andere processen.
Dat doet een process (verzameling van processen) zoals de mens in zijn eigen belang. Dus en process zal in het belang van zijn eiegen process een bepaalde invloed proberen uit te oefenen op het creeren van andere processen. Als de mens, of een dier of een soort, wegvalt dan is de kans dat nieuwe processen ontstaan die hem/haar/het gebaat zouden hebben op zijn minst afgenomen.


\section{Conclusion}

%------------------------------------------------------------------------------
%------------------------------------------------------------------------------
\chapter{How does IT influence creativity explained with a PP view}


Elk niew process bied nieuwe mogelijkheden zo ook IT. Door It zijn nieuwe processen op te bouwen die relaties met het IT-process hebben? Maar wat is zo bijzonder aan IT?

Misschien, als ik geen goed link kan vinden, draai ik de zaak om en probeer een voorstel te doen voor een simulatie van PP, het genereren, verbinden, draaien en afsterven van processen middels software

\section{Provisional section}

\section{Conclusion}

%------------------------------------------------------------------------------
%------------------------------------------------------------------------------
\chapter{Conclusion}

\begin{verbatim}
IT-> easier/faster access to existing ideas and concepts (patents and copy rights are obstructions)
IT-> virtual world, free building tools (Java), low machine investment (PC, internet access) -> 
high user participation, small companies, building for fun, free non commercial usage (Hippel)
IT-. High interaction grade (smartphone, youtube, facebook, twitter) -> fast dispersion of ideas
IT-> no contemplation, overwelming information supply changing fast
IT-> new, people have to learn how to use it?
IT-> process philosophy -> every thng is connected via internet WIFI etc, concept of rocess comes close to the workings of a IT program. Relations can be process automatically-> big data?
IT-> process philosophy -> would it be possible to build, immitated, reality in IT processes? and predict the outcome? In fact this is what we do with the weather system for example. Is not IT a model cennecting to PP? Data = matter, process and relations.
Only the program does not change itself-> processes and relations are fixed. People build new programs (Processes) make relations and store data. It is possible to model this with physical things but much harder to make connections. Software is very easy to change compared to hardware, unless changes have go deep in protocols that could influence existing programs.

IT-> Ideation support systems, visualizing concepts, idea generation

IT-> maybe discuss if computers can be creative? Or skip because not part of the research?

It-> Could be seen as replacing the brain wherease machines replace the body? Is that dualism?
\end{verbatim}

\section{Provisional section}

%------------------------------------------------------------------------------
%------------------------------------------------------------------------------
\bigskip
%------------------------------------------------------------------------------
\bibliography{psts-master-thesis}
\bibliographystyle{apacite}
%------------------------------------------------------------------------------
\end{document}

