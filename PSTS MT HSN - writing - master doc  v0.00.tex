%%% Local Variables: 
%%% mode: latex
%%% TeX-master: t
%%% End: 
%\documentclass[a4paper]{report}
\documentclass[a4paper]{Thesis}
% for report \usepackage{suthesis-2e}
\usepackage[colorlinks,citecolor=black]{hyperref}
\usepackage{apacite}
\usepackage[english]{babel}
\usepackage[pdftex]{graphicx}
\usepackage{subfig}
\usepackage{verbatim}
%ToDoNotes
\usepackage{todonotes}
%\usepackage{cancel}
\usepackage{soul}

\usepackage{vhistory}
\begin{document}
% Start of the revision history table
%\begin{versionhistory}
%	\vhEntry{0.05}{20.07.2015}{HSn}{created, Structure and Introduction}
%	\vhEntry{0.06}{27.07.2015}{HSn}{Comment on introduction from SSn and Rdb}
%\end{versionhistory}

%\begin{titlepage} 
\begin{center}
		
% Upper part of the page. The '~' is needed because \\ % only works if a paragraph has started. \includegraphics[width=0.15\textwidth]{./logo}~\\[1cm]
		
%\textsc{\Large Creativity and ICT technology}\\[1.5cm]
		
%\textsc{\Large Final year project}\\[0.5cm]
		
% Title 
\HRule \\[0.4cm] { \huge \bfseries Creativity and ICT \\[0.4cm] }
\HRule \\[0.4cm] { \large \bfseries A process philosophical view on the influence of ICT on technological ideation \\[0.4cm] }
\HRule \\[0.4cm] { \large \emph{Author:} Harry Snijders \\[0.4cm] }

\vfill		
% Author and supervisor 
%\noindent 
\begin{minipage}[b]{\textwidth} 
	\begin{flushleft} 
		\large Thesis for the degree of Master of Science \\[0.4cm]

\begin{tabular}{ll}
	
		\large \emph{Studentnumber:}	 & S1143972	\\
		\large \emph{Place:} 			 & Oldenzaal	\\
		\large \emph{Date:} 			 & \today \\
		\large \emph{Institution:} 		 & University of Twente \\
		\large \emph{Department:} 		 & Department of Philosophy  \\
		\large \emph{Programme:} 		 & Philosophy of Science, Technology and Society \\
		\large \emph{Specialization:} 	 & Technology and the Human Being \\
		\large \emph{First supervisors:} & Dr. Johnny Hartz Søraker and Dr. Aimee van Wynsberghe \\
		\large \emph{Second supervisor:} & Prof. Dr. Philip Brey \\
\end{tabular} 		

	\end{flushleft} 
\end{minipage}
		
\end{center} 
\end{titlepage}


%content 
\bigskip
\bigskip
\tableofcontents
%\listoffigures

%PREFACE AND ACKNOWLEDGEMENT

%\begin{abstract}
%\end{abstract}
%\bigskip
%\bigskip

%\providecommand{\keywords}[1]{\textbf{\textit{keywords---}} #1}
%\keywords{}

%\bigskip

%\begin{flushright} 
% But in the real world it is more important that a proposition be interesting than that it be true.
% (A.N. Whitehead)
%\end{flushright} 


%\chapter{Introduction}
\chapter{dit is een test van het include command}
\include{PSTS MT HSN - writing - ch Introduction  v0.00.tex}




%\chapter{Process Philosophy (What is process Philosophy)}
\include{PSTS MT HSN - writing - ch process philosophy  v0.00.tex}
\chapter{The concept of creativity}

-Laat de verschillende concepten in filosofie zien.
-Laat de verschillende verklaringen, uitleg over de werking in de psychology zien.
-Leg uit wat het begrip creativiteit betekend in de PP en schets een mogelijke werking. Gebruik daarvoor de verschillende PP auteurs, zie boek rescher.

\section{Provisional section head}

\section{Conclusion}

%------------------------------------------------------------------------------
%------------------------------------------------------------------------------
\chapter{The concept of IT-software}
Wat bedoel ik met it-technology:

- programming environments for making new programs
- tools that support engineering
- tools that support ideation specified to technology ideationsubsection{}
- visualisation, prototyping
- build a model of PP that creates new processes in de creatie way?

\section{Provisional section head}
\section{Conclusion}

%------------------------------------------------------------------------------
%------------------------------------------------------------------------------
\chapter{What is the role of Process Philosophy(What is the relation between PP and creativity)}

In deze sectie: Wat is de relatie tussen PP en creativiteit/TI.

\section{Provisional section head}
\section{Conclusion}

%------------------------------------------------------------------------------
%------------------------------------------------------------------------------
\chapter{How do I think Process Philosophy and IT are related (what is the relation PP and IT)}
\section{Provisional section}



IT lijkt in zekere zin op het process model. Als we de hardware buiten beschouwing laten dan bestaat IT vooral uit processen die onderling met elkaar in relatie staan. De structuur bestaat uit de code regels.

Is het denkbaar dat processen spontaan ontstaan? Zou het mogelijk zijn om de pp te simuleren?
Nu waarschijlijk nog niet.
De wereld zonder it veranderd steeds als gevolg van nieuwe processen.
\todo{rdb de wereld met it dus niet?} Als ik de geschiedenis bekijk dan hebben mensen, ook een process neem ik aan, een grote impact op de ontwikkeling doordat zij, meer dan dieren bv, gericht ontwikkelen, dus processen creeren. Een process creer gericht andere processen.
Dat doet een process (verzameling van processen) zoals de mens in zijn eigen belang. Dus en process zal in het belang van zijn eiegen process een bepaalde invloed proberen uit te oefenen op het creeren van andere processen. Als de mens, of een dier of een soort, wegvalt dan is de kans dat nieuwe processen ontstaan die hem/haar/het gebaat zouden hebben op zijn minst afgenomen.


\section{Conclusion}

%------------------------------------------------------------------------------
%------------------------------------------------------------------------------
\chapter{How does IT influence creativity explained with a PP view}


Elk niew process bied nieuwe mogelijkheden zo ook IT. Door It zijn nieuwe processen op te bouwen die relaties met het IT-process hebben? Maar wat is zo bijzonder aan IT?

Misschien, als ik geen goed link kan vinden, draai ik de zaak om en probeer een voorstel te doen voor een simulatie van PP, het genereren, verbinden, draaien en afsterven van processen middels software

\section{Provisional section}

\section{Conclusion}

%------------------------------------------------------------------------------
%------------------------------------------------------------------------------
\chapter{Conclusion}

\begin{verbatim}
IT-> easier/faster access to existing ideas and concepts (patents and copy rights are obstructions)
IT-> virtual world, free building tools (Java), low machine investment (PC, internet access) -> 
high user participation, small companies, building for fun, free non commercial usage (Hippel)
IT-. High interaction grade (smartphone, youtube, facebook, twitter) -> fast dispersion of ideas
IT-> no contemplation, overwelming information supply changing fast
IT-> new, people have to learn how to use it?
IT-> process philosophy -> every thng is connected via internet WIFI etc, concept of rocess comes close to the workings of a IT program. Relations can be process automatically-> big data?
IT-> process philosophy -> would it be possible to build, immitated, reality in IT processes? and predict the outcome? In fact this is what we do with the weather system for example. Is not IT a model cennecting to PP? Data = matter, process and relations.
Only the program does not change itself-> processes and relations are fixed. People build new programs (Processes) make relations and store data. It is possible to model this with physical things but much harder to make connections. Software is very easy to change compared to hardware, unless changes have go deep in protocols that could influence existing programs.

IT-> Ideation support systems, visualizing concepts, idea generation

IT-> maybe discuss if computers can be creative? Or skip because not part of the research?

It-> Could be seen as replacing the brain wherease machines replace the body? Is that dualism?
\end{verbatim}

\section{Provisional section}

%------------------------------------------------------------------------------
%------------------------------------------------------------------------------
\bigskip
%------------------------------------------------------------------------------
\todo{Denk er aan dat automatische referenties van bv scholar niet altijd volledig zijn en ook van het verkeerde type kunnen zijn bv aticle ipv book}
\bibliography{psts-master-thesis}
\bibliographystyle{apacite}
%------------------------------------------------------------------------------
\end{document}

