\chapter{Process Philosophy (What is process Philosophy)}

\section{Provisional Content}
\paragraph{What is the problem with the common western philosophical view of reality?}
\paragraph{What is the process philosophical view of reality}
\paragraph{Basic Doctrines}
\paragraph{How does process philosophy help to understand creativity?}
\paragraph{Conclusion}

\section{Provisional section head}


%one of the principles on which a belief or theory is based:
%It is a tenet of contemporary psychology that an individual's mental health is supported by having good social networks.
%An idea or theory on which a statement or action is based:
%[+ that] They had started with the premise that all men are created equal.
%The research project is based on the premise stated earlier.


\paragraph{What is the problem with the common western philosophical view of reality?}
De wetenschappelijke methode is gabasserd op het paradigma van de matery fysica.
Rap heeft daar een stuk over geschreven.
In de materie fysica en afgeleide logica is geen plaats voor feiten die niet op een duidelijke oorzaak gevolg gebasseerd zijn zoals creativiteit en bijvoorbeeld ervaringen of gevoelens.
Ik denk dat de sociale wetenschappen daar beter op de plaats zijn omdat bv het gedrag van een maatschappij ook niet is te beredeneren op basis van een rechtstreekse relatie oorzaak en gevolg.







Western metaphysics has long been obsessed with describing reality as an assembly of static individuals whose dynamic features are either taken to be mere appearances or ontologically secondary and derivative. 

The common world view in the western world is that of a world constructed out of matter, an Aristotelian heritage. In this metaphisics of substance, entities are constructed out of small, standard, building blocks that last for ever. 
% Quantum physics problem shows different
The static existance of the entities is the primairy ontology and the dynamic features of these entities are secondairy and derivate.
Entities stand on them selves

The scientifical method is tuned for this worldview and the 

Knowledge comes from science
Scientifical method very fruitfull but fails to describe phenomena like creativity
Logic, descartes object-subject, see Rapp in ch 5
labratory, reductionist, isolating and testing, not as a combination (interwined verbeek)
paradigma, limits seeing

Modern science, by contrast, seem to \cite[chapter 5]{rapp1990whitehad}


The static view on the world is reflected in the current scientific method.
\todo{Friedrich Rapp creativity and modern science}
fruitfull (oh ja, wat als we anders hadden gekeken? Dat zou betekenen dat we de meest effieciente weg volgen?)

Scientific method, take subject of research in isolation-> taken out of the rich natural environment-> environment should be included in the subject of research because it will influence it behaviour.

Creativity, the generation of novelty, conflicts with this idea of endurance and is therefore more difficult to explain in a world that exists of fixed and ever lasting building blocks.
\todo{see Freidrich Rapp ch 5 in witeheads metaphysics of creativity about how modern sience is not capable of dealing with principles like creativity because of the scientific method}
In a process view of the world, change and creativity is a basic principle. The world is no longer constructed out of matter but consists of processes. Micro processes, related in myriad ways with other micro processes, build up to macro processes, that because of their recurred characteristic construct the world as we experience it. In this world view every thing is explained as a process, also we humans. 
\cite[ch 5]{rapp1990whitehead}
the atomistic view of a mere mechanical aggregation of parts

\paragraph{ISABELLE STENGERS review by desmet}
Deleuze holds that
philosophy is not a perpetual discussion among scholars engaged in contemplation
and reflection, which aims at an all-encompassing and consensual vision of the world.
Rather, it is a mode of creative thought, involving a plurality of problem-driven
conceptual constructions
\cite{keylist}
@article{desmet2011thinking,
	title={Thinking With Whitehead},
	author={Desmet, Ronny},
	journal={Process Studies},
	volume={40},
	number={1},
	pages={179--186},
	year={2011}
	}

She reveals that she was first incited to think with Whitehead,
because as a young chemist-cum-philosopher she was troubled by the catastrophic
indifference of scientists in relation to what they judge to be ‘non-scientific.’ Instead,
she shared Whitehead’s concern as regards to the bifurcation of nature into a ‘true’
world of science and an ‘illusory’ world of non-scientific experience. However,
Thinking with Whitehead is not merely a matter of Stengers’ youthful affinity with
Whitehead’s work. Stengers claims that today, more than ever, in light of the
prevalence of scientific reductionism, we may all share Whitehead’s initial concern.
For Stengers, “Whitehead belongs to our epoch because he asks a question that is
ours” (12).

In Process and Reality everything is
“redone” again. The central problem is the problem of becoming, that is, the
challenge of unifying endurance and originality, tradition and innovation, habit and
adventure, conformity and autonomy, determinism and self-determination. The
unifying concept is ‘actual entity’ as that which is causa sui, meaning, that which
decides, not what its causes are, but how it takes them into account. And the ultimate
category is ‘creativity,’ the principle that the many become one and are increased by
one.
	
	
There is no doubt that Thinking with Whitehead exemplifies Deleuze’s dictum that
“all concepts are connected to problems without which they would have no meaning”
(What is Philosophy? 16).


creativity is coined by Whitehead 
\cite{steven meyer}
http://muse.jhu.edu/login?auth=0&type=summary&url=/journals/configurations/v013/13.1meyer.pdf
24-07

\cite{Ford}
@book{ford1984emergence,
	title={The Emergence of Whitehead's Metaphysics, 1925-1929},
	author={Ford, Lewis S},
	year={1984},
	publisher={SUNY Press}
}

\section{stenger the book}
@article{stengers2011thinking,
	title={Thinking with Whitehead: a free and wild creation of concepts},
	author={Stengers, Isabelle},
	year={2011}
	}
p54
It is an exhibition of the process of nature that each duration happens
and passes. The process of nature can also be termed the passage of nature
(CN, 54 ) .
Insofar as i t happens and passes, each duration exhibits the passage of
nature, but no duration provides a pri vi leged testimony with regard to
this passage. Every Whitehead ian passage is qualified insofa r as it contains
other durations and is contained in other durations. Thus, the duration of
the experience of the customs-officer-in-front-of-whom-a-traveler-passes
could have been broken down into shorter durations

p72
When Whitehead beca me a metaphysician,
conscious experience became a creature of passage, which itself has
become creativity.

ch16
IN ALL PHILOSOPHICAL THEORY there is an ultimate which is actual
in virtue of its accidents. It is only then capable of characterization
through its accidental embodiments, and apart from these
accidents is devoid of actuality. In the philosophy of organism this ultimate
is termed "creativity " ( PR, 7 ) .

p256 PR21-22 over creativity and novelty
" Creativity," "many," "one " are the ultimate notions involved in the
meaning of the synonymous terms.

%???
%Accidental
%Aristotle made a distinction between the essential and accidental properties of a thing. For example, a chair can be made of wood or metal, but this is accidental to its being a chair: that is, it is still a chair regardless of the material from which it is made.[2] To put this in technical terms, an accident is a property which has no necessary connection to the essence of the thing being described.[3][4][5]
%In modern philosophy, an accident (or accidental property) is the union of two concepts: property and contingency. In relation to the first, an accidental property (Greek symbebekos)[7] is at its most basic level a property. The color "yellow", "high value", "Atomic Number 79" are all properties, and are therefore candidates for being accidental. On the other hand, "gold", "platinum", and "electrum" are not properties, and are therefore not classified as accidents.
%
%There are two opposed philosophical positions that also impact the meaning of this term:
%Anti-Essentialism (associated with Willard Van Orman Quine) argues that there are no essential properties at all, and therefore every property is an accident.
%Modal Necessitarianism (associated with Saul Kripke), argues for the veracity of the modal system "Triv" (If P is true, then P must be true). The consequence of this theory is that all properties are essential (and no property is an accident).


%proposition
%Rhetoric. a statement of the subject of an argument or a discourse, or of the course of action or essential idea to be advocated.
%Logic. a statement in which something is affirmed or denied, so that it can therefore be significantly characterized as either true or false.
these oppositions and this alignment are instrumental in emphasizing that it is
more important for Whitehead in Process and Reality that a proposition elicits
interest, and entails a future-oriented creation of novelty, than that it be true.
For
example, while distancing the concept of proposition in Process and Reality from
“verification in the logical empiricist sense” (406), Stengers brings it closer to the
pragmatic verification criterion, and writes that “the true verification of a proposition
concerns its consequences, or the new possibilities that it makes conceivable […] the
future it makes possible to envisage” (20-21).
Stengers departs here “from any
nostalgia for the verification processes designating what physicists or chemists call
‘objectivity’” (439) and she holds “that the truth, or importance, of an idea is nothing
other than its process of verification, the creative process in which the eventual
consequences of these ideas are produced and put to the test” (438).

Stengers
argues that the God of Process and Reality, like the God of Science and the Modern
World, has “nothing in common with a God of religion, and particularly with the God
of the Christians” (220), but solely “answers to a need produced by metaphysics”
(225). Consequently, she argues for the need for “a severe sorting process between
religious statements and metaphysical statements, with the former testifying to what
human consciousness has made itself capable of at a given epoch, and the latter
accepting themselves as obliged only by the imperative of coherence” (394).
Stengers’ presentation of Process and Reality’s concept of God as the outcome of a
purely metaphysical construction so as to deal with the problem of becoming is one of
the strengths of Thinking with Whitehead.

\section{shaviro}
@book{shaviro2012without,
	title={Without Criteria: Kant, Whitehead, Deleuze, and Aesthetics},
	author={Shaviro, Steven},
	year={2012},
	publisher={MIT press}
}
No book is ever written in a vacuum; and my intellectual indebtedness
in the case of Without Criteria is especially great. My book is largely written in
the margins of Isabelle Stengers’s magnificent Penser avec Whitehead. With this
text, Stengers both made Whitehead accessible to me for the first time, and
opened up the question of Whitehead’s affinity with Deleuze

XV
(such as questions about commodity fetishism, about immanence
and transcendence, about the role of autopoietic or self-organizing systems,
and about the ways that “innovation” and “creativity” seem to have
become so central to the dynamics of postmodern, or post-Fordist, capitalism).

p22
In Process and Reality, “morphological description is replaced by
description of dynamic process. Also Spinoza’s ‘modes’ now become the
sheer actualities; so that, though analysis of them increases our understanding,
it does not lead us to the discovery of any higher grade of reality” (7).
For Whitehead, there is nothing besides the modes, no unified substance
that subsumes them—not even immanently. Even God, Whitehead suggests,
is natura naturata as well as natura naturans, “at once a creature of creativity
and a condition for creativity. It shares this double character with all creatures”
(31). In itself, every individual “actual entity satisfies Spinoza’s notion
of substance: it is causa sui ” (222). The modes, affections, or actual occasions
are all there is.6

ch4	
Both Whitehead and Deleuze place creativity, novelty, innovation, and the
new at the center of metaphysical speculation. These concepts (or at least these
words) are so familiar to us today—familiar, perhaps, to the point of nausea—
that it is difficult to grasp how radical a rupture they mark in the history of
Western thought. In fact, the valorization of change and novelty, which we so
take for granted today, is itself a novelty of relatively recent origin. Philosophy
from Plato to Heidegger is largely oriented toward anamnesis (reminiscence)
and aletheia (unforgetting), toward origins and foundations, toward the past
rather than the future. Whitehead breaks with this tradition when he designates
the “production of novelty” as an “ultimate notion,” or “ultimate metaphysical
principle” (1929/1978, 21). This means that the new is one of those
fundamental concepts that “are incapable of analysis in terms of factors more
far-reaching than themselves” (Whitehead 1938/1968, 1). Deleuze similarly
insists that the new is a value in itself: “the new, with its power of beginning
and beginning again, remains forever new.” There is “a difference . . . both formal
and in kind” between the genuinely new and that which is customary and
established (Deleuze 1994, 136).

For both Whitehead and Deleuze, novelty is the highest
criterion for thought; even truth depends on novelty and creativity, rather than
the reverse. As for creativity itself, it appears “that Whitehead actually coined
the term—our term, still the preferred currency of exchange among literature,
science, and the arts . . . a term that quickly became so popular, so omnipresent,
that its invention within living memory, and by Alfred North Whitehead of
all people, quickly became occluded” (Meyer 2005, 2–3).

p92
Of course, contemporary biology is not prone to speak of final causes,
or to define life in the way that Whitehead does. According to the mainstream
neo-Darwinian synthesis, “pure physical inheritance,” when combined with
occasional random mutation and the force of natural selection, is sufficient to
account for biological variation. On this view, innovation and change are not
primary processes, but adaptive reactions to environmental pressures. Life is
essentially conservative: not oriented toward difference and novelty as Whitehead
would have it, but organized for the purposes of self-preservation and
self-reproduction. It is not a bid for freedom, but an inescapable compulsion.
The image of a “life force” that we have today is not anything like Bergson’s
élan vital; it is rather a virus, a mindlessly, relentlessly self-replicating bit of
DNA or RNA. Even the alternatives to the neo-Darwinian synthesis that are
sometimes proposed today—like Maturana and Varela’s theory of autopoiesis
(1991), Stuart Kauffman’s exploration of complexity and self-organizing systems
(2000), Lynn Margulis’s work on symbiosis (Margulis and Sagan 2002),
James Lovelock’s Gaia theory (2000), and Susan Oyama’s developmental systems
theory (2000)—share mainstream biology’s overriding concern with
the ways that organisms maintain homeostatic equilibrium in relation to
their environment and strive to perpetuate themselves through reproduction.
It would seem that organic beings only innovate when they are absolutely
compelled to, and as it were in spite of themselves.

The context of this “return” to beauty is an exceedingly disagreeable one. On
the one hand, beauty today has become a mere adjunct of advertising and
product design—just as “innovation” has become a managerial buzzword,
and creativity has become “a value in itself ” for the corporate sector (Thrift
2005, 133). There’s scarcely a commodity out there that doesn’t proclaim its
beauty as a selling point, together with its novelty and the degree of creativity
that ostensibly went into developing it.

After all, Whitehead’s
great topic is precisely the manner in which something radically new can
emerge out of the prehension of already existing elements. Innovation is all a
matter of “ ‘subjective form,’ which is how [a particular] subject prehends [its]
datum” (1929/1978, 23). Whitehead’s aesthetics, with its intensive focus on
this how, takes on a special urgency in a culture, such as ours, that is poised on
the razor’s edge between the corporate ownership, and interminable recycling,
Chapter 6
158 159
of “intellectual property,” on the one hand, and the pirating, reworking, and
transformation of such alleged “property,” often in violation of copyright laws,
on the other.

Whitehead warns us that “the chief error in philosophy is overstatement.
The aim at generalization is sound, but the estimate of success is exaggerated”
(1929/1978, 7).

\paragraph{Creativity as a key concept..rapp ch 5}
\cite[ch 5]{rapp1990whitehead}
Ultimate universal categories: creativity, many, one
philosophy of organism
mechanical 
prehension: PHILOSOPHY
an interaction of a subject with an event or entity which involves perception but not necessarily cognition.
factors expressed by: 'actual entity', 'prehension', 'nexus'
actual entity replaces substance




\paragraph{What is the process philosophical view of reality}
See Reschner introduction en SEP art \cite{Rescher-2012-sep}
Seibt SEP entry \cite{Seibt-2013-sep}


Process philosophy, or process theology, or simply process thought, is a tradition in philosophy that is in progress. Different scholars contribute different views but there are some commonalities.
 
In the traditional western metaphysics, started with Aristotle and build on ideas of Descartes for dualism and Newton for a mechanical world view, reality is based on a description of entities that have a persistent character. Change is represented as an alteration of the attributes of these persisting entities and is of a secundary ontology. The primairy ontology is the entity that is made out of of ever lasting substance.

I this methaphisics ???


This metaphisic is based on a cause and effect principle that.

its substance, The basic principle is persistance, reality as we experience it, is an reality that is already exsiting.

In the methaphysics of process philosophie], change is the basic principle. The reality as we experience it is a reality that is becoming, being is becoming.
The continuously going on and coming about of reality is explained by the working of processes. The temporally stable and recurrend aspects what the substance philosophy explains by the concept of matter is in process philosophy explained with the regular behavior of a dynamic system that is the result ot the interaction of processes.

Processes can be grouped to macro process. Processe are related to other existing processes (in hoeverre?).
In the atomistic view of pp the reduction of processes is finit and ends at the categorie of basic processes. It is however also possible to see processes as construction of processes in to infinity, and deny the existance of basic categories of processes.

The acceptance of a basic categorie of processes brings with it a limitation in novelty because agregate processes inheret characteristics of the basic processes. Compare this with the evolution theory where new species depend on the genetic structure of their predecessor.
???If I accepte the reduction into inifinite view, there is no limitation in what is possible.

creative activity (transforming potentiality into actuality)

Voorbeeld met de begrippen van Whitehead:
Process
Actueal-entity
Creativity
Concrenesse
Prehension
Misschien ook de categorien van zijnden.


???With the process as a universal building block process philosophy tries to overcome the object subject dualism.

 and in this way the body is a groupexist of the combination of ther processes

Because everything in pp is expained with processe
With the explanation of y is explained with processesm




The modern scientific method is 
in this metaphysics is presented as 


has long been based on a description of reallity existing of 
is based on entities
Process philosophy is a metaphysical endeavour that  explanation of the phenomena in the world based on the developmental nature of reality where as the dominant, western, explenation of the world is based on a static reality. 
Process philosophy share the idea that to understand the world and answer the basic philosophical questions it is best to understand the world as an ever changing reality, is not what is, but what is becoming.

There is not one process philosophy view but there are different views.
In the field of process philosophy Alfred North Whitehead and Charles Hartshorn are seen as the most important contributers to the contempory view of process philosophy. 
But thes ar by far not the only scholars. Especially in the North America's, process philosophie had and has a large community of philosophers. 

Nichlas Rescher has writen a clear introduction into PP \cite{rescher1996process}.

Process philosophers try to find one principle of explanation for al questions and in this the western  substance philosophy has failed, (Descartes gave us dualism).
With one and get rid of the object-subject dichotomy. For this they define one basic principle that lies behind all the entities of reality. 
In the substance philosophy not everything is to explain from the basic unit matter. In science this is neglected.
Newton mechanica -> quentum physics.

becoming and changing over static being. The dominant western worl view is that of a world consisting of entities that consist of matter that is constructed from small building blocks. The idea is that these blocks exist for ever.

\paragraph{Basic Doctrines}

(http://www.ctr4process.org/about/what-process-thought)

The principle of process philosophy ranges back to the Greek Heraclitus of Ephesus (born ca. 560 B.C.E.) who is commonly recognized as the founder of the process approach \cite{Seibt-2013-sep}. 
but there are two contemporary philosophers that are most associated with the term process philosophy namely Alfred North Whitehead (1861-1947) and Charles Hartshorne (1897-2000).
Next to these are scholars like





\paragraph{How does process philosophy help to understand creativity?}


\paragragph(What is creativity)
beschrijving?
Algemene omschrijving PP
Wat is een process
Wat is creativiteit


\section{Content}
\section{Conclusion}
