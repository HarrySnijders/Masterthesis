%\documentclass[a4paper]{report}
\documentclass[a4paper]{Thesis}
% for report \usepackage{suthesis-2e}
\usepackage[colorlinks,citecolor=black]{hyperref}
\usepackage{apacite}
\usepackage[english]{babel}
\usepackage[pdftex]{graphicx}
\usepackage{subfig}
\usepackage{verbatim}
%ToDoNotes
\usepackage{todonotes}
%\usepackage{cancel}
\usepackage{soul}



\begin{document}
	

\chapter{Introduction}
\section{An other world view}
The common world view in the western world is that of a world constructed out of matter an Aristotelian heritage. Matter consist of small, standard, building blocks that last for ever. 
Creativity, the generation of novelty, conflicts with this idea of endurance and is therefore more difficult to explain in a world that exists of fixed and ever lasting building blocks.
\todo{see Freidrich Rapp ch 5 in witeheads metaphysics of creativity about how modern sience is not capable of dealing with principles like creativity because of the scientific method}
In a process view of the world, change and creativity is a basic principle. The world is no longer constructed out of matter but consists of processes. Micro processes, related in myriad ways with other micro processes, build up to macro processes, that because of their recurred characteristic construct the world as we experience it. In this world view every thing is explained as a process, also we humans. 

Change is a basic principle in process philosophy. There is not one version of process philosophy, different scholars have different description which contribute to the overall idea of process philosophy.
On of the goals of process philosophy is to overcome the limitations of the substance paradigm and cause a paradigm view.
American philosophy turned out as a fertile source for process philosophy with scholars as C.S. Peirce, John Dewey and William James. The best known process philosopher from the 20th century with the most comprehensive description of Process philosophy was A.N. Whitehead. For Whitehead creativity is the absolute principle of existence. It is his process philosophy I like to use for this case.
\todo{Bergson, Hartshorn}
\todo{twijfels zijn er ook, zijn atomistisch idee van basis processen staat me wat tegen, liever ga ik uit van het oneindig opdelen  van processen, maar dat zien we verderop in het hoofdstuk over proces philosophie}

\todo{moet hier nog meer bij, bv welke auteurs, dat er verschilende stromingen zijn}
\todo{misschien refereren naar de presentatie van Mieke Boon over anders kijken. Met PP kijk ik anders dan met de gangbare vaste stof metaphysica}

%------------------------------------------------------------------------------
%------------------------------------------------------------------------------
\chapter{What is process Philosophy}

\paragraph{What is the pronlem with the common western philosophical view of reality?}
	Knowledge comes from science
	Scientifical method very fruitfull but fails to describe phenomena like creativity
		Logic, descartes object-subject, see Rapp in ch 5
		labratory, reductionist, isolating and testing, not as a combination (interwined verbeek)
		paradigma, limits seeing
		
\paragraph{What is the process philosophical view of reality}
See Reschner introduction en SEP art \cite{Rescher-2012-sep}
Seibt SEP entry \cite{Seibt-2013-sep}


Process philosophy is a view of reality that emphasizes becoming and changing over static being. The principle of process philosophy ranges back to the Greek.
In the Western tradition it is the Greek theoretician Heraclitus of Ephesus (born ca. 560 B.C.E.) who is commonly recognized as the founder of the process approach \cite{Seibt-2013-sep}. 
 but there are two contemporary philosophers that are most associated with the term process philosophy namely Alfred North Whitehead (1861-1947) and Charles Hartshorne (1897-2000).



\paragraph{How does process philosophy help to understand creativity?}

\paragragph(What is creativity)
 beschrijving?
Algemene omschrijving PP
Wat is een process
Wat is creativiteit




\section{fuzzy logic}
Enclopedia of creativity chaper AI
see also fuzzy set


\section{ctr4process}
http://www.ctr4process.org/about/what-process-thought

Basic Doctrines

Process metaphysics, in general, seeks to elucidate the developmental nature of reality, emphasizing becoming rather than static existence or being. It also stresses the inter-relatedness of all entities. Process describes reality as ultimately made up of experiential events rather than enduring inert substances. The particular character of every event, and consequently the world, is the result of a selective process where the relevant past is creatively brought together to become that new event. Reality is conceived as a process of creative advance in which many past events are integrated in the events of the present, and in turn are taken up by future events. The universe proceeds as "the many become one, and are increased by one" in a sequence of integrations at every level and moment of existence. Process thought thus replaces the traditional Western "substance metaphysic" with an "event metaphysic." Terms that further characterize process thought are inter-relatedness, unity-in-diversity, non-dualism, panentheism, mutual transformation, person-in-community, and panexperientialism. The following links are helpful short essays written by scholars that describe and summarize process thought.

Process Philosophy

A Synopsis of Process Thought, by Sheela Pawar
Process Philosophy: Stanford Encyclopedia of Philosophy, by Nicholas Rescher
Process Philosophy: The Internet Encyclopedia of Philosophy, by J.R. Hustwit
Process Philosophy: Wikipedia
Process Thought: It's Value and Meaning to Me, by Charles Birch


\section{Sheela Pawar}
A Synopsis of Process Thought


\section{SEP rescher is opgevolgt door seibt}
http://plato.stanford.edu/archives/sum2012/entries/process-philosophy/

\section{Section Overview}

-1 de big bang, waar komt deze vandaan?
	http://www.big-bang-theory.com/
		Geen duidelijkheid over het begin, in het begin was er niks, het universum en tijd is uit de big bang voorgekomen.
		Geen ontploffing maar een expansie, senl op blazende ballon
	
Whitehead
	Process

Whitehead concept of creativity and modern science
 	



Vragen beantwoorden:
	waarom pp? als model van de cosmos voor deze thesis?
	wat is PP
	
	Wat is een process, micro proces macro proces
	Wat is de drijvende kracht die nieuwe processen doet ontstaan
	waar komen nieuwe processen vandaan
	Aan welke eisen moet een process voldoen
	Hoe ontstaan nieuwe relaties tussen processen
	wat verstaat men onder creativiteit
	wat is het concept van een (nieuw) idee

Begin in deze sectie met een algemene verklaring wat PP is, gebasseerd op een SEP artikel van Johanna Seibt. Er zijn veel verschillende stromingen,waar de wat oudere een totaal oplossing ptoberen te brengen zijn de huidige vooral specialisaties die bijdragen aan een groter geheel.
Zet PP af tegen het bestaande westerse paradigma van de materie-philosopie en licht toe waarom pp zo geschikt is voor de verklaring van creativiteit. Omdat creativiteit over vernieuwing gaat en dat het basis principe van pp is.

Geef een overzicht van N>3 (waarom 3?) verschillende perspectieven van verschillende auteurs op creativiteit. Whitehead is zeer bekend en heeft een zeer uitgebreide pp geschreven, zijn bekendste werk is \cite{whitehead1929process}. Hij ziet creativiteit als het universele principe, maar schijnt niet erg duidelijk te zijn om uit te leggen hoe nieuwe relaties tussen processen tot stand komen bovendien speelt God een belangrijke rol als een soort van primaire invloed op alle processen.

Ik heb gelezen dat Whitehead een atomistic PP is. Hij begint met een soort van basis blokken waaruit processen zijn opgebouwd. Deze ziens wijze speelt mij niet erg aan. Ik volg liever Rescher die stelt dat processen uit processen zijn opgebouwd en dit tot in het oneindige doorloopt. Met vaste bouwstenen wordt het wijzigings principe van PP een beetje onderuit gehaald, immers zijn zijn overanderbare basis blokken. Een God zoals bij Whitehead is in deze visie niet nodig..ik heb ook liever dat deze niet nodg is omdat ik niet in een God geloof dieaan de basis van alles ligt.

Voorlopige auteurs zijn geselecteerd uit het artikel van Seibt.

Zijn er grensen aan welke processen kunnen ontstaan? Is het bijvoorbeeld mogelijk dat een planeet ontstaat in de form van een perfecte 3d driehoek?
Deze vraag is relevant om na te gaan of er grenzen zijn te stellen aan de ideen die kunnen ontstaan

Is het mogelijk dat er processen ontstaan die een planeet/wereld voortbrengen waar geen zwaartekracht heerst?

m.a.w. zijn er basis regels waar processen zich aan moeten houden of zijn de mogelijkheden onbegrenst? 
Rdb..te grote vraag, hoort bij een nieuw project.

Worden de mogelijkheden door de bestaande processen beperkt? Onstaat er een soort van Focus, hoe meer processen er zijn hoe groter de waarschijnlijkheid op een bepaalde uitkomst voor volgende processen? Of anders gezegd met nieuwe processen ontstaan er mogelijkheden voor nieuwe relaties maar misschien beperken processen elkaar ook wel.
Is bijvoorbeeld zwaarte kracht de uitkomst van een process of een bais regel waar processen zich aan moeten houden?
IS de big-bang the schepper van de eerste processen of het gevolg van een zeer lang of eewig lopend process. God was hier een handige verklaring geweest.
Wat zeggen de verschillende auteurs hier over?

Is bijvoorbeeld een mens een cluster van processen of een process?

Zijn er processen die niet kunnen bestaan? Worden dan halve processen gemaakt die direct ophouden te bestaan analoog bijvoorbeeld instabiele materie?
Wat is dan het toetsings criterium? Voorbeeld stel het God-IT proces genereerd een nieuw algortime, als dit niet compileert dan is het als het ware dood geboren en wordt verwijderd/vervalt. Sommige algoritmes creeren een software proces dat lang duurt, mischien met vele herhalingen (zie hier alsof het blijvende materie is), andere een process dat kort is en geen herhalingen bevat.

PP is opgebouwd op andere logica. Welke? Is het mogelijk deze na te bootsen in IT?

Er ontstaan veel nieuwe succesvolle processen. Is dat het gevolg van een goede begeleiding of van het grote aantal pogingen. Voor sommige verschijnsels zijn heel veel micro procesen nodig zoals bijvoorbeeld de werking van de zon, zwaartekracht, bewustzijn etc.

\section{fuzzy logic}
Enclopedia of creativity chaper AI
see also fuzzy set

IV. FUZZY LOGIC
Creative thinking is often required to solve illformed
or poorly formed problems. These are problems
where there is a great deal of uncertainty and difficulty
in making statements with complete precision.
Fuzzy logic has a role to play helping us to get to grips
with such problems. Fuzzy logic is multivalued (as opposed
to binary) logic. While classical logic holds that
everything can be expressed in binary terms—0 or 1,
black or white, yes or no, etc.—fuzzy logic permits
values between 0 and 1, shades of gray, and even partial
membership in a set. Moreover, when the approximate
reasoning of fuzzy logic is used with an expert
system, logical inferences can be drawn from imprecise
relationships.
Neural network technology can be used to produce
a fuzzy logic system which does not provide precise
answers and outcomes to every problem but which will
give reasonably correct estimations. A fuzzy logic system
attempts to categorize patterns according to other
patterns which it has ‘‘learned’’ and makes use of this
learning to suggest answers. This allows more fuzzy input
to be used in the neural network and greatly decreases
the learning time of such networks.
Fuzzy logic has many domestic applications. Home
appliances are common applications and fuzzy logic is
a part of the AI that helps to control such products as
enhanced washing machines, vacuum cleaners, and
air conditioners. some clothes washing machines automatically
adjust for load size and dirtiness of the
clothes. Some vacuum cleaners adjust their suction
power according to the volume of dust and the nature
of the floor. Fuzzy logic is also used to control passenger
elevators, cameras, automobile subsystems, and
smart weapons. Fuzzy logic seems to have potential for
development as far as creative thinking is concerned.
In conjunction with neural networks and expert systems,
fuzzy logic offers a means of improving the
power of such tools.

X. AI AND CREATIVITY
The debate as to whether a machine can think for
itself has some bearing on the role that AI can play in
the sphere of creativity.We can take the general debate
a step further and ask the question of whether a computer
can be creative or whether creativity is separable
from the human mind. Computers clearly can be programmed
to produce a credible, grammatically correct
English sentence given a set of rules and a database of
words. However, the ability to judge the creative value
of such a sentence, and if necessary to modify and
improve its creative value, seems to demand an extensive
base of experience and complex logic that is
so far unique to human cognition. Given these limitations
expert systems would seem to be most relevant to
the notion of providing support for idea processing
rather than its total automation. We might then ask
whether expertise on creativity can be captured and effectively
utilized within an interactive, user-controlled
computer support system.
Arguably, people are intrinsically much more creative
than even the best computer. Human experts can
reorganize information and use it to synthesize new
knowledge. An expert system, in contrast, is apt to
behave in a somewhat uninspired, routine manner.
Human experts handle unanticipated events by using
imaginative and novel approaches to problem solving,
including drawing analogies to situations in completely
different problem domains. Programs have not had
much success at doing this. All humans possess commonsense
knowledge which represents a very broad
spectrum of general knowledge about the world and
how it functions. This commonsense knowledge is immense
and represents a considerable challenge to program
designers. On first thought it might seem that
there does not appear to be any feasible way of programming
it into a computer, though attempts have
been made by the CYC project at MCC in Austin in this
direction. It is commonsense knowledge which makes
humans aware of what they do not know as well as
what they do know. This essential difference allows the
human to avoid wasting time searching for solutions
that are impossible and to concentrate only on finding
feasible solutions. A human would know, for example,
that you cannot put a camel through the eye of a needle
but a computer might search endlessly and in vain for
a solution unless it was programmed initially to appreciate
that this problem was an impossible one to solve.
Finally, human experts can appreciate the overall aspects
of a problem and conceptualize how it relates to
the central issue. Expert systems, however, tend to focus
on the problem itself and do not take account of
issues which are relevant but separate from the
problem.
Given our present knowledge about the processes
used to develop and enhance creativity it might be con-
cluded that there is no deep experience on the subject
but that there are a number of useful methods and
guidelines. A system that can guide the user in the application
of such methods might not seem to be worthy
of being referred to as an expert system—but this of
course may be a matter of opinion.
Both experts and expert systems must possess a large
repertoire of complex knowledge and be able to utilize
and operationalize it within a problem situation. Expert
systems should offer advice which can be acted
upon and not merely prescriptions for how users can
arrive at their own conclusions. Moreover, an expert
system should also be able to explain its own reasoning
as to how it reached its conclusions and advice to enable
the user to assess the value of the advice proffered.
An expert system capable of acting as a creative consultant
would have to be able to produce novel problem
definitions and be able to respond to human reactions
to these definitions with further meaningful
comments, explanations, or modifications. The implication
of this would be that such a system should possess
an experience base as extensive as that of a highly
experienced human adult. Moreover, a potentially useful
knowledge base would not have to be limited to
any particular domain since idea-generating methods
themselves are premised on the notion that creativity
requires the breaking down of overly constraining categories
of knowledge and finding previously undefined
associations.
While automation or expert systems seem to offer a
basis of assisting the creative process, at the present time
the nature of the creative process hinders the effective
use of the pure forms of both these approaches. To
give some idea of the potential problems involved, the
CYC (from enCYClopedia) Project at MCC in Austin, a
10-year project, was begun in 1984 and aimed to enumerate
tens of millions of commonsense facts that will
ultimately ‘‘add up’’ to logical intelligence. The goal is
a system that can understand and speak ordinary language,
and detect violations of common sense as readily
as humans can. The total number of ‘‘rules’’ required for
this was subsequently revised upward by a factor of ten
(to 20–40 million), and extended the time needed by
another 10 years.
Case-based reasoning seems to offer one of the best
short-term prospects for producing suitable vehicles
to assist in creative problem solving. The database in
this instance might contain documented problem solving
case histories across many different domains. Such
a database might be accessed in a way that it provides
insights for problem solving through analogical
reasoning.


\section{Section Overview}

-1 de big bang, waar komt deze vandaan?
http://www.big-bang-theory.com/
Geen duidelijkheid over het begin, in het begin was er niks, het universum en tijd is uit de big bang voorgekomen.
Geen ontploffing maar een expansie, senl op blazende ballon

Whitehead
Process

Whitehead concept of creativity and modern science




Vragen beantwoorden:
waarom pp? als model van de cosmos voor deze thesis?
wat is PP

Wat is een process, micro proces macro proces
Wat is de drijvende kracht die nieuwe processen doet ontstaan
waar komen nieuwe processen vandaan
Aan welke eisen moet een process voldoen
Hoe ontstaan nieuwe relaties tussen processen
wat verstaat men onder creativiteit
wat is het concept van een (nieuw) idee

Begin in deze sectie met een algemene verklaring wat PP is, gebasseerd op een SEP artikel van Johanna Seibt. Er zijn veel verschillende stromingen,waar de wat oudere een totaal oplossing ptoberen te brengen zijn de huidige vooral specialisaties die bijdragen aan een groter geheel.
Zet PP af tegen het bestaande westerse paradigma van de materie-philosopie en licht toe waarom pp zo geschikt is voor de verklaring van creativiteit. Omdat creativiteit over vernieuwing gaat en dat het basis principe van pp is.

Geef een overzicht van N>3 (waarom 3?) verschillende perspectieven van verschillende auteurs op creativiteit. Whitehead is zeer bekend en heeft een zeer uitgebreide pp geschreven, zijn bekendste werk is \cite{whitehead1929process}. Hij ziet creativiteit als het universele principe, maar schijnt niet erg duidelijk te zijn om uit te leggen hoe nieuwe relaties tussen processen tot stand komen bovendien speelt God een belangrijke rol als een soort van primaire invloed op alle processen.

Ik heb gelezen dat Whitehead een atomistic PP is. Hij begint met een soort van basis blokken waaruit processen zijn opgebouwd. Deze ziens wijze speelt mij niet erg aan. Ik volg liever Rescher die stelt dat processen uit processen zijn opgebouwd en dit tot in het oneindige doorloopt. Met vaste bouwstenen wordt het wijzigings principe van PP een beetje onderuit gehaald, immers zijn zijn overanderbare basis blokken. Een God zoals bij Whitehead is in deze visie niet nodig..ik heb ook liever dat deze niet nodg is omdat ik niet in een God geloof dieaan de basis van alles ligt.

Voorlopige auteurs zijn geselecteerd uit het artikel van Seibt.

Zijn er grensen aan welke processen kunnen ontstaan? Is het bijvoorbeeld mogelijk dat een planeet ontstaat in de form van een perfecte 3d driehoek?
Deze vraag is relevant om na te gaan of er grenzen zijn te stellen aan de ideen die kunnen ontstaan

Is het mogelijk dat er processen ontstaan die een planeet/wereld voortbrengen waar geen zwaartekracht heerst?

m.a.w. zijn er basis regels waar processen zich aan moeten houden of zijn de mogelijkheden onbegrenst? 
Rdb..te grote vraag, hoort bij een nieuw project.

Worden de mogelijkheden door de bestaande processen beperkt? Onstaat er een soort van Focus, hoe meer processen er zijn hoe groter de waarschijnlijkheid op een bepaalde uitkomst voor volgende processen? Of anders gezegd met nieuwe processen ontstaan er mogelijkheden voor nieuwe relaties maar misschien beperken processen elkaar ook wel.
Is bijvoorbeeld zwaarte kracht de uitkomst van een process of een bais regel waar processen zich aan moeten houden?
IS de big-bang the schepper van de eerste processen of het gevolg van een zeer lang of eewig lopend process. God was hier een handige verklaring geweest.
Wat zeggen de verschillende auteurs hier over?

Is bijvoorbeeld een mens een cluster van processen of een process?

Zijn er processen die niet kunnen bestaan? Worden dan halve processen gemaakt die direct ophouden te bestaan analoog bijvoorbeeld instabiele materie?
Wat is dan het toetsings criterium? Voorbeeld stel het God-IT proces genereerd een nieuw algortime, als dit niet compileert dan is het als het ware dood geboren en wordt verwijderd/vervalt. Sommige algoritmes creeren een software proces dat lang duurt, mischien met vele herhalingen (zie hier alsof het blijvende materie is), andere een process dat kort is en geen herhalingen bevat.

PP is opgebouwd op andere logica. Welke? Is het mogelijk deze na te bootsen in IT?

Er ontstaan veel nieuwe succesvolle processen. Is dat het gevolg van een goede begeleiding of van het grote aantal pogingen. Voor sommige verschijnsels zijn heel veel micro procesen nodig zoals bijvoorbeeld de werking van de zon, zwaartekracht, bewustzijn etc.


\section{Content}
\section{Conclusion}

%------------------------------------------------------------------------------
%------------------------------------------------------------------------------
\end{document}

