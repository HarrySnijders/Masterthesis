%\documentclass[a4paper]{report}
\documentclass[a4paper]{Thesis}
% for report \usepackage{suthesis-2e}
\usepackage[colorlinks,citecolor=black]{hyperref}
\usepackage{apacite}
\usepackage[english]{babel}
\usepackage[pdftex]{graphicx}
\usepackage{subfig}
\usepackage{verbatim}
%ToDoNotes
\usepackage{todonotes}
%\usepackage{cancel}
\usepackage{soul}



\begin{document}
	


\chapter{overzicht What is process Philosophy}

%------------------------------------------------------------------------------
%------------------------------------------------------------------------------



\chapter{What is process Philosophy 006 37-07}

\paragraph{What is the pronlem with the common western philosophical view of reality?}
Knowledge comes from science
Scientifical method very fruitfull but fails to describe phenomena like creativity
Logic, descartes object-subject, see Rapp in ch 5
labratory, reductionist, isolating and testing, not as a combination (interwined verbeek)
paradigma, limits seeing

\paragraph{What is the process philosophical view of reality}
See Reschner introduction en SEP art \cite{Rescher-2012-sep}
Seibt SEP entry \cite{Seibt-2013-sep}


Process philosophy is a view of reality that emphasizes becoming and changing over static being. The principle of process philosophy ranges back to the Greek.
In the Western tradition it is the Greek theoretician Heraclitus of Ephesus (born ca. 560 B.C.E.) who is commonly recognized as the founder of the process approach \cite{Seibt-2013-sep}. 
but there are two contemporary philosophers that are most associated with the term process philosophy namely Alfred North Whitehead (1861-1947) and Charles Hartshorne (1897-2000).



\paragraph{How does process philosophy help to understand creativity?}

\paragragph(What is creativity)
beschrijving?
Algemene omschrijving PP
Wat is een process
Wat is creativiteit




\section{fuzzy logic}
Enclopedia of creativity chaper AI
see also fuzzy set


\section{ctr4process}
http://www.ctr4process.org/about/what-process-thought

Basic Doctrines

Process metaphysics, in general, seeks to elucidate the developmental nature of reality, emphasizing becoming rather than static existence or being. It also stresses the inter-relatedness of all entities. Process describes reality as ultimately made up of experiential events rather than enduring inert substances. The particular character of every event, and consequently the world, is the result of a selective process where the relevant past is creatively brought together to become that new event. Reality is conceived as a process of creative advance in which many past events are integrated in the events of the present, and in turn are taken up by future events. The universe proceeds as "the many become one, and are increased by one" in a sequence of integrations at every level and moment of existence. Process thought thus replaces the traditional Western "substance metaphysic" with an "event metaphysic." Terms that further characterize process thought are inter-relatedness, unity-in-diversity, non-dualism, panentheism, mutual transformation, person-in-community, and panexperientialism. The following links are helpful short essays written by scholars that describe and summarize process thought.

Process Philosophy

A Synopsis of Process Thought, by Sheela Pawar
Process Philosophy: Stanford Encyclopedia of Philosophy, by Nicholas Rescher
Process Philosophy: The Internet Encyclopedia of Philosophy, by J.R. Hustwit
Process Philosophy: Wikipedia
Process Thought: It's Value and Meaning to Me, by Charles Birch


\section{Sheela Pawar}
A Synopsis of Process Thought


\section{SEP rescher is opgevolgt door seibt}
http://plato.stanford.edu/archives/sum2012/entries/process-philosophy/

\section{Section Overview}

-1 de big bang, waar komt deze vandaan?
http://www.big-bang-theory.com/
Geen duidelijkheid over het begin, in het begin was er niks, het universum en tijd is uit de big bang voorgekomen.
Geen ontploffing maar een expansie, senl op blazende ballon

Whitehead
Process

Whitehead concept of creativity and modern science




Vragen beantwoorden:
waarom pp? als model van de cosmos voor deze thesis?
wat is PP

Wat is een process, micro proces macro proces
Wat is de drijvende kracht die nieuwe processen doet ontstaan
waar komen nieuwe processen vandaan
Aan welke eisen moet een process voldoen
Hoe ontstaan nieuwe relaties tussen processen
wat verstaat men onder creativiteit
wat is het concept van een (nieuw) idee

Begin in deze sectie met een algemene verklaring wat PP is, gebasseerd op een SEP artikel van Johanna Seibt. Er zijn veel verschillende stromingen,waar de wat oudere een totaal oplossing ptoberen te brengen zijn de huidige vooral specialisaties die bijdragen aan een groter geheel.
Zet PP af tegen het bestaande westerse paradigma van de materie-philosopie en licht toe waarom pp zo geschikt is voor de verklaring van creativiteit. Omdat creativiteit over vernieuwing gaat en dat het basis principe van pp is.

Geef een overzicht van N>3 (waarom 3?) verschillende perspectieven van verschillende auteurs op creativiteit. Whitehead is zeer bekend en heeft een zeer uitgebreide pp geschreven, zijn bekendste werk is \cite{whitehead1929process}. Hij ziet creativiteit als het universele principe, maar schijnt niet erg duidelijk te zijn om uit te leggen hoe nieuwe relaties tussen processen tot stand komen bovendien speelt God een belangrijke rol als een soort van primaire invloed op alle processen.

Ik heb gelezen dat Whitehead een atomistic PP is. Hij begint met een soort van basis blokken waaruit processen zijn opgebouwd. Deze ziens wijze speelt mij niet erg aan. Ik volg liever Rescher die stelt dat processen uit processen zijn opgebouwd en dit tot in het oneindige doorloopt. Met vaste bouwstenen wordt het wijzigings principe van PP een beetje onderuit gehaald, immers zijn zijn overanderbare basis blokken. Een God zoals bij Whitehead is in deze visie niet nodig..ik heb ook liever dat deze niet nodg is omdat ik niet in een God geloof dieaan de basis van alles ligt.

Voorlopige auteurs zijn geselecteerd uit het artikel van Seibt.

Zijn er grensen aan welke processen kunnen ontstaan? Is het bijvoorbeeld mogelijk dat een planeet ontstaat in de form van een perfecte 3d driehoek?
Deze vraag is relevant om na te gaan of er grenzen zijn te stellen aan de ideen die kunnen ontstaan

Is het mogelijk dat er processen ontstaan die een planeet/wereld voortbrengen waar geen zwaartekracht heerst?

m.a.w. zijn er basis regels waar processen zich aan moeten houden of zijn de mogelijkheden onbegrenst? 
Rdb..te grote vraag, hoort bij een nieuw project.

Worden de mogelijkheden door de bestaande processen beperkt? Onstaat er een soort van Focus, hoe meer processen er zijn hoe groter de waarschijnlijkheid op een bepaalde uitkomst voor volgende processen? Of anders gezegd met nieuwe processen ontstaan er mogelijkheden voor nieuwe relaties maar misschien beperken processen elkaar ook wel.
Is bijvoorbeeld zwaarte kracht de uitkomst van een process of een bais regel waar processen zich aan moeten houden?
IS de big-bang the schepper van de eerste processen of het gevolg van een zeer lang of eewig lopend process. God was hier een handige verklaring geweest.
Wat zeggen de verschillende auteurs hier over?

Is bijvoorbeeld een mens een cluster van processen of een process?

Zijn er processen die niet kunnen bestaan? Worden dan halve processen gemaakt die direct ophouden te bestaan analoog bijvoorbeeld instabiele materie?
Wat is dan het toetsings criterium? Voorbeeld stel het God-IT proces genereerd een nieuw algortime, als dit niet compileert dan is het als het ware dood geboren en wordt verwijderd/vervalt. Sommige algoritmes creeren een software proces dat lang duurt, mischien met vele herhalingen (zie hier alsof het blijvende materie is), andere een process dat kort is en geen herhalingen bevat.

PP is opgebouwd op andere logica. Welke? Is het mogelijk deze na te bootsen in IT?

Er ontstaan veel nieuwe succesvolle processen. Is dat het gevolg van een goede begeleiding of van het grote aantal pogingen. Voor sommige verschijnsels zijn heel veel micro procesen nodig zoals bijvoorbeeld de werking van de zon, zwaartekracht, bewustzijn etc.

\section{Section Overview}
Vragen beantwoorden:
waarom pp? als model van de cosmos voor deze thesis?
wat is PP
Wat is een process, micro proces macro proces
Wat is de drijvende kracht die nieuwe processen doet ontstaan
waar komen nieuwe processen vandaan
Aan welke eisen moet een process voldoen
Hoe ontstaan nieuwe relaties tussen processen
wat verstaat men onder creativiteit
wat is het concept van een (nieuw) idee

Begin in deze sectie met een algemene verklaring wat PP is, gebasseerd op een SEP artikel van Johanna Seibt. Er zijn veel verschillende stromingen,waar de wat oudere een totaal oplossing ptoberen te brengen zijn de huidige vooral specialisaties die bijdragen aan een groter geheel.
Zet PP af tegen het bestaande westerse paradigma van de materie-philosopie en licht toe waarom pp zo geschikt is voor de verklaring van creativiteit. Omdat creativiteit over vernieuwing gaat en dat het basis principe van pp is.

Geef een overzicht van N>3 (waarom 3?) verschillende perspectieven van verschillende auteurs op creativiteit. Whitehead is zeer bekend en heeft een zeer uitgebreide pp geschreven, zijn bekendste werk is \cite{whitehead1929process}. Hij ziet creativiteit als het universele principe, maar schijnt niet erg duidelijk te zijn om uit te leggen hoe nieuwe relaties tussen processen tot stand komen bovendien speelt God een belangrijke rol als een soort van primaire invloed op alle processen.

Ik heb gelezen dat Whitehead een atomistic PP is. Hij begint met een soort van basis blokken waaruit processen zijn opgebouwd. Deze ziens wijze speelt mij niet erg aan. Ik volg liever Rescher die stelt dat processen uit processen zijn opgebouwd en dit tot in het oneindige doorloopt. Met vaste bouwstenen wordt het wijzigings principe van PP een beetje onderuit gehaald, immers zijn zijn overanderbare basis blokken. Een God zoals bij Whitehead is in deze visie niet nodig..ik heb ook liever dat deze niet nodg is omdat ik niet in een God geloof dieaan de basis van alles ligt.

Voorlopige auteurs zijn geselecteerd uit het artikel van Seibt.

\section{Te beantwoorden vragen}
Zijn er grensen aan welke processen kunnen ontstaan? Is het bijvoorbeeld mogelijk dat een planeet ontstaat in de form van een perfecte 3d driehoek?
Deze vraag is relevant om na te gaan of er grenzen zijn te stellen aan de ideen die kunnen ontstaan

Is het mogelijk dat er processen ontstaan die een planeet/wereld voortbrengen waar geen zwaartekracht heerst?

m.a.w. zijn er basis regels waar processen zich aan moeten houden of zijn de mogelijkheden onbegrenst? 
Rdb..te grote vraag, hoort bij een nieuw project.

Worden de mogelijkheden door de bestaande processen beperkt? Onstaat er een soort van Focus, hoe meer processen er zijn hoe groter de waarschijnlijkheid op een bepaalde uitkomst voor volgende processen? Of anders gezegd met nieuwe processen ontstaan er mogelijkheden voor nieuwe relaties maar misschien beperken processen elkaar ook wel.
Is bijvoorbeeld zwaarte kracht de uitkomst van een process of een bais regel waar processen zich aan moeten houden?
IS de big-bang the schepper van de eerste processen of het gevolg van een zeer lang of eewig lopend process. God was hier een handige verklaring geweest.
Wat zeggen de verschillende auteurs hier over?

Is bijvoorbeeld een mens een cluster van processen of een process?

Zijn er processen die niet kunnen bestaan? Worden dan halve processen gemaakt die direct ophouden te bestaan analoog bijvoorbeeld instabiele materie?
Wat is dan het toetsings criterium? Voorbeeld stel het God-IT proces genereerd een nieuw algortime, als dit niet compileert dan is het als het ware dood geboren en wordt verwijderd/vervalt. Sommige algoritmes creeren een software proces dat lang duurt, mischien met vele herhalingen (zie hier alsof het blijvende materie is), andere een process dat kort is en geen herhalingen bevat.

PP is opgebouwd op andere logica. Welke? Is het mogelijk deze na te bootsen in IT?

Er ontstaan veel nieuwe succesvolle processen. Is dat het gevolg van een goede begeleiding of van het grote aantal pogingen. Voor sommige verschijnsels zijn heel veel micro procesen nodig zoals bijvoorbeeld de werking van de zon, zwaartekracht, bewustzijn etc.

\section{Te beantwoorden vragen}
Zijn er grensen aan welke processen kunnen ontstaan? Is het bijvoorbeeld mogelijk dat een planeet ontstaat in de form van een perfecte 3d driehoek?
Deze vraag is relevant om na te gaan of er grenzen zijn te stellen aan de ideen die kunnen ontstaan

Is het mogelijk dat er processen ontstaan die een planeet/wereld voortbrengen waar geen zwaartekracht heerst?

m.a.w. zijn er basis regels waar processen zich aan moeten houden of zijn de mogelijkheden onbegrenst? 
Rdb..te grote vraag, hoort bij een nieuw project.

Worden de mogelijkheden door de bestaande processen beperkt? Onstaat er een soort van Focus, hoe meer processen er zijn hoe groter de waarschijnlijkheid op een bepaalde uitkomst voor volgende processen? Of anders gezegd met nieuwe processen ontstaan er mogelijkheden voor nieuwe relaties maar misschien beperken processen elkaar ook wel.
Is bijvoorbeeld zwaarte kracht de uitkomst van een process of een bais regel waar processen zich aan moeten houden?
IS de big-bang the schepper van de eerste processen of het gevolg van een zeer lang of eewig lopend process. God was hier een handige verklaring geweest.
Wat zeggen de verschillende auteurs hier over?

Is bijvoorbeeld een mens een cluster van processen of een process?

Zijn er processen die niet kunnen bestaan? Worden dan halve processen gemaakt die direct ophouden te bestaan analoog bijvoorbeeld instabiele materie?
Wat is dan het toetsings criterium? Voorbeeld stel het God-IT proces genereerd een nieuw algortime, als dit niet compileert dan is het als het ware dood geboren en wordt verwijderd/vervalt. Sommige algoritmes creeren een software proces dat lang duurt, mischien met vele herhalingen (zie hier alsof het blijvende materie is), andere een process dat kort is en geen herhalingen bevat.

PP is opgebouwd op andere logica. Welke? Is het mogelijk deze na te bootsen in IT?

Er ontstaan veel nieuwe succesvolle processen. Is dat het gevolg van een goede begeleiding of van het grote aantal pogingen. Voor sommige verschijnsels zijn heel veel micro procesen nodig zoals bijvoorbeeld de werking van de zon, zwaartekracht, bewustzijn etc.

\section{fuzzy logic}
Enclopedia of creativity chaper AI
see also fuzzy set

IV. FUZZY LOGIC
Creative thinking is often required to solve illformed
or poorly formed problems. These are problems
where there is a great deal of uncertainty and difficulty
in making statements with complete precision.
Fuzzy logic has a role to play helping us to get to grips
with such problems. Fuzzy logic is multivalued (as opposed
to binary) logic. While classical logic holds that
everything can be expressed in binary terms—0 or 1,
black or white, yes or no, etc.—fuzzy logic permits
values between 0 and 1, shades of gray, and even partial
membership in a set. Moreover, when the approximate
reasoning of fuzzy logic is used with an expert
system, logical inferences can be drawn from imprecise
relationships.
Neural network technology can be used to produce
a fuzzy logic system which does not provide precise
answers and outcomes to every problem but which will
give reasonably correct estimations. A fuzzy logic system
attempts to categorize patterns according to other
patterns which it has ‘‘learned’’ and makes use of this
learning to suggest answers. This allows more fuzzy input
to be used in the neural network and greatly decreases
the learning time of such networks.
Fuzzy logic has many domestic applications. Home
appliances are common applications and fuzzy logic is
a part of the AI that helps to control such products as
enhanced washing machines, vacuum cleaners, and
air conditioners. some clothes washing machines automatically
adjust for load size and dirtiness of the
clothes. Some vacuum cleaners adjust their suction
power according to the volume of dust and the nature
of the floor. Fuzzy logic is also used to control passenger
elevators, cameras, automobile subsystems, and
smart weapons. Fuzzy logic seems to have potential for
development as far as creative thinking is concerned.
In conjunction with neural networks and expert systems,
fuzzy logic offers a means of improving the
power of such tools.

X. AI AND CREATIVITY
The debate as to whether a machine can think for
itself has some bearing on the role that AI can play in
the sphere of creativity.We can take the general debate
a step further and ask the question of whether a computer
can be creative or whether creativity is separable
from the human mind. Computers clearly can be programmed
to produce a credible, grammatically correct
English sentence given a set of rules and a database of
words. However, the ability to judge the creative value
of such a sentence, and if necessary to modify and
improve its creative value, seems to demand an extensive
base of experience and complex logic that is
so far unique to human cognition. Given these limitations
expert systems would seem to be most relevant to
the notion of providing support for idea processing
rather than its total automation. We might then ask
whether expertise on creativity can be captured and effectively
utilized within an interactive, user-controlled
computer support system.
Arguably, people are intrinsically much more creative
than even the best computer. Human experts can
reorganize information and use it to synthesize new
knowledge. An expert system, in contrast, is apt to
behave in a somewhat uninspired, routine manner.
Human experts handle unanticipated events by using
imaginative and novel approaches to problem solving,
including drawing analogies to situations in completely
different problem domains. Programs have not had
much success at doing this. All humans possess commonsense
knowledge which represents a very broad
spectrum of general knowledge about the world and
how it functions. This commonsense knowledge is immense
and represents a considerable challenge to program
designers. On first thought it might seem that
there does not appear to be any feasible way of programming
it into a computer, though attempts have
been made by the CYC project at MCC in Austin in this
direction. It is commonsense knowledge which makes
humans aware of what they do not know as well as
what they do know. This essential difference allows the
human to avoid wasting time searching for solutions
that are impossible and to concentrate only on finding
feasible solutions. A human would know, for example,
that you cannot put a camel through the eye of a needle
but a computer might search endlessly and in vain for
a solution unless it was programmed initially to appreciate
that this problem was an impossible one to solve.
Finally, human experts can appreciate the overall aspects
of a problem and conceptualize how it relates to
the central issue. Expert systems, however, tend to focus
on the problem itself and do not take account of
issues which are relevant but separate from the
problem.
Given our present knowledge about the processes
used to develop and enhance creativity it might be con-
cluded that there is no deep experience on the subject
but that there are a number of useful methods and
guidelines. A system that can guide the user in the application
of such methods might not seem to be worthy
of being referred to as an expert system—but this of
course may be a matter of opinion.
Both experts and expert systems must possess a large
repertoire of complex knowledge and be able to utilize
and operationalize it within a problem situation. Expert
systems should offer advice which can be acted
upon and not merely prescriptions for how users can
arrive at their own conclusions. Moreover, an expert
system should also be able to explain its own reasoning
as to how it reached its conclusions and advice to enable
the user to assess the value of the advice proffered.
An expert system capable of acting as a creative consultant
would have to be able to produce novel problem
definitions and be able to respond to human reactions
to these definitions with further meaningful
comments, explanations, or modifications. The implication
of this would be that such a system should possess
an experience base as extensive as that of a highly
experienced human adult. Moreover, a potentially useful
knowledge base would not have to be limited to
any particular domain since idea-generating methods
themselves are premised on the notion that creativity
requires the breaking down of overly constraining categories
of knowledge and finding previously undefined
associations.
While automation or expert systems seem to offer a
basis of assisting the creative process, at the present time
the nature of the creative process hinders the effective
use of the pure forms of both these approaches. To
give some idea of the potential problems involved, the
CYC (from enCYClopedia) Project at MCC in Austin, a
10-year project, was begun in 1984 and aimed to enumerate
tens of millions of commonsense facts that will
ultimately ‘‘add up’’ to logical intelligence. The goal is
a system that can understand and speak ordinary language,
and detect violations of common sense as readily
as humans can. The total number of ‘‘rules’’ required for
this was subsequently revised upward by a factor of ten
(to 20–40 million), and extended the time needed by
another 10 years.
Case-based reasoning seems to offer one of the best
short-term prospects for producing suitable vehicles
to assist in creative problem solving. The database in
this instance might contain documented problem solving
case histories across many different domains. Such
a database might be accessed in a way that it provides
insights for problem solving through analogical
reasoning.



\section{Overview}


\begin{quotation}
	Process philosophy opposes ‘substance metaphysics,’ the dominant research paradigm in the history of Western philosophy since Aristotle. 
\end{quotation}
\cite{Seibt2013}

Process philosophy is based on the premise that the world as existing of an organism of dynamic behaviour rather then constructed out of ever lasting material. The basic building block is not matter but what is often called a process. A process is a sequence of actions in time that are behaving according a certain pattern that gives the process a unique structure. Processes have relations with each other and influence each other. What in the substance philosophy is called matter is in the process philosophy explained as recurring processes with similar structures.

???Mijn idee van iets bestaat doordat het een evenwicht form in een systeem met wisselende mogelijkheden. Een soort rustpunt in een chaos van mogelijkheden.
Aangezien alles in beweging is is het ook waarschijnlijk dat de evenwichtspunten oplossen in tijd om nieuwe knopen van evenwicht te formen.

In process philosophy primary entities are not static as in the substance paradigm with a view of an eternal world, but the primary entities are dynamic and the world is a becoming one.

In substance philosophy the basic principle is that things do not change, producing new is not included in the basic principle.
In process philosophy the basic principle is that things do change, creating new is the basic principle.

In the substance paradigm the change on which creativity is build is an anomaly from the basic principle whereas in the process paradigm creativity lies close to its basic principle. Some process philosophers, like Whitehead \cite{whitehead1929process} Science in the paradigm of Substance Philosophy can not explain everything as in the case of creativity.

Process Philosophy versus Substance Philosophy.

???A Wide range of different  perspectives amongst process philosophers

???ToDo

\section{Stanford process philosophy}
\begin{verbatim}
file:///C:/AsusMiniMijnData/Zotero/Profiles/0rf6p293.default/zotero/storage/TRSB7DQH/process-philosophy.html
\end{verbatim}


\begin{verbatim}
@InCollection{Seibt-2013,
author       =	{Seibt, Johanna},
title        =	{Process Philosophy},
booktitle    =	{The Stanford Encyclopedia of Philosophy},
editor       =	{Edward N. Zalta},
howpublished =	{\url{http://plato.stanford.edu/archives/fall2013/entries/process-philosophy/}},
year         =	{2013},
edition      =	{Fall 2013},
}
\end{verbatim}


>>Stanford Encyclopedia of Philosophy
\begin{quotation}
	>>Process philosophy is based on the premise that being is dynamic and that the dynamic nature of being should be the primary focus of any comprehensive philosophical account of reality and our place within it. 
\end{quotation}

\begin{quotation}
	>>Even though we experience our world and ourselves as continuously changing, Western metaphysics has long been obsessed with describing reality as an assembly of static individuals whose dynamic features are either taken to be mere appearances or ontologically secondary and derivative. 
	For process philosophers the adventure of philosophy begins with a set of problems that traditional metaphysics marginalizes or even sidesteps altogether: 
	>>what is the role of mind in our experience of reality as becoming? 
	>>Are there several varieties of becoming — for instance, the uniform going on of activities versus the coming about of developments? 
	>> Do all developments have the same way of occurring quite independently of what is coming about? 
	>>How can we best classify into different kinds of occurrences what is going on and coming about? 
	!!>>How can we understand the emergence of apparently novel conditions?
	
\end{quotation}

\begin{quotation}
	Process philosophy opposes ‘substance metaphysics,’ the dominant research paradigm in the history of Western philosophy since Aristotle.
\end{quotation}

\begin{quotation}
	Central among these is the notion of a basic entity that is individuated in terms of what it ‘does.
\end{quotation}

\begin{quotation}
	his type of functionally individuated entity is often labeled ‘process’ in a technical sense of this term that does not coincide with our common-sense notion of a process. Some of the ‘processes’ postulated by process philosophers are—in agreement with our common-sense understanding of processes—temporal developments that can be analyzed as temporally structured sequences of stages of an occurrence, with each such stage being numerically and qualitatively different from any other.
\end{quotation}

\begin{quotation}
	\todo{belangrijk}
	What holds for all dynamic entities labelled ‘processes’, however, is that they occur — that they are somehow or other intimately connected not only to temporal extension but also to the directionality or passage of time.
\end{quotation}

\begin{quotation}
	The perhaps most powerful argument for process philosophy is its wide descriptive or explanatory scope. 
	\todo{belangrijk}
	If we admit that the basic entities of our world are processes, we can generate better philosophical descriptions of all the kinds of entities and relationships we are committed to when we reason about our world in common sense and in science: from quantum entanglement to consciousness, from computation to feelings, from things to institutions, from organisms to societies, from traffic jams to climate change, from spacetime to beauty. 
\end{quotation}

\begin{quotation}
	As may have become clear from the brief review of historical contributions to process philosophy in section 1, process philosophy is a complex and highly diversified field that is not tied to any school, method, position, or even paradigmatic notion of process.
\end{quotation} 
Voor wie ga ik kiezen?

\begin{quotation}
	Or again, process thinkers hold that philosophical research may legitimately address ‘creative’ phenomena that cannot be described as modifications of permanent units, such as phenomena of emergence or self-organization. What unifies contemporary process-philosophical research more than any other aspect, however, is its metaphilosophical aim to revise long-standing theoretical habits
\end{quotation}
>>emergence:  the fact of something becoming known or starting to exist.
Hoe komt het dat ik hieraan denk bijvoorbeeld.



\begin{quotation}
	Given its current role as a rival to the dominant substance-geared paradigm of Western metaphysics, process philosophy has the overarching task of establishing the following three claims:
	
	(Claim 1) The basic assumptions of the ‘substance paradigm’ (i.e., a metaphysics based on static entities such as substances, objects, states of affairs, or instantaneous stages) are dispensable theoretical presuppositions rather than laws of thought.
	
	(Claim 2) Process-based theories perform just as well or better than substance-based theories in application to the familiar philosophical topics identified within the substance paradigm.
	
	\textbf{(Claim 3) There are other important philosophical topics that can only be addressed within a process metaphysics.}
\end{quotation}
Het is waarschijnlijk dat ik de nadruk leg op 2-3 omdat creativiteit met de substance filosofie niet goed is uit te leggen naar mijn idee terwijl de hele PP is gebasseerd op verandring, vernieuwing en creativiteit.
(creativiteit begrijp ik in pp als een new process met een andere nog niet eerder voorgekomen structuur). In PP is het een komen en gaan van processen dus er wordt telkens vernieuwd. Maar niet al deze vernieuwingen zijn ook creative vernieuwingen.
Hoe zit dat met de evolutie theorie bij het ontstaan van nieuwe processen?





\begin{verbatim}
@book{poli2010theory,
title={Theory and applications of ontology: Philosophical perspectives},
author={Poli, Roberto and Seibt, Johanna},
year={2010},
publisher={Springer}
}
\end{verbatim}
%p253
\begin{quotation}
	Of the three methods of reasoning, induction draws a generalization from
	instances, but only abduction can introduce a truly novel idea. 
	\todo{belangijk}
	\textbf{In Peirce’s semiotics,
		abduction is the basis for creativity}, and there is no need for Descartes’s innate ideas
	or Kant’s synthetic a priori judgments. \textbf{In a computer implementation, abduction
		can be implemented by a process for selecting appropriate chunks of information
		from memory and reconfiguring them in a novel combination.} It can be performed
	at various levels of complexity:
\end{quotation}

%p252
\begin{quotation}
	Although Peirce knew that deduction is important, he realized that it can only
	derive the implications of already available premises. Without some method for
	deriving the premises, deduction is useless. He observed that two other methods,
	induction and abduction, are required for deriving the starting assumptions:
	\todo{standaard argumentatie regels zie vak mieke boon}
	1. Deduction. Apply a general principle to infer some fact.
	Given: Every bird flies. Tweety is a bird.
	Infer: Tweety flies.
	2. Induction. Assume a general principle that subsumes many facts.
	Given: Tweety, Polly, and Hooty are birds.
	Tweety, Polly, and Hooty fly.
	Assume: Every bird flies.
	3. Abduction. Guess a new hypothesis that explains some fact.
	Given: Vampy flies. Vampy is a bat. Vampy and Tweety have wings.
	Guess: Every animal with wings flies.
	
\end{quotation}

\section{Overview from Nicholas Recher}

\subsubsection{Nicholas Rescher}
\begin{verbatim}
@book{rescher1996process,
title={Process metaphysics: An introduction to process philosophy},
author={Rescher, Nicholas},
year={1996},
publisher={Suny Press}
}
\end{verbatim}

Geeft een mooi overzicht van de diverse auteurs en een stuk over creativiteit en innovatie.

Waarom de keuze voor Whitehead:
\begin{quotation}
	p3
	Like any philosophical movement of larger scale, process philosophy has internal
	variations. The main difference at issue here has roots in the issue of what type of
	process is taken as paramount and paradigmatic. Some important contributors, and
	especially A. N. Whitehead, see physical processes as central and other sorts of processes
	as modeled on or superengrafted upon themthe conception of an all-integrating physical
	field being pivotal even for Whitehead's organic/biological reflections. Others (especially
	Henri Bergson) saw biological processes as fundamental and conceived of the world in
	essentially organismic terms. Still others (especially William James) based their ideas of
	process on a psychological model and saw human thought as idealistically paradigmatic.
	And, consonant with the perspectives as issue here, some (e.g., Whitehead) articulated
	their process philosophy in essentially scientific terms, while others (especially Bergson)
	relied more on intuition and, indeed, an almost mystical sort of sympathetic
	apprehension. All the same, these are differences in style and emphasis that nevertheless
	leave the teachings of the several processists in the position of variations on a common
	theme.
\end{quotation}


\begin{quotation}
	4
	Process and Universals
	Synopsis
	(1) One of the perennial key issues of metaphysics is the "problem of universals"of how to account for the nature
	and status of types, kinds, and similar modes of generality. A process approach to universals affords a natural and
	economic way of addressing this problem. (2) Creativitythe initiation of noveltyis a salient feature of a process-laden
	world, and process philosophy sees this as a pivotal feature of the real. However, the world's processes bring not
	just new individuals (new items) into being but new universals (new kinds) as well. (3) Taxonomic
	complexificationthe emergence of new modes of existence timeis accordingly one of the characteristic doctrines of
	process philosophy.
\end{quotation}

\subsubsection{Peirce}
\subsubsection{James}
\begin{quotation}
	Such a manifold of
	activity is a law unto itselfeven the classic logical laws of excluded middle and
	noncontradiction do not bind it, seeing that concrete activity everywhere manifests the
	potential for breaking out into the most contradictory characterizations.13 In expressing
	his agreement with Peirce, James remarked that to "an observer standing outside of its
	generating causes, novelty can appear only as so much 'chance', while to one who stands
	inside it is the expression of 'free creative activity'. . . . The common objection to
	admitting novelties is that by jumping abruptly, ex nihilo, they shatter the world's rational
	continuity.14 But, he continues, novelty "doesn't arrive by jumps and jolts, it leaks in
	insensibly, for adjacents in experience are always interfused, the smallest real datum
	being both a coming and a going."15 The expression "block universe" served James as a
	term of derogation, because he scorned and abhorred the idea of a closed world that has
	no place for novelty and adventure.
\end{quotation}

\begin{quotation}
	Like his spiritual kinsman, Henri Bergson, James believed that arguments along the lines
	of Zeno's classical paradoxes demonstrated the incapacity of stable concepts to
	characterize the fluidities of an ever-changing reality. But whereas Bergson looked for
	escape from conceptual rigidities
	to the biological sphere, James saw them in the psychological sphere. For him, it is the
	nature of human experience which, above all, prevents the imposition of conceptual
	fixities from giving an adequate account of reality. Accordingly, James strongly
	emphasized the processual nature of experience
	
	James's worldview of flux, spontaneity, and creative novelty projects a philosophy of
	substantiality without substance.
	
	Even truth and knowledge come within the
	realm of the Jamesean dynamism: They are not things we find but things we make.18
\end{quotation}


\subsubsection{Bergson}
\begin{quotation}
	Change, innovation, creativity are nature's essence and
	organic life in their most powerful expression. Evolution and élan vitalorganic life7s
	driving force of creative vitalityare everywhere at work. And this creativity and innovation
	are no mystery to us: We experience them in our own activitiesabove all, in our own acts
	of free will. 20
\end{quotation}

\subsubsection{Dewey}
\begin{quotation}
	The combination of pragmatism with processism at work in C. S. Peirce and William
	James is also found in the thought
	Page 19
	of John Dewey.
	
	Like these thinkers, Dewey emphasized that experience is self-creation, citing with favor
	Bergson's example of "an artist standing before a blank canvas [who] puts up his brush,
	[and] no onenot even he himselfcan know ahead of time what the result will be." 22
	
	Dewey accordingly envisioned "an intrinsic connection of time with individuality" because
	"[individual] development cannot occur when an individual has power and capacities that
	are not actualized at a given time," although the potentialities at work are not
	Aristotelian ("connected to fixed, predestined ends") but rather open ended and novelty
	admitting.
	
	The free individuality which is the source of art is also the final
	source of creative development in time.
\end{quotation}

\subsubsection{vWhitehead}
\begin{quotation}
	Whitehead fixed on "process" as a central category of his philosophy
	because he, toolike James and Bergson before himregarded time, change, and creativity
	as representing salient metaphysical factors.
	
	Whitehead saw two principal sorts of creative process at work in nature: those that are
	operative in shaping the internal make-up of a new concrete particular existent
	("concretion") and those that are operative other-orientedly when existents function so as
	to bring new successors to realization ("transition").
	
	Whitehead was first and foremost a geometer:
	the branch of mathematics that deals with the deduction of the properties, measurement, and relationships of points, lines, angles, and figures in space from their defining conditions by means of certain assumed properties of space.
	
	A unit of
	reality is "the ultimate creature derivative from the creative process," he remarked.
	
	For him, novelty and innovation is ever the order of the day; as he
	saw It, the natural world is a sea of process. He emphatically rejected the idea of clear
	separations in nature:
	
	Whitehead's influence on the development and diffusion of process philosophy was
	immenseindeed, decisive. His challenging writings, his many years of teaching at
	Harvard, and the force of his example as a scientifically literate philosopher combined to
	make for a widely sympathetic reception of his ideas. One recent historian rightly says
	that "it was not strange that for many professional philosophers he became not merely a
	major thinker with whom dialogue was possible but even something of a cult figure."34	
\end{quotation}

\subsubsection{Sheldon}
\subsubsection{Retrospect}

\begin{quotation}
	Indeed, the historical process of process philosophys own development
	instantiates and vividly illustrates process philosophy's message that we live in a world
	where nothing stands still and that change is of the very essence of reality.	
\end{quotation}


\subsubsection{2.Basic Ideas}
\begin{quotation}
	p28
	Thus, in general
	terms, process philosophy is predicated on two contentions:
	In a dynamic world, things cannot do without processes. Since
	substantial things change, their nature must encompass some impetus to
	internal development.
	In a dynamic world, processes are more fundamental than things. Since
	substantial things emerge in and from the world's course of changes,
	processes have priority over things.
	Becoming and changethe origination, flourishing, and passing of the old and the
	innovative emergence of ever-new existenceconstitute the central themes of process
	metaphysics.
\end{quotation}

\begin{quotation}
	We cannot
	adequately describe (let alone explain) processes in terms of something nonprocessual
	any more than we can describe (or explain) spatial relation in nonspatial terms of
	reference. The processual order is, in this sense, conceptually closed. 1
\end{quotation}

\begin{quotation}
	Zo ook ideation.
	
	Natural and human history is
	commonly regarded as a collection of notable eventsthe separation of the moon from the
	earth, say, or the extinction of dinosaurs or the assassination of Abraham Lincoln. But, of
	course, when any one of these "events" is examined in detail, it soon becomes clear that,
	in fact, a long and complicated process is involved, a sequence of activities and
	transactions that in each case constitutes an elaborate story of interconnected
	developments. On closer inspection, the idea of discrete "events" dissolves into a
	manifold of processes which themselves dissolve into further processes.	
\end{quotation}

\begin{quotation}
	p31
	Accordingly, ''process philosophy" is best understood as a doctrine committed, or at any
	rate inclined, to certain basic teachings or contentions:
	that time and change are among the principal categories of metaphysical
	understanding
	that process is a principal category of ontological description
	that processesand the force, energy, and power that they make
	manifestare more fundamental, or at any rate not less fundamental, than
	things for the purposes of ontological theory
	that several if not all of the major elements of the ontological repertoire
	(God, nature-as-a-whole, persons, material substances) are best
	understood in process terms
	that contingency, emergence, novelty, and creativity are among the
	fundamental categories of metaphysical understanding.	
\end{quotation}

\begin{quotation}
	Accordingly, process philosophy is best seen as a broad movement that urges a particular
	sort of approach to the
	Page 33
	problems of metaphysicsa general strategy to the description and explanation of the real.
	However, while process philosophy is a generalized approach rather than a unified
	doctrine, it does, nevertheless, involve a certain doctrinal tendency or inclination, namely,
	to see substantial things as subordinate to processes both ontologically (in the order of
	being) and conceptually (in the order of understanding).	
\end{quotation}


\begin{quotation}
	No ontological doctrine could be more emphatic regarding the ontological primacy of
	stable substances than Greek atomism, which, for this very reason, is the quintessence of
	everything to which process philosophy is opposed.	
\end{quotation}

\begin{quotation}
	p37
	Accordingly, Leibniz, Bergson, and Whitehead (to take only three outstanding examples)
	are all metaphysicians whose processism is closely geared to an organicist approach.
\end{quotation}

\begin{quotation}
	p37
	2.3. What is a Process?
	
	A process is a
	coordinated group of changes in the complexion of reality, an organized family of
	occurrences that are systematically linked to one another either causally or functionally.
	
	A process consists in an
	integrated series of connected developments unfolding in conjoint coordination in line
	with a definite program. Processes are correlated with occurrences or events: Processes
	always involve various events, and events exist only in and through processes.
	
	The unity of a process is the unity of a lawful
	order that need not be fully determinative but is at least delimitative.
	
	Since Whitehead, and indeed since Boscovitch, the concept of physical (gravitational or
	electromagnetic) field has
	Page 40
	been an important paradigm for process philosophy.
	
	(The "fuzzy" character
	of the real is a key theme of Bergson and James.) The contribution of the process idea is
	to help us to keep together in function things that thought inclines to separate in idea.
	
	Rather,
	structural identity of operation is the crux: The two concrete processes involved are
	simply two different spatiotemporal instances of the same generic production
	procedurethat is, the same general recipe is followed in either case.
	
	A particular process is (by hypothesis) a fixed sort of eventuation sequence. But this, of
	course, does not stand in the way of innovation. On one hand, there is the emergence of
	new processes that have not been instantiated before. On the other hand, there is the
	novel concatenation of old microprocesses into new macroprocessesthe combination of
	old processes into new processual structures.
	
	The concept of programmatic (rule-conforming) developments is
	definitive of the idea of process: The unity/identity of a process is the unity/identity of its
	program.
	
	If the ''connection" at issue in that "sequence of connected developments" is
	one of actual causality, then we have a physical process; if it is mental or mathematical
	operations, then we have process of different sorts. However, for present purposes, it is
	the physical processes that constitute the natural world about us which will be the focus
	of concern.
\end{quotation}

\begin{quotation}
	p41 2.4 Modes of Process
	productive and transformative processes
	
	This distinction is important for present purposes because process philosophy is
	characterized by its insistence on the fundamentality of transformative processes, with
	their potential detachment from substantial things.
	
	The distinction between owned and unowned processes also plays an important role in
	process philosophy.
	From the process philosopher's
	point of view, the existence of unowned processes is particularly important because it
	shows that the realm of process as a whole is something additional to and separable from
	the realm of substantial things.
	
	One of the most important ways of classifying processes is by the thematic nature of the
	operations at issue. On this basis we would have (for example) the distinction between
	processes of the following kinds:
	physical causality (in relation to physical changes)
	purposive/teleological (in relation to achieving deliberate objectives)
	cognitive/epistemic (in relation to intellectual problem solvinge.g.,
	programming ourselves for solving a certain sort of problem)
	communicative (in relation to transmitting information)
	
	\todo{Find a class for ideation=transformative process}
\end{quotation}

\begin{quotation}
	5. The Priority of Process: Against the Process Reducibility Thesis
	In sum, processes without substantial entities are perfectly feasible in the conceptual
	order of things, but substances without processes are effectively inconceivable.
\end{quotation}

\begin{quotation}
	6. Processes and Dispositions
	
\end{quotation}

\subsubsection{3.Process and Particulars}
\begin{quotation}
	3
	Process and Particulars
	
	3.1
	The future has its place within the processual present, seeing that the present
	is pregnant with the future (to use Leibniz's metaphor).
	
	3.2
	2. Complexification
	Processes are Janus faced: They look in two directions at once, inwards and outwards.
	They form part of wider (outer) structures but themselves have an inner structure of
	some characteristic sort. For processes generally consist of processes: microprocesses,
	that combine to form macroprocesses. Process theorists often use organismic analogies
	to indicate this idea of different levels of units: Smaller, subordinate (or subsidiary)
	processes unite to form larger, superordinate (or supersidiary) process-units, as in cells
	combining into organs that, in turn, constitute organisms.
	
	Processes, after all, come in all sizes, from the submicroscopic to the cosmic. And when
	smaller processes join to form large ones, the relation is not simply one of part to whole
	but of productive contributory to aggregate result.
	
	3.3
	3. Ongoing Identity as a Matter of Ongoing Reidentifiability: An Idealistic Perspective
	
	Process identification thus involves two components: type
	specification and coordinative spatio-temporal placement.
	
	but rather that the identity of things involves
	processes conceptually.6
	
	As these deliberations indicate, process ontology comes in two major forms or versions,
	the one stronger and causal (ontological), the other weaker and explanatory
	(conceptual):
	Causal processism: processes are causally/existentially fundamental
	relative to things; substances are merely appearances, the correlates of
	processes of "being taken to be a thing."
	Conceptual processism: processes are conceptually fundamental vis à vis
	things in that (1) the explanatory characterizations of what a thing is
	always involves recourse to a processual account of what that thing
	does; and (2) the identification of any particular thing as such always
	involves reference to various processes, a thing being constituted as
	what it is by means of identificating processes.
	
	Accordingly, process philosophers incline to an idealistic view of substantial
	things.
	
	And some process philosophers move even further in an idealistic direction. They hold,
	with Whitehead (and Hartshorne after him), that concrete things are not just
	experientiable but also somehow experiencing items.
	
	\todo{Strawson is blijkbaar een tegenstander, ga na of hij oets over creativiteit zegt. Uberhaubt zoek of er critice op ppp is te vinden gerelateerd aam creativiteit}
	4. Against Strawson's Critique of Processism
\end{quotation}

\begin{quotation}
	\todo{Ideation}
	35. Difficulties of Substantialism
	One major problem with a traditionalist ontology of substances and properties lies in the
	difficulty of handling change-indicative processes and activities on this basis.
	
	Some process philosophers take a still more decidedly negative view of substance
	metaphysics, however. For example, Johanna Seibt has argued 11 that the idea of a
	substantial object as standardly conceived in the philosophical literature is logically
	incoherent
\end{quotation}

\begin{quotation}
	\todo{Belangrijk stuk voor het ontstaan van ideen. Hoe komt eichenlijk iets van niets into existance? Zie ook hoofdstuk 9}
	6. The Origination of Particulars	
	6. The Origination of Particulars
	How do particulars originate and terminate in time? A process philosophy unable to come
	to satisfactory terms with
	Page 66
	this issue would ipso facto be in serious difficulty. In this context, however, one category
	of processes requires special consideration, namely those which, like starting/stopping
	and birth/death and beginning/ending, are oriented toward the anterior or posterior
	nonexistence of the item at issue. To be sure, subsuming these eventuations under the
	characterization of a process requires a recourse to such unorthodox transitions as those
	between existence and nonexistence (or the reverse), and thus means that we must take
	the somewhat unorthodox step of treating nonexistence as a (sort of state of affairs.
	Classical substance metaphysics considers substance origination to be instantaneous:
	there is a "moment of conception," an instant of origination, prior to which the substance
	never existed and after which the substance as such always exists up to some
	subsequent time of its expiry. It is clear, however, that such instantaneous origination is
	not the only theoretically available possibility. An alternative view would contemplate the
	prospect of an interval of concrescencea noninstantaneous coming (rather than springing)
	into existence akin to the building of a house or a ship. (At just what temporal instant or
	point did the Queen Mary come into existence?) This alternative model accordingly sees
	origination itself as a process, envisioning a gestation period between nonexistence and
	existencean interval during which the thing at issue comes into being, that is, literally
	emerges into existence. Such a contrasting way of looking at the matter of origination in
	terms of transitional processuality characterizes the position of most process
	philosophers. Process metaphysicianspreeminently Whiteheadinsist that concrete
	(physical) particulars always arise through processes and inevitably owe them their very
	existence. And, indeed, its involvement and enmeshment in a matrix of process is
	inherent in the very concept of a particular.
	And if coming-into-being is itself actually to be a process, then there has to be a period or
	interval of transitionof reification or concrescenceduring which it can neither be said truly
	that the thing at issue actually exists nor on the other hand that it does not exist at all. 13
	Page 67
	Three salient facts must be noted about such an interval of substance-origination:
	1.It is a "fuzzy" interval that has no definite, clearly specifiable temporal
	beginning or end.
	2.
	During this interval we can say neither that the thing (already) exists
	nor that it does not (yet) exist; during that gestation period the
	substance's existence is actually indeterminate.
	3.
	If the world had been annihilated during this interval, then it would
	neither be correct to say that the thing has (ever) existed in the world
	nor that it has not (ever) existed. The theses that the thing has existed
	(at some time or other) would also have to be classified as
	indeterminate. (From an ontological point of view, it could be said that,
	figuratively speaking, the world "has not managed to make up its mind"
	about the existence of the thing. The world itself is, in this regard,
	indeterminate.)
	As these observations indicate, a conceptually rigorous implementation of the idea of
	reification as a processof a thing's coming into existence over a course of timewill require
	the deployment of two historically unorthodox items of concept-machinery:
	i.
	a "fuzzy logic"or at any rate a fuzzy mathematicsthat puts the
	conception of indefinite (imprecisely bounded) intervals and regions at
	our disposal 14
	ii.
	a semantics of truth-value gaps, serving to countenance propositions
	that are neither (definitely) true nor (definitely) false but indeterminate
	in lacking a classical truth value.15
	Neither of these unorthodoxies is in any way absurd, and both represent theoretical
	resources that are today well known and widely employed in logic and the theory of
	information management. (Indeed, the second idea goes back all the way to Aristotle
	himself, in the sea battle example of the discussion of future contigency in chapter 9 of
	De interpretatione.16 )
	
\end{quotation}

\subsubsection{3.Process and Universals}
\begin{quotation}
	4
	Process and Universals	
	
	(2) Creativitythe initiation of noveltyis a salient feature of a process-laden
	world, and process philosophy sees this as a pivotal feature of the real. However, the world's processes bring not
	just new individuals (new items) into being but new universals (new kinds) as well.
	
	4.1. Process and "The Problem of Universals"
	
	As the process philosopher sees it, what
	is at bottom at issue here is a matter of lawful modus operandi rather than of common
	properties of some sort.
	
	For the process metaphysician, by
	contrast, all the processes of a given type constitute a field or realm; even as all
	economic processes constitute the economic realm and all biological processes constitute
	the biological realm, all physical processes constitute the physical realm. And from the
	angle of a process metaphysics, the conceptually advantageous fact is that realms are
	composed of their subdomains by way of straightforward inclusionand thus in a way in
	which propertyuniversals are not and cannot be composed of the properties that
	constitute them. Realms are megaprocesses, as it were. And such megaprocesses need
	not necessarily be continuous in space and time; it makes perfect sense to see all
	processes of the same sort (all pencil sharpenings, for example) as constituting not
	merely a processual thing-kind but also as thereby constituting a spatiotemporally
	distributed megaprocess.
	
	Take phenomenal colors, for example. A mental process such as perceiving or
	imagining a certain shade of red is simply a way of perceiving redly or imagining redlythat
	is to say, in a certain particular way. And here, the relevant universal is not the abstract
	quality red, but the generic process at issue in perceiving (seeing, apprehending)
	something redly. (Qualities of this sort are fundamentally adverbial rather than adjectival
	in nature.)
	
	How distinct minds can conceive the same universal is now no more mysterious than how
	distinct birds can share the same song. In both cases it is simply a matter of doing things
	in the same generic way. Otherwise mysterious-seeming universals such as odors or fears
	are simply shared structural features of mental processes.
	For the process philosopher,
	quality, too, is what quality does.
	
	Processes, seen abstractly, are inherently structural and programmaticand, in
	consequence, universal and repeatable. To be a process is to be a process of a certain
	structural sort, a certain specifiable make-up. What concretizes processes of a given
	abstract characterizationa rain shower, for exampleis simply their spatiotemporal
	emplacement within the wider matrix of natural process, their positioning by way of
	concrete realizations or instantiations in the framework of reality.
	
	Since something
	cannot be a process without exhibiting structure, it must be of a process-type. With
	processes, universality is inescapable.
	
\end{quotation}

\subsubsection{4.2 Creativity}
\begin{quotation}
	\todo{de link naar ideation?}
	4.2. Novelty, Innovation, Creativity	
	
	\todo{onderzoek deze auteur}
	Stephen C. Pepper was a process philosopher who devoted much attention to clarifying
	the idea of novelty.
	
	Innovation by definition involves novelty, and very different things can be at issue here.
	Those of principal concern to process philosophy are
	ontological innovation: new things, processes, products, states of affairs
	(Note: "New" in the presently operative sense will mean new in kind)
	phenomenical innovation: new types of events or courses of events
	(natural or personal histories)
	\textbf{epistemic (or conceptual or cognitive) innovation: new sorts of
		knowledge, ideas, information, problems, questions.
		\todo{hier ligt denk ik de focus}
		
		Given its commitment to the centrality of time, process philosophy considers the specious
		present to be the movable entryway separating a settled and determinate past from an
		open and (as yet) unrealized and indeterminate future. And since this future always
		brings new situations to realization, the present is ever the locus of novelty, innovation,
		and creativity.
		A preprogrammed world of the sort derogated by William James as a ''block universe" is
		anathema to processists. Like him, they react sharply against a predeterminist position
		where, everything is constrained to be by what has been and any replay of events would
		lead to the same preprogrammed result that precludes not only choice and chance but
		grants the dead hand of the past an absolute authority over the future. Such
		Page 75
		innovation-precluding preprogramming is wholly at odds with the creative openness in
		nature and in human affairs upon which process philosophers have always insisted. A
		Nietzschean world of eternal recurrence is anathema to processists. As they see it,
		processual novelty sets limits on the extent to which a particular domainbe it nature at
		large or human affairsfalls subject to the automatism of mechanized routines. They insist
		upon maintaining the creative impetus of natural process which abrogates what
		Whitehead somewhere called "the tedium of indefinite repetition" that leaves no room for
		novelty and rigidly circumscribes the world's processual nature.
		Stephen C. Pepper was a process philosopher who devoted much attention to clarifying
		the idea of novelty. As Pepper has it, novelty is a pervasive quality of the real. He
		distinguished three senses of the concept. First, there is the novelty of uniqueness. Every
		event differs from every other event merely by its occurrence; each is unique and hence
		novel simply by being this-here-now and no other. Second, we have the intrinsic vividness
		of occurrences "there is first the fresh event intrinsically glowing with novelty." 3 Yet
		because the novelties of uniqueness and vibrancy grow stale, a third sort of novelty
		arises: intrusive novelty. It is manifest in the unexpected and the jarring; and in its
		extreme case, it is tantamount to destabilization and conflict. Hence the category of
		creativity, as the source of novelty in all its senses, brings to the world process special
		hazards, and must therefore be supplemented with another categoryorder. Cosmic
		process, for Pepper, proceeds through the dialectic of novelty and order.
		Innovation by definition involves novelty, and very different things can be at issue here.
		Those of principal concern to process philosophy are
		ontological innovation: new things, processes, products, states of affairs
		(Note: "New" in the presently operative sense will mean new in kind)
		phenomenical innovation: new types of events or courses of events
		(natural or personal histories)
		epistemic (or conceptual or cognitive) innovation: new sorts of
		knowledge, ideas, information, problems, questions.
		Page 76
		Perhaps no form of novelty is more intractable from the predictive point of view than
		conceptual noveltyinnovations in ideas and conceptions. To predict a new sort of thing or
		event, we need not produce it; but to predictconceivably and specificallya new idea or
		conception, we must produce it at the time. By its nature epistemic innovation can only
		be predicted in abstracto and cannot be preindicated in its concrete detail.
		After all, human creativity and inventiveness defies predictive foresight. We realize full
		well that there will be unforeseen innovations in art, literature, technology, science,
		business, and other large-scale areas of human endeavor in the decades and centuries
		ahead. But of course we cannot predict what they will be. We can predict that new modes
		of communication will still be invented by the year 3000 but not what they will be. (To
		characterize them now in detail would be to invent them now.) The fruits of human
		ingenuity always come as a surprise (and frequently an unwelcome one). If national
		foresight could anticipate its operations, creativity as such would not exist. In particular,
		the course of discoveryof cognitive innovationis a major source of predictive incapacity.
		The prediction of novelty is obviously infeasible in those instances where specifying the
		purported innovation at issue involves anticipating it. The historian can say that "X was
		the first person to state (or realize) that the human body consists principally of water" but
		no predictive sage can foretell in advance that "X will be the first person to state (or
		realize) that the human body consists principally of water." Not only would any such
		prediction be automatically self-falsifying, but the claim at issue is literally paradoxical.
		The crucial fact here is that of impredictability. When we see an innovationwhatever it
		befirst emerging in nature's handwriting on the wall of existence, we cannot yet tell what
		is to come. Process philosophy does not deny the existence of patterns as suchquite on
		the contrary: It insists that processes themselves both instantiate and transmit structural
		patterns. What process metaphysics denies is the exclusive prevalence of inevitable
		preestablished patterns that make prediction unfailingly possible. It is against the
		background of an emphasis on the role of chance and choice in nature
	\end{quotation}
	
	\begin{quotation}
		4.3. Taxonomic Complexification
		From such a perspective, the role of processual innovation is a crucial factor of nature.
		And the result of such innovation is not just the organization of new things but the
		organization of new kindsnot just of individuals but of universals. The ongoing
		development of taxonomic complexity is a key feature of the world.
		
		This matter of radical novelty is one of the pivotal and characteristic themes of process
		philosophy. Whitehead, for example, saw in creativity a "category of the ultimateit is a
		"universal of universals" that is at work in every department of nature, everywhere
		bringing novelty to realization. 5 For him, creativity represented the very essence of
		reality: To exist is to be creative, in the first instance of oneself (self-creation) and by
		reflection influencing the self-creation of the other existents to which things are linked by
		process of interaction. With Whitehead, as with Bergson, it is through the impetus of
		creativity that all of the processes that constitute nature transpire, with each day seeing
		the drawing of a new phase in the
		world's creative advancethe ongoing transit of world history into pioneer territory where
		nothing and no one has been before.
		
		To be sure, it is certainly possible to minimize the significance of novelty by conceiving of
		the role of process in nature as consisting in a fixed number of elemental process types
		themselves fixed for all timethrough whose combination and interplay all other natural
		processes arise. But such a hard-edged, atomistically stabilitarian view of process does
		violence to the spirit of the enterprise of process philosophizing. Even as Darwin
		abolished the Aristotelian fixity of biological species, so process philosophy denies the
		unalterable fixity of the various species of natural kindsand, indeed, of natural processes
		at large. It sees processes not as coming in hard-edged inflexible types but as fluid and
		shifting, themselves in an ongoing course of processual development, with innovation and
		noveltythe fading of old types and the emergence of new onesas ever being the order of
		the day. Processuality does not happen simply at the ground-floor level of things, events,
		and phenomena. The types of items at issue can change as well. Darwin's discovery holds
		not just in biology but everywhere: The fundamental novelty at issue with creativity and
		the innovation of new kinds of species is pervasive. Reclassification is a constant
		necessity, since species, too, are transitory and impermanent, with the old opening the
		way for development of the new. The theory of evolution powerfully encouraged the view
		of the universe as a processual manifold rather than as assemblage of fixed and
		unchanging essences that perdure unaltered over the course of time. Short of introducing
		a deus ex machina of some sort, there is no machinery-internal reason why novelty and
		innovation should ever come into a domain of thingsas the difficulties encountered by
		substantialists from Epicurus to Newton attest. But with processes the situation is very
		different. For since processes are by nature transient and transformatory, they naturally
		make room for evolutionary development. After all, novelty emergence and innovation
		are themselves types of process.
		
		Process metaphysicians thus see nature as a stage setting for a creativity that engenders
		radical novelty. Such
		Page 82
		creativity clearly requires room for the alternativeness of indetermination. In general, the
		world's processes do not totally determine their successors and are not totally determined
		by the predecessors. 
		
		Process theorists accordingly maintain that nature exhibits some
		modicum of a causal slack that makes room for chance and choice, for novelty,
		innovation, and creativity. As they see it, nature's processes follow patternsbut not in a
		rigidly programmed and preordained predetermined way. The world constantly affords
		illustrations of pattern-breaking novelty and innovation. Accordingly, insistence on the
		"creative evolution" of new patterns, new modes of processnew universals, in shortis a
		characteristic doctrine of the school.
		
		\todo{! een systeem van kleine afwijking? Het valt op dat Rescher telkens op Darwins evolutie leer terug valt, is dat werkelijk het basis principe? Ik herken waarschijnlijk niet de impact van deze leer omdat ik de voorliggende visie, van aristotle over wat ik begrijp vaste order in biology niet ken. Ik ben opgegroeid met de evolutie leer en heb deze als een vast staand feit geaccepteerd..dan blijkt gelijk hoe moeilijk het is om een paradigma te wisselen. Heeft dat te maken met een uitwisseling van concepten? Alsof je de fundamenten van een huis wisselt. Is dat wat het paradigma afbakend een soort van basis van onze kennis en wijzigen betekend wijzigen van alles wat daar op gebasseerd is..een revolutie, een her programmering van je hersenen? Het heeft misschien te maken met hoe wij kennis pbouwen. Zou een kind bijvoorbeeld makkelijk verschillende paradigmas kunnen leren en kunnen switcehn tuseen deze. Dus heeft het met ons leer systeem te maken of met de wijze waarop onze hersenen werken en kennis opslaan. Is leren gebasseerd o de werking van de hersenen of zijn de hersenen geconditioneerd door ons leer systeem?}
		Process theorists accordingly maintain that nature exhibits some
		modicum of a causal slack that makes room for chance and choice, for novelty,
		innovation, and creativity. 
		
		As they see it, nature's processes follow patternsbut not in a
		rigidly programmed and preordained predetermined way. The world constantly affords
		illustrations of pattern-breaking novelty and innovation. Accordingly, insistence on the
		"creative evolution" of new patterns, new modes of processnew universals, in shortis a
		characteristic doctrine of the school.	
	\end{quotation}
	
	\subsubsection{8.1. Basic Ideas of a Process Philosophy of Nature}
	\begin{quotation}
		\todo{een mooi vergelijk pp en sp}
		8.1. Basic Ideas of a Process Philosophy of Nature
		We now turn to an examination of the characteristic features of a process philosophy of
		nature. The basic idea of the process is to view the world as a unified macroprocess that
		consists of a myriad of duly coordinated subordinate micro-processes. A clear contrastcase
		to such a view of the real is afforded by classical atomism based on the Democritean
		conception of atoms and the void. While the reciprocal relationship of such itemsatoms
		and the voidis supposed to give rise to motion (and so to processes), these fundamental
		resources themselves are unprocessual items that stand entirely outside the order of
		process, seeing that, as classically conceived, they themselves are wholly impervious to
		change. On the other hand, a physics based exclusively on fields and forces that operate
		on their own, without any embedding in substantial things of some sort, is the
		quintessence of a process philosophy of nature. These itemsfields and distributions of
		forceallow processual change to occur ''all the way through," so to speak.	
		
		\todo{zo ook het begrijpen van creativiyeit, gebruiken in de intro}
		But what advantages does such a process-geared view have to offer? The answer
		ultimately lies in the extent to which it enables us to synthesize and understand the
		cognitive phenomena that confront us throughout the study of the natural world we
		inhabit.
		
		The basic drive or nisus the
		world's furnishings is not so much (as with Spinoza) one of self-preservation (conatus se
		preservandi) as one of self-realization, of bringing itself to its own fullest actualization
		(conatus se realizandi). Creative self-determination rules: Everywhere there are
		processes at work bringing heretofore nonexistent beings and modes of being to
		realization.
		
		\todo{kan did worden gebruikt om de zuigkracht uit te leggen, de effiecient cause van een idee?}
		For we now become committed to the thesis that everything
		in nature has an efficient cause in some other natural thing that is its causal source, its
		reason for being.
		(The "efficient cause" of an object is equivalent to that which causes change and motion to start or stop (such as a painter painting a house) (see Aristotle, Physics II 3, 194b29). In many cases, this is simply the thing that brings something about.)
		
		A highly problematic bit of
		metaphysics is involved here. Dogmas about explanatory homogeneity aside, there is no
		discernible reason why an existential fact cannot be grounded in nonexistential ones and
		the existence of substantial things be explained on the basis of some nonsubstantival
		circumstance or principle. Once we give up the principle of genetic homogeneity and
		abandon the idea that existing things must originate in existing things, the whole
		enterprise of explanation assumes a different mien. Explanation can now proceedin
		theory, at leastin terms of
		processes rooted in the operation of fields or forces that pervade nature at large and in
		their turn engender the particularized powers and potentialities of identifiable things. 8
		
		Quantum mechanica ondersteund PP met verschijselen die niet te meten zijn en deeltjes die niet bestaan.
		
		In particular,
		quantum theory has it that at the level of the very small there are no ongoing "material"
		things (substances, objects) at all in nature, no particulars with a continuing descriptive
		identity of their own; there are only patterns of process that exhibit stabilities.
		
		Kritiek op Whitehead met zijn actual occasions, een process atomistic view van nasis eenheden. Waarom zis deze verklaring nodig, is het niet mogelijk om process en met processen te verklaren tot in het oneindige door, zonder een basis bouwsteen? 
		In this regard, however, Whitehead's approach in process philosophy was somewhat
		flawed. For Whitehead insisted upon irreducibly atomic units of process"actual
		occasions"themselves altogether indecomposible and serving as basic units or building
		blocks out of which all larger processes are then constituted.
		
	\end{quotation}
	
	\begin{quotation}
		In this regard as in others, Teilhard de Chardin was a typical process philosopher. For
		him, "the universe is no longer a State but a Process." 9 As Teilhard saw it, nature
		everywhere strives to produce something new in this world's scheme of things. Nature is
		not something fixed and given; it is "a world that is ever being born instead of a world
		that is"10change, development, and evolutionary emergence are the world's only
		pervasive and enduring features. We live in a world centering on a specious present that
		is ever in transit from a realized past to an open future: "Every particle of reality, instead
		of constituting a self-contained point in itself, extends from the previous fragment to the
		next along an inevitable thread running back to infinity."11 And not only do natural
		processes produce ''things," it is their no less crucial contribution to bring into being new
		types (species) and new patterns of order (laws). The evolutionary origination of new
		kinds is operative not in the biological realm above but throughout natureat every level of
		detail and size (micro and macro alike).
	\end{quotation}
	
	\begin{quotation}
		Process metaphysics accordingly stresses the developmental aspect of the real in ways
		which natural science enables us to grasp more concretely. On its approach, primacy
		belongs to active (productive) nature (natura naturans) at the creative frontier of the
		innovative present over against the finished realm of accomplished fact (natura
		naturata). The temporal aspect of an ever-new present that gives embodiment to a
		novelty-enhancing manifold of possibility and potential through concretization into a body
		of realized actuality lies at the forefront of process philosophy's conception of the natural
		world.
		It is a position that insists on seeing
		nature as a manifold of processesand a mixed manifold at thatone that includes sectors
		both of rigid causal determinism and open unforeseeability, since only a world that
		embodies chance and free choice can provide that "universe with elbow room" which
		William James envisioned as the indispensable setting of a satisfying human existence.
		
	\end{quotation}
	
	\begin{quotation}
		3. Process and the Laws of Nature
		Following the lead of C. S. Peirce, process metaphysics
		firmly rejects this contention. As it sees the matter, process invades the world's lawstructure
		as well; the laws of nature, too, are merely transitory stabilities that emerge at
		one phase of cosmic history only to lapse from creation and give way to variant modes of
		operation in the fullness of time.
		
		And so, not only do the world's phenomena change but
		so do the natural laws that govern their modus operandi. On this perspective, the world's
		only pervasive permanence is \textbf{change itself}. Even the so-called laws of nature are
		themselves little more than islands of relative stability in a sea of process.
		
		C. S. Peirce thought of laws of nature as settled but acquired habitsstable modes of
		operation that the universe has acquired over time and, once developed, retained for
		good. But given the prominence of chance and chaos in contemporary science, it might be
		more plausible to see the laws of nature as themselves constituting pervasive processes,
		consisting in transitory (and thus mortal) regularity patterns that hold for large sections of
		space-time and then give way, be it
		Page 93
		cataclysmically or gradually, to different process patterns. On such a view, the machinery
		of process provides a vehicle for the engendering of order in place of potential chaos.
		
		But how can order manage to preempt disorder? In theory, there are three principal
		ways:
		by imposition through some sort of causative agency acting ab extra
		by some sort of external persuasion (à la Plato's Timaeus)
		by some sort of internal development, some intrinsically generated mode
		of evolution-analogous emergence.
		Process metaphysics rejects the first two (externally initiated) alternatives.
		
		Accordingly, process
		metaphysics looks to a world-internaland, indeed, a process-internalorigination of order.
		
		To be sure, the forces of destabilization are at work, too. Indeterminism, unpredictability,
		and the emergence of novelty are all crucial to the process metaphysicians' view of
		things. But what they envision as central is not so much the organization of new things or
		types of things, but the ongoing emergence of new modes of comportmentnew types of
		process.
		
		Beginning with Leibniz, processists reject a departmentalized world that has separate
		compartments for physical, biological, and psychological sectors of nature. They regard all
		existence as prismatically many-sided.
		
	\end{quotation}
	
	\begin{quotation}
		4. Space-Time
		
		But there are two ways of conceiving of space-time: (1) as a container within (or a stage
		upon) which natural processes transpire, or (2) as itself an actual complex or state of
		process, a process-manifold of sorts that is itself constituted by the interlocking structure
		of individual process. The first perspective sees space-time as independent of the
		processes that happen with it, the second as itself an aspect of natural process, as a
		resultant part of the natural interrelatedness of such processes. And process philosophy
		adopts the second view.
		
		Its exponents abandon the Newtonian hypostatization of space
		and time, seen as fixed stable containers for nature's thingsa view that goes back to the
		Atomists of Greek antiquity and plays into the hands of substance philosophy. Instead,
		they adopt, with Whitehead, Leibniz's relativistic conception of space-time as a manifold
		defined by the structure of natural process itself (coordinate with the diffusion of
		electromagnetic signal-processes in nature).
		
		Time is typified
		by the outward flow of a wave pattern, and space by the stability of a configuration of
		standing waves. Space and time are, in the final analysis, no morebut also no lessthan
		inherent aspects of the characteristic interrelationship of physical processes.
		
		Accordingly, process metaphysics insists on the present status of temporality as
		manifesting the dynamism of an ever-innovative present. George Herbert Mead typified
		the processists' special stress upon the role of the present as representing the emergent
		frontier of existence in the world's process of development. "Reality," he emphasized,
		"exists in a present" 13 Each present is a locus of emergence that yields novelty in its
		actual being
		
		George Herbert Mead
		Mead specifically rejected Bergson's view that
		the present somehow accumulates all the past: ''The present," he said, "does not carry
		any such burden with it.15 Instead of total accumulation, the past merely serves
		as a generative condition of the present. "The actual passage of reality is in the passage
		of one present into another, where alone is reality, and a present which has merged in
		another is not a past. Its reality is always that of a present."
		
		In the natural philosophy
		of process, the idea of time stands correlative with a transient present of ever-changing
		creativity. Time is so central and important in process philosophy because temporality is
		the definitive characterizing feature of the processual nature of the real. To be real is to
		occupy a place in the order of time.
		
		\todo{the origin, get rid of god}
		Is the origin and nature of our processual world itself explicable? Processists part ways
		here. Some invoke the will of God (Leibniz). Others see things happening by an inner
		impetus of not so much logical as ontological necessity (Hegel). Yet others acknowledge
		the operation of brute chance (Laplace). And some allow matters to rest in the lap of
		mystery, agreeing with John Dewey: "We can account for a change by relating it to other
		changes, but existences we have to accept for just what they are. . . . The mystery is that
		the world is as it is."17 But sophisticated processists join forces with C. S. Peirce here.
		Given that human intelligence is a resource developed over time by creatures that are
		themselves an evolved part of nature, our capacity to understand the world should not be
		seen as all that surprising.18 Here as elsewhere, process philosophy can and does make
		the most of a recourse to the evolutionary process.
	\end{quotation}
	
	
	\begin{quotation}
		5. The Quantum Aspect	
		Twentieth-century physics has thus turned the tables on classical atomism. Instead of
		very small things (atoms) combining to produce standard processes (windstorms and
		such), modern physics envisions very small processes (quantum phenomena) combining
		in their modus operandi to produce standard things (ordinary macro-objects).
	\end{quotation}
	
	\begin{quotation}
		The quantum view of the world is inherently probabilisticindeed, it has trouble coming to
		terms with concrete definiteness (with the "collapse of the wave packet" problem). And
		this, too, is congenial to processists, seeing that process philosophy rejects a pervasive
		determinism of law-compulsion. Processists see the laws of nature as imposed from
		below rather than aboveas servants rather than masters of the world's existents. Process
		metaphysics envisions a limit to determinism that makes room for creative spontaneity
		and novelty in the world (\todo{creativity in substance view
		}be it by way of random mutations with naturalistic processists
		or purposeful innovation with those who incline to a theologically teleological position).
	\end{quotation}
	
	\begin{quotation}
		Moreover, process philosophers have reason to favor quantum physics over relativistic
		physics.
		...
		This serves to explain
		why Whitehead sought to provide a new theoretical basis to relativity theory and
		reconstrue space-time, as well as the conception of other physical objects, as being
		constructions made from "fragmentary individual experiences." 20
		
		Processes are not the
		machinations of stable things; things are the stability patterns of variable processes. All
		such perspectives of modern physics at the level of fundamentals dovetail smoothly into
		the process approach.
	\end{quotation}
	
	\begin{quotation}
		6. Process Philosophy and Evolutionary Optimism
		Evolution is thus an emblematic and paradigmatic process for process philosophy. For not
		only is evolution a process that makes philosophers and philosophy possible, but it
		provides a clear model for how processual novelty and innovation comes into operation in
		nature's self-engendering and self-perpetuating scheme of things.
		
		Change pervades
		nature. Process at once destabilizes the world and is the cutting edge of advance to
		novelty. And evolution of every levelphysical, biological, and cosmiccarries the burden of
		the work here. But does it work blindly?
		
		\todo{of key issue in creativity!! kies een zijde, change gedrreven of om reden van doel}
		On the issue of purposiveness in nature, process philosophers divide into two principal
		camps. On the one side is the naturalistic (and generally secularist) wing that sees
		nature's processuality as a matter of an inner push or nisus to something new and
		different. On the other side is the teleological (and often theological) wing that sees
		nature's processuality as a matter of teleological directness toward a positive destination.
		Both agree in according a central role to novelty and innovation in nature. But the one
		(naturalistic) wing sees this in terms of chance-driven randomness that leads away from
		the settled formulations of an established past, while the other
		Page 100
		(teleological) sees this in terms of a goal-directed purposiveness preestablished by some
		value-geared directive force.
		
		Process philosophy correspondingly has a complex, two-sided relationship with the theory
		of evolution. For secular, atheological processists, evolution typifies the creative workings
		of a self-sustaining nature that dispenses with the services of God. For theological
		processists like Teilhard de Chardin, evolution exhibits God's handwriting in the book of
		nature.22 But processists of all descriptions see evolution not only as a crucial instrument
		for understanding the role of intelligence in the world's scheme of things but also as a key
		aspect of the world's natural development.
		And, more generally, the evolutionary process
		has provided process philosophy with one of its main models for how large-scale
		collective processes (on the order of organic development at large) can inhere in and
		result from the operation of numerous small-scale individual processes (on the order of
		individual lives), thus accounting for innovation and creativity also on a macrolevel scale.	
	\end{quotation}
	
	\begin{quotation}
		Taken in all, cognitive evolution involves
		both components, superimposing rational selection on biological selection. Our cognitive
		capacities and faculties are part of the natural endowment we owe to biological
		evolution. But our cognitive methods, procedures, standards, and techniques are
		socioculturally developed resources that evolve through rational selection in the process
		of cultural transmission through successive generations. Our cognitive hardware
		(mechanisms and capacities) develops through Darwinian natural selection, but our
		cognitive software (the methods and procedures by which we transact our cognitive
		business) develops in a Teilhardian process of rational selection that involves purposeful
		intelligence-guided variation and selection
		
		Biology produces the instrument, so to speak,
		and culture writes the musicwhere obviously the
		Page 101
		former powerfully constrains the latter. (You cannot play the drums on a piano.)
		
		But
		nevertheless the overall course of processual change tends to the development of an ever
		richer more complex and sophisticated condition of things on the world's ample stage.
		
		\todo{er zit dus een "richting" aan de ontwikkeling, welke invloed heeft dat op creativiteit?
			Maar is het ook mogelijk om hobbels te nemen? Een soort wordt niet dood genoren maar zal zich als deze niet succesvol is verder ontwikkelen maar uitsterven (dan moet de soort wel eerst opbloeien!)
			Maar zal een rivier de beste weg vinden als er een berg voor staat, dan gaat deze om de berg heen maar zal deze het betere pad vinden dan de berg blokkeerd, o wordt de berg ondermijnd, verdwijnt en vind de rivier alsnog zijn weg?}
		
		Recognizing that this is so, process philosophy has always
		accentuated the positive and worn a decidedly optimistic mien. For it regards natures
		microprocesses as components of an overall macroprocess whose course is upward rather
		than downward, so to speak. Hitching its wagon to the star of a creative evolutionism,
		process philosophy sees nature as encompassing creative innovation, productive
		dynamism, and an emergent development of richer, more complex and sophisticated
		forms of natural existence.
		
		At every level of world historythe
		cosmic, the biological, the social, the intellectualprocess philosophers have envisioned a
		developmental dynamic in which later is better, somehow superior in being more
		differentiated and sophisticated. Under the influence of Darwinian evolutionism, most
		process philosophers have envisioned a course of temporal development within which
		value is somehow survival-facilitative so that the arrangements which do succeed in
		establishing and perpetuating themselves will, as a general tendency, manage to have
		done so because they represent actual improvements in one way or another. (A decidedly
		optimistic tenor has prevailed throughout process philosophy. 24 )
		
		As process philosophy sees it, the world's processuality involves not only change but
		improvementthe evolutionary realizationat large and on the whole, of what is not only
		different but also in some way better. Accordingly, novelty and fruitfulness compensate
		for transiency and mortality in process philosophy's scheme of things.
		
		Process philosophy centers around the idea of a natural processual dialectic that brings
		innovation and novelty into being at all points on the compass. But its capacity to depict
		this course of development as not just a matter of change but one of superiority in
		matters of complexity and sophistication is entirely dependent upon the use that
		processists make of the explanatory resources of evolutionary theory.
		25
		25. On these themes see George R. Lucas Jr., ''Evolutionist Theories and Whitehead's
		Philosophy," Process Studies 14 (1985): 287300; and also the chapter "Evolution and the
		Emergence of Process Metaphysics" in his Rehabilitation of Whitehead. Lucas highlights
		the role in Whitehead's metaphysics of the salient ideas of evolutionary cosmology.
		
		
	\end{quotation}
	
	\begin{quotation}
		7. Validation
		
		To be sure, at the level of generality of a rather elevated abstraction, a process
		philosophy of nature and a traditional substance philosophy of nature along the lines of a
		classical atomism exhibit some similaritiesfor example, both see the world as a manifold
		of identifiable things that are identifiable particulars located in space-time and Interacting
		under the aegis of causal laws. But the crucial fact is that each and every one of
		the conceptions involved here is interpreted differently in the two camps. With respect to
		"manifold of things," "identifiable particulars," "space-time location," and "causal
		interaction" are provided with a totally different interpretation. And for process
		philosophy, the crucial consideration is that its own interpretations are throughout better
		attuned to the spirit and letter of contemporary natural science than those of its
		substantialist rival. Indeed, process philosophy sees this factor as so pivotal an asset as
		to stake upon it the main burden of its claims to acceptability.27
	\end{quotation}
	
	
	
	\begin{quotation}
		6
		Process and Persons
		
		As
		G. H. Mead stressed,6 the community-sustaining role of communication among social
		beings such that it is effectively impossible to study them sensibly in isolation, abstracting
		from the interpersonal relations that shape virtually the whole spectrum of their activities.
		\todo{en dus can je een creatve persson niet in een isolement nemen maar moet je de omgeving er bij betrekken}
	\end{quotation}
	
	\subsubsection{7 Process Logic and Epistemology}
	\begin{quotation}
		
		1.Truth and Knowledge: The Processual Perspective
		
		\todo{kan wat voor knowledge geld van Dewey, de interactie met de omgeving ook voor creativiteit gelden? Bv in IT bezig zijn voor de idee? Sluit aan bij dat een idee ontwikkeling nodig heeft}
		Process philosophers and pragmatists alike
		haw always stressed the status of knowledge as a process. John Dewey, in particular.
		dismissed the "spectator theory of knowledge," insisting that the acquisition and
		management of factual knowledge is always a matter of interacting with our
		circumambient human and natural environment. We come here to a distinction that has
		far-reaching implications.
		
		To see
		information as a literal product (a substance of some sort) is to commit the fallacy of
		improper reificationof "misplaced concreteness," as Whitehead called it
		
		It is instructive in this context to contrast three different approaches to the theory of
		knowledge. 
		First is the theory of representative, cognition (in the tradition of Descartes
		with his mind/matter dualism), which sees true thought about factual matters as
		somehow representing a totally nonmental state of affairs. (This winds up in the
		bankruptcy of the Kantian thing-in-itself.) 
		Second is the theory of constructive cognition,
		which (in the best idealistic tradition) takes the idealistic line that in true thought we are
		dealing with mind-engendered constructionseither a theological construction, as with
		Berkeley, or a naturalistic construction of inquiring individuals or communities, as with Hegel. (This winds up with the artificiality of the various versions of all-out
		idealism.) 
		The prime remaining possibility is the causal-commerce theory of cognition,
		which sees true thought and its object as simply two phases of the real caught up in a
		process of mutual interconnection. This approachfavored by processismof factual
		knowledge in terms of a harmonious overall manifold of causal commerce in which the
		"external" order of nature and the "internal" order of physiologico-psychological come into
		suitable alignment. The traditional representational and constructive theories of cognition
		are now not so much refined as replaced.
	\end{quotation}
	
	\todo{hier verder 2. Aristotle and Truth-Value Indeterminacy}
	
	\todo{Vraag. Wat zet het intstaan van een nieuw process in gang. maw Wat is het startpunt van creativiteit/van nieuw. Is er een Impetus dat alles aandrijft? Brengen processen elkaar aan de gang. Waaruit, waardoor, waarom onstaat een nieuw process}
	
	\begin{quotation}
		The fluidity of natural process (its "analog' nature and its kinship to
		the continuities captured by differential equations and the mathematics of infinitesimals)
		is out of alignment with the rigidities of everyday language (with its "digital,'' yes-or-no,
		on-or-off nature). Recourse to a grey region of truth-status Indeterminacy is accordingly a
		useful mediating device for handling those aspects of the world's, modus operandi that
		process philosophy is concerned to emphasize. So useful, indeed, is this device that
		process philosophers would have had to devise it themselves had Aristotle not already
		done so for them.
	\end{quotation}
	
	\begin{quotation}
		But real
		things always have more experientially manifestable properties than they can ever
		actually manifest in experience. The experienced portion of a thing is just the part of the
		iceberg that shows above water. All real things are necessarily thought of as having
		hidden depths that extend beyond the limits not only of experience but also of
		experientiability.
	\end{quotation}
	
	\begin{quotation}
		4. Process and Experience
		Idea formation is a salient and characteristic capacity of intelligent beings, and
		experiencethat is to say, the interaction between minds and natureis the pivotal mode of
		process here. The empiricists were right at any rate in this: that all our ideas have a basis
		in experience. But they were wrong in thinking that this experience had to be sensory
		rather than more broadly speaking intellectual. As Kant stressed, empiricists neglected
		the crucial fact that all our experience is ideational (or conceptual) in nature.
		
		
		This aspect of the cognitive enterprise is brought to the fare in the relationship between
		observation and process that was emphasized by A. P. Ushenko in his Power and Events.
		What we actually experience, he maintained, is not a matter of
		Page 133
		the sense data of earlier perception theorists, emplaced in a static copresence like letters
		to a printed page. We experience the powers of thingsnot just the engine but its
		throbbings. not just the flames but their radiating warmth, not just the redness of Smith's
		face but his anger. We observe not just things but their activities and operations. And this
		brings their processtheir causal efficacyinto the orbit of experience. Human observation is
		a commerce with the world's processes, a dynamic encounter that is not just a matter of
		a noting of its fixed features.
		
		\todo{language is limited, wat is daarvan de invloed op creativiteit? Ondekken we steeds nieuwe details die ons aanzetten tot andre ideen. Details die ons pas nu, in het heden duisleijk worden door de nieuwe stand van de techniek?}
		At any given temporal juncture, language is something limited and finite. But reality
		outruns any such limitation. Even with such familiar things as birds, trees, and clouds, we
		are involved in a constant reconceptualization in the course of progress in genetics,
		evolutionary theory, and thermodynamics. Our conceptions of things always present a
		moving rather than a fixed object of scrutiny, and this historical dimension must also be
		reckoned with.
		
		\todo{wat is een idee in de pp cosmos? is onderzoek altijd concept vernieuwing?}
		Not only is the substance of our knowledge (information, facts) deeply processual, but so
		are the very ideas that are at work here which, after all, themselves operate processually
		defined states of the cognitive art. Any adequate theory of inquiry must recognize that
		the ongoing process of information acquisition at issue in science is a process of
		conceptual innovation, which always leaves certain facts about things wholly outside the
		cognitive range of the inquirers of any particular period.
		..
		\todo{dit geld denk ik ook voor TI}
		To
		grasp such a fact means taking a perspective of consideration that we as yet simply do
		not have, since the state of knowledge (or purported knowledge) has not reached a point
		at which such a consideration is feasible.	
		
		For present purposes, the crucial consideration is that our knowledge of reality itself has
		an operational/practical and thus processual dimension. One of the most significant and
		characteristic kinds of know-how is the knowledge of how to operate at the level of theoryhow to conjure with theoretical knowledge-that over the
		range from obtaining it to using and conveying it.
		
		Here we come to the whole domain of
		securing, recording, communicating, and processing information. Achieving any sort of
		knowledge-that is itself one of the most extensive and significant forms of praxis in which
		our species is involved. And, of course, praxis is by its very nature something processual.
		Any adequate worldview must recognize that the ongoing progress of scientific inquiry is
		a process of conceptual innovation that always leaves various facts about the things of
		this world wholly outside the cognitive range of the inquirers of any particular period.
		Ideas cannot be grasped before their timebefore the developmental unfolding of the
		beliefs and notions that alone can provide an intellectual entryway into their domain. The
		novelty that arises with the emergence of new cognitive processes is crucial both to the
		nature and to the availability of our ideas. This dynamics of ideas strikingly marks the
		processual aspect of epistemology. Throughout the cognitive enterprise we are
		confronted with an ever-changing state of the art. Knowledge is not a thing, let alone a
		commodity of a fixed and stable make-up; it is irremediably processual in nature, affected
		as deeply by the fluid nature of reality as anything else.
		\todo{hoe kom je tot een idee. Is dat gebasseerd op kennis? Altijd? De kennis dat mensen iets willen..de kennis van een probleem. Waarvoor dient technologie? Is dan kennis van dat doel wat maakt dat nieuwe ideen ontstaan?}
		\todo{uitzoeken wat het doel van technologie is, waarom en waarvoor nieuwe technologie?}
	\end{quotation}
	
	\begin{quotation}
		5. Process and Communication
		
		And this processuality holds not just for the development and comprehension of
		knowledge but for its communication as well. In general, processes haw three phases or
		stages:
		initiating precursors Þ the process itself Þ resultant successors.
		Processes standardly exhibit such a pattern of sequential order, be it a temporal or a
		conceptual order that is at issue. And in the case of specifically communicative processes
		we have
		informative inputs Þ the communication itself Þ informative outputs.
		
		\todo{Het verspreiden van nieuwe kennis o bevestiging te krijgen lijkt misschien wel op het verspreiden van een idee om een waarde oordeel te krijgen. Als novelty een waarde moet hebben om als creatief gewaardeerd te kunnen worden dan zullen andere mensen er van op de hoogte moeten zijn. Ik denk zelfs dat zij die een idee goed kunnen verkopen als meer creatief aangemerkt zouden kunnen worden. Behalve als het gaat om de eigen beleving van creativiteit, zie creativity as an experience. Of ga ik die uitsluiten..om te kunnen focussen?}	
		This view of communication as a process of information transmission across the diversity
		of persons and times is characteristic of a process epistemology which sees this sort of
		convenience in information not as a matter of manipulating with ideational objects
		(beliefs, messages, or the like) but rather as one of conveying information with a view to
		its use by oneself and others within a characteristic range of processual operations.7 After
		all, the communication of information that is indispensably necessary to the acquisition
		and confirmation of knowledge in a communal setting is clearly also a thoroughly
		processual enterprise. And, moreover, the interpretation of communicated symbols the
		transformation of physical signals into informative messages is itself inevitably a process,
		one to which the recipient also has to make an active contribution.	
	\end{quotation}
	
	\begin{quotation}
		\todo{geld dit ook voor TI? loopt de taal achter..twitter/facebook..was daar al taal voor? Of zijn dit geen creative ideen/technologien}
		Our conceptions are too
		inflexible and inert to accommodate an ever-changing reality. Yet the net effect is the
		same either way: Our human conceptual drinking always involves some distortion of fact,
		some unfaithfulness to reality. And here the role of metaphor is crucial. At bottom,
		linguistic communication as we know it builds on a comparatively small basis of
		practical]y geared literal language use. But this is infinitely enriched and amplified by
		metaphorical flightsprocesses of assimilation that extend its communicative resources
		into every direction. (And note that this very sentence itself manifests the very process at
		issue in its use of "implement," "flight." and "extend in a direction.") The metaphorical
		nature of linguistic communication is yet another aspect of the many-sidedly processual
		nature of human cognition. 8
		
		Human knowledge is geared
		to activity; its terms of reference are given by verbs like asserting, questioning,
		understanding, communicating, and so forth.
		..
		To
		make substancelike "things" of cognitive and communicative matters is once again to
		commit Whitehead's fallacy of misplaced concreteness.
		
		\subsubsection{8 A Processual View of Scientific Inquiry}
		\todo{misschien is dit bruikbaar om te zetten naar TI}
		Human cognition is something that
		actively develops, and with this development we have not only change but the
		emergence of a novelty that always divides the present from what has gone before.
		
		The crux of process epistemology lies in its (clearly right-minded) insistence on seeing
		the enterprise of rational inquirybe it in natural science or elsewhereas being a process. It
		is concerned not with the embodiment of information and knowledge as a physical
		artifact along the lines of a book or blueprint or library catalog but with the living and
		changing dynamics of such materials.
		
		The equilibrium achieved by natural science at any given stage of its development is
		invariably an unstable one. The history of science shows all to clearly that many of our
		scientific theories about the workings of nature have a finite lifespan. They come to be
		modified or replaced in the light of further investigation effected through improved
		techniques of experimentation, more powerful means of observation and detection, superior procedures
		for data processing, and the like. The "state of the art" of natural science is a human
		artifact that, like all other human creations, falls subject to the ravages of time.
		
		The landscape of science is ever changing. As fallibilists since C. S. Peirce's
		day haw insisted, we must acknowledge an inability to attain definitive truth in scientific
		matters.
		
		And
		this impermanencethis vulnerability to file pervasive dominion of chaos over the things of
		this worldholds just as much for our intellectual constructions as for our physical
		structures.
		
		The future of science is an enigma. Innovation is the very name of the game. Not only do
		the theses and themes of science change but so do the very questions.
		
		Innovation is the order of the day in science, and surprises are unavoidable.
		
		New scientific
		questions arise from answers we give to previous ones, and thus the issues of future
		science simply lie beyond our present horizons.
		
		Not only can one never claim with confidence that the science of tomorrow will not
		resolve the issues that the science of today sees as intractable, but one can never be
		sure that the science of tomorrow will not endorse principles that the science of today
		rejects.
		
	\end{quotation}
	3. Scientific Progress Is Driven by Technological Escalation
	
	In natural science, the driving force of innovation is provided
	by technology, for observation and experimentation both need to be technologically
	mediated.
	
	Observational, experimental, and information-processing
	technology is driven on an endless quest for substantial (order-of-magnitude)
	improvements in performance with regard to such information-providing parameters as
	measurement exactness, data-processing volume, detection sensitivity, high voltages,
	and high or low temperatures. Throughout the natural sciences, technological progress is
	a crucial requisite for cognitive progress.
	
	\todo{Is dit ook zo bij technology? Is er steeds een nieuwe behoefte die nieuwe technology naar voren roept die weer nieuwe behoeftes oproept? Veroorzaakt technology zelf de vraag naar nieuwe technology, komt er nooit een evenwicht? Nee omdat we altijd meer of anders willen, verder willen? Verder kunnen. Een sneeuwbal effect}
	They endeavor to create, once more, an
	equilibrium between theory and data. And then the ball returns to the experimentalists
	court, and the whole process starts over again with its demand for new and more
	powerful technological capacity.
	
	For not only the procedures but also the products of scientific inquiry are inherently
	processual, subject to the process-characteristic phases of initiation, development, and
	decline. The world as science teaches us to see itis both pervasively variable to our
	perspectual sight and itself in process of unending development
	\begin{quotation}
		
	\end{quotation}
	
	\begin{quotation}
		10
		Process in Philosophy
		
		The philosophy of process is also a
		philosophy in process. As yet,
		Page 168
		we do not have a fully developed and adequately articulated process philosophy in hand.
		For all that has been said and done by its exponents in various places and times, process
		philosophy is still in many ways incomplete and imperfect.
		
		Insofar as process philosophy is true to itself, it will have to proceed in its own terms of
		reference. For a processist to hold that it has arrived at a fully fixed set of categories or a
		definitive array of explanating principles would be treason to the spirit of the enterprise.
		The complexity and volatility of experience precludes finality. A framework of thought
		that is to be adequate to experience must forgo claims to completeness, for the very idea
		of definitive correctnessof achieving at once the generality and precision that is desiredis
		rendered infeasible by philosophy's unavoidable reliance on imperfect instrument of
		human language. All philosophizing is a matter of imperfect approximation.
		
		Interestingly enough, the student looking for elaborate evaluations and detailed criticisms
		of process thought will come away disappointed. While process ontology itself is a critical
		position that finds its reason for being in the critique and refutation of a substantialist
		approach, it itself has not undergone much critical examination. Those who are
		disinclined to its teachings have simply gone their own way, elaborating as best they can
		their own substance-oriented positions without worrying about the processual alternative
		processists.
		
		The main exception to this rule is one that was examined at length abovenamely, P. F.
		Strawson's putative refutation of processism in his influential book, Individuals, 3 which
		does indeed subject fundamental commitments of process philosophy to a frontal-assault
		critique.
		
		\todo{voor de intro van pp..dat het een benadering is en niet een thesis theory. Veel is blijkbaar nog niet vastgelegd in theses}
		As
		we have seen, it is rather a general line of approach than a particular philosophical thesis
		or theory
		Theses and theories are definitethey can be pinned down
		and critically dissected. But approaches are too plastic, too mercurial to admit
		conveniently of critical evaluation.
		
		This expectation is, in fact, fully borne out by circumstances that prevail,
		seeing that 
		\textbf{philosophy is an ongoing process of conjecture that exploits the data of
			experience to resolve the "big questions" that represent the traditional focus of the
			enterprise. 
		}For experience is an ever-changing landscapeparticularly in the present
		context of philosophical experience, where the deliverances of past philosophizing always
		form part of that experience to which present philosophizing must address itself. A
		process-oriented view of philosophizing is accordingly very much in order.
		
		In sum, philosophy as such is less an object
		than a process; its crux is not the stable formulation of intentions but the active, living
		development and interplay of ideas.
		Moreover, philosophizing is a teleological process; it is designed with an end in view.
		For,
		as has been stressed throughout these deliberations, philosophy is a purposive enterprise
		aimed at problem solving, at providing answers to our questions regarding the world's
		scheme of things and our place within it.
		
		\todo{Vergelijk pp en substance phil.}
		Substance, insofar as we can develop a cogent theoretic account of the
		matter, cannot be freed from a recourse to process (neither as regards
		its inner nature nor as regards our knowledge of it).
		A rigorously construed substance metaphysics cannot as such explain
		agency and change, whereas processes are by nature self-potentiating
		and issue further processes (i.e., changes).
		Moreover, a process approach to the nature and existence of substances
		can be successfully implemented.
		The identity and identification of substance is inextricably bound up with
		the occurrence of processes.
		The process approach averts or minimizes the difficulties inherent in the
		traditional problem of universals.
		The process perspective is smoothly attuned to and substantially
		consonant with the view of nature (physical, biological, and social) that is
		articulated in modern science.
		The process approach provides a more natural account of persons and
		personhood than the substance approach makes available.
		The process approach yields an explanation of acquisition, development,
		and management of information in a way that gives process
		epistemology marked advantages.
		The process approach provides an effective framework for better
		understanding both the conduct and product of rational inquiry.
		Process theology makes it possible to avert some of the perplexities and
		theoretical anomalies of a substance approach to God.
		The nature of philosophy and philosophizing itself is best understood as
		the basis of a process perspective.
	\end{quotation}
	
	
	\subsubsection{Appendix: Process Semantics}
	\begin{quotation}
		In general, a process semantics should beand isin a position to accomplish with verbs and
		adverbs whatever a semantics of individuals can manage to do with properties and
		relations. When the one says "X has the property F", the other says "X functions F-ly" for
		some suitably F-corresponding process. (Thus, "X is triangular' becomes "X disports itself
		triangularly." And where the one says ''X is north-of-Y" becomes "X locates itself to-thenorth-
		of-Y-ly.)
		
		
	\end{quotation}
	
	%------------------------------------------------------
	%------------------------------------------------------
	the journal Process Studies,
	
	\todo{Samenvatting van 4.2 maken}
	Sectie \textbf{4.2. Novelty, Innovation, Creativity} is een belangrijke sectie voor deze thesis.
	
	
	\todo{Geeftantwoord op een aantal vragen}
	\begin{quotation}
		The classical, Platonic view of universals is that they constitute a fixed and unchanging
		realm. Here, process metaphysics takes a very different line, adopting the ideas of C. S.
		Peirce. Peirce saw the opening up of new domains of phenomena in developmental
		terms. Early in world history, before the evolution of complex molecules, there was no
		place for biological laws; in the era of Neanderthal man there was no room for political
		economy. As the cosmos grows older, new modes of natural organization gradually
		evolve to afford new phenomena that are governed by emerging laws of their ownlaws
		that previously had no opportunity to come into operation. There is progression from laws
		of individual physical particles to laws of increasingly elaborate organized complexes
		thereof. Since the universe affords a varied panorama of modalities of physical process
		evolving over time, a science that reflects this will continue to find new grist for its mill.
	\end{quotation}
	
	\begin{quotation}
		
	\end{quotation}
	
	\begin{verbatim}
	@book{rescher2007philosophical,
	title={Philosophical dialectics: An essay on metaphilosophy},
	author={Rescher, Nicholas},
	year={2007},
	publisher={SUNY Press}
	}
	\end{verbatim}
	
	\begin{quotation}
		Philosophical distinctions are thus creative innovations. There is
		nothing routine or automatic about them—their discernment is an act of
		inventive ingenuity. They do not elaborate preexisting ideas but introduce
		new ones. They not only provide a basis for understanding better something
		heretofore grasped imperfectly, but they shift the discussion to a
		new level of sophistication and complexity. Thus, to some extent they
		“change the subject.” (In this regard they are like the conceptual innovations
		of science, which revise rather than explain prior ideas.)
		
		The continual introduction of new concepts via new distinctions
		means that the ground of philosophy is always shifting beneath our feet.
		New distinctions for our concepts and new contexts for our theses alter the
		very substance of the old theses. The development is dialectical—an exchange
		of objection and response that constantly moves the discussion
		onto new ground. The resolution of antinomies through new distinctions
		is a matter a of creative innovation whose outcomes cannot be foreseen.
		
		Unlike creative activity in music
		or the fine arts, philosophizing is not a matter of transiently all-embracing
		styles but of ever-recurrent doctrines.
		
		Philosophizing is a creative
		process, and in philosophy, as in natural science, there is and can
		be no substance-oriented “law of inner development” with regard to
		matters of substantive detail.
		
		Exhibit B: Process Atomism
		A. N. Whitehead’s philosophy of process, as expounded in his
		1929 masterwork, Process and Reality,13 was one of the most important
		and influential metaphysical doctrines projected in the twentieth
		century. Now, be it rightly or wrongly, some process theorists regard
		it as a fundamental feature of Whitehead’s thought that there are in
		nature elemental atomic processes from which all other processes can
		be constituted. They see Whiteheads’s “actual occasions” as them-
		selves altogether indecomposible units that serve as building blocks
		out of which all larger processes are then constituted in successive
		layers of concatenation. (Whether Whitehead himself held such
		a view is discussable but by no means certain.)
		
	\end{quotation}
	
	\begin{quotation}
		especially non-Whiteheadian process thought, does not clearly stand out as a unified effort at theory revision
	\end{quotation}
	\begin{quotation}
		Some process philosophers (e.g. Whitehead) took organismic processes as their central model for a concept of occurrence that generates internal and external coherence of an entity. Others (e.g. James) chose as their canonical illustrations individual psychological processes, or (e.g. Alexander) took evolutionary development as paradigmatic. Some process philosophers (e.g. Whitehead and Morgan) articulated their approach in the form of an axiomatic theory and in close relation to science, while others (e.g. Bergson) worked from an almost mystical sort of sympathetic apprehension of reality and insisted that process metaphysics could, if at all, only be expressed by means of a highly metaphorical use of language. Some processists (e.g., Roger Boscovich) championed a materialist position, while others endorsed idealism (e.g., Leibniz and Hegel).
	\end{quotation}
	\begin{quotation}
		need to be supplemented by introductions to, e.g., contemporary Whiteheadian process thought and French process thought (Gilles Deleuze, Alain Badiou). In addition, pointers to process-philosophical contributions to the philosophy of religion are all but omitted, since extensive and in-depth treatments are provided in the entries on Charles Hartshorne and process theism.
	\end{quotation}
	\begin{quotation}
		(iii) Processists also have offered novel approaches to the problem of persistence, either by taking persistent entities to be “enduring” patterns of processes (Whitehead), 
	\end{quotation}
	\begin{quotation}
		In some process organizations (e.g., far-from-equilibrium systems) each component process presupposes every other for its own occurrence; in the context of these particular process organizations the dependence amongst the single processes is not merely a matter of linear causation but constrained by the simultaneous interactions of the entire system, ensuring that each process is ‘functional for’ the occurrence of the system (Bickhard 2004)
		
	\end{quotation}
	
	
	Het gaat mij om hoe nieuwe ideen ontstaan. En hoe we daarop komen.
	Ik zou dan een model van de werkelijk moeten hebben waarin blijkt waaruit ideeen ontstaan en een model van de mind waaruit zichtbaar wordt hoe deze de nieuwe ideen ontdekt of bedenkt. \todo{ Rdb als creativiteit een proces is dan wordt het wat, zo niet dan wordt het nx.}
	
	
	Is het zo dat er mogelijkhedden ontstaan die vervolgens ontdekt moetn worden of creeren wij die mogelijkheden door een idee te bedenken en de voorwaarden vervolgens te scheppen??
	Ik zeg al enige tijd dat wat er is het gevolg is van een evenwicht, er is een ruimte ontstaan dat vervolgens gevuld wordt.
	Waarom ontstaat precies die ene ruimte en hoe wordt deze ontdekt. Is het een soorvan water dat zijn weg zoekt naar het laagte punt? Vinden we dan niet de ideen die achter een heuvel liggen?
	
	\todo{Rdb het een sluit het ander niet uit.het s geen of vraag maar een hoe en waarom vraag. Of vragen beperken de vrijheid van het antwoord. Probeer dus open vragen te stellen om de mogelijkheden van het antwoord breed te houden}
	
	Welke ideen kunnen ontstaan, alleen die die een link met het verleden hebben op een of andere wijze of alleen ideeen die aan de mens gerelateerd zijn, aan zijn lichaam, of dat de motoriek is, de sensoren als gehoor en gzicht of aan werkingen van de hersenen. Licht het an de ons bekende omgeving? Wie komt er bv op het idee om naar jupiter te reizen als we jupiter niet kennen. Zijn zwarte gaten een hersen spinsel of het antwoord op een probleem dat is ontstaan tijdens het verkenne van onze omgeving, in dit geval de ruimte?
	
	Hoe kom ik p een nieuw idee. Is het om een pronleem op te lossen of kmt er spntaan iets op dat nergens mee in verband te brengen is?
	Komt het vanuit mijzelf of komt het als reactie op de omgeving?
	Techno idee -> dan is het om een bepaalde taak of funtie te verrichten. Is dat om meer mogelijk te maken, dus wat ik kan uit te breiden of om iets wat ik al doe en kan te vergemakkelijken? Technology heeft niet alleen een practische functie maar ook een signaal functie.
	Omschrijven van ik met technology bedoel...
	
	Zoeken op PP en creativiteit.

\section{Which author(s) as basis}
	inurl:www.informationphilosopher.com creativity
	
	\subsubsection{vWhitehead}
	creativity is een basis principe, begint ij god.
	
	Internet encyclopedie of philosophy
	\begin{quotation}
		. Basic Metaphysics
		
		a. Creativity as Ultimate
		
		Whitehead argues that the best description of ultimate reality is through the principle of creativity. Creativity is the universal of universals—that which is only actual in virtue of its accidents or instances. Thus, creativity is frequently compared to the notions of Aristotle’s “being qua being,” Martin Heidegger’s “Being itself” (more appropriately “Becoming itself”), or even the material cause of all events. Creativity is the most general notion at the base of all that actually exists. Thus, all actual entities, even God, are in a sense “creatures” of creativity.
		
		\todo{belangrijk}
		Whitehead also characterizes \textbf{creativity as the principle of novelty}. The events of the past are ceaselessly synthesized into a new and unique event, which becomes data for future events. “The many become one, and are increased by one,” (Whitehead, Process and Reality, 20). This focus on oscillation between one and many forms the foundation of the process metaphysic.
	\end{quotation}
	
	Wikipedia
	\begin{quotation}
		Whitehead's influences were not restricted to philosophers or physicists or mathematicians. He was influenced by the French philosopher Henri Bergson (1859–1941), whom he credits along with William James and John Dewey in the preface to Process and Reality.[6] Process philosophy is also believed[by whom?] to have influenced some 20th-century modernists, such as D. H. Lawrence, William Faulkner and Charles Olson.[citation needed]
		
		The ultimate abstract principle of actual existence for Whitehead is creativity. Creativity is a term coined by Whitehead to show a power in the world that allows the presence of an actual entity, a new actual entity, and multiple actual entities.[10] Creativity is the principle of novelty.[9] It is manifest in what can be called 'singular causality'. This term may be contrasted with the term 'nomic causality'. An example of singular causation is that I woke this morning because my alarm clock rang. An example of nomic causation is that alarm clocks generally wake people in the morning. Aristotle recognizes singular causality as efficient causality. For Whitehead, there are many contributory singular causes for an event. A further contributory singular cause of my being awoken by my alarm clock this morning was that I was lying asleep near it till it rang.
	\end{quotation}
	
	\subsubsection{William James}
	http://www.informationphilosopher.com/solutions/philosophers/james/
	\begin{quotation}
		Here James is influenced by the pragmatic ideas of his colleague Charles Sanders Peirce, who saw three levels of thought - abduction (hypotheses based on pure chance), induction (working through many examples), and deductions (drawing the right logical conclusions)
		
		We find that William James was the first of two dozen philosophers and scientists who have proposed a two-stage model for free will and creativity.
		
		The first stage involves chance that generates alternative possibilities for action. 
		The second stage is an adequately determined choice by the will.
		
		First chance, then choice. First "free," then "will." 
		Compare the Cogito Model.
		
		Our thoughts come to us freely. Our actions go from us will fully.
	\end{quotation}
	
	
	\subsubsection{samuel Alexander}
	
	Alexander describes the process of emergence:
	The facts can best be descibed as follows. New orders of finites come into existence in Time; the world actualy or historically develops from its first or elementary condition of Space-Time, which possesses no quality except what we agreed to call the spatio-temporal quality of motion. But as in the course of Time new complexity of motions come into existence, a new quality emerges, that is, a new complex possesses as a matter of observed empirical fact a new or emergent quality. The case which we are using as a clue is the emergence of the quality of consciousness from a lower level of complexity which is vital. The emergence of a new quality from any level of existence means that at that level there comes into being a certain constellation or collocation of the motions belonging to that level, and this collocation possesses a new quality distinctive of the higher complex...
	(Space, Time, and Deity (1920), vol. 2, p. 45)
	
	\subsubsection{C. Lloyd Morgan}
	In his 1920 book Space, Time, and Deity Samuel Alexander cited Lloyd Morgan as the source of emergentism, and wrote:
	
	Mind is, according to our interpretation of the facts, an 'emergent' from life, and life an emergent from a lower physico-chemical level of existence.
	(Space, Time, and Deity (1920), vol. 2, p. 14)
	Later, in his 1922 Gifford Lectures and 1923 book Emergent Evolution, Lloyd Morgan defined emergent evolution and introduced the related "top-down" concept of hierarchical supervenience:
	
	...in the physical world emergence is no less exemplified in the advent of each new kind of atom, and of each new kind of molecule. It is beyond the wit of man to number the instances of emergence. But if nothing new emerge - if there be only regrouping of pre-existing events and nothing more - then there is no emergent evolution.
	Under emergent evolution there is progressive development of stuff which becomes new stuff in virtue of the higher status to which it has become raised under some supervenient kind of substantial gotogetherness.
	
	(Emergent Evolution (1923), pp. 1-6)
	
	\subsubsection{Henri Bergson}
	\subsubsection{Roger Boscovich}
	\subsubsection{Gilles Deleuze}
	\subsubsection{Alain Bafiou}
	\subsubsection{Charles Hartstone (religie)}
	\subsubsection{Mark Bickhard}
	\subsubsection{Charles S. Peirce}




\end{document}

