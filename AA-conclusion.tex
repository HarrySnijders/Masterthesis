\chapter{Conclusion}

\begin{verbatim}
IT-> easier/faster access to existing ideas and concepts (patents and copy rights are obstructions)
IT-> virtual world, free building tools (Java), low machine investment (PC, internet access) -> 
high user participation, small companies, building for fun, free non commercial usage (Hippel)
IT-. High interaction grade (smartphone, youtube, facebook, twitter) -> fast dispersion of ideas
IT-> no contemplation, overwelming information supply changing fast
IT-> new, people have to learn how to use it?
IT-> process philosophy -> every thng is connected via internet WIFI etc, concept of rocess comes close to the workings of a IT program. Relations can be process automatically-> big data?
IT-> process philosophy -> would it be possible to build, immitated, reality in IT processes? and predict the outcome? In fact this is what we do with the weather system for example. Is not IT a model cennecting to PP? Data = matter, process and relations.
Only the program does not change itself-> processes and relations are fixed. People build new programs (Processes) make relations and store data. It is possible to model this with physical things but much harder to make connections. Software is very easy to change compared to hardware, unless changes have go deep in protocols that could influence existing programs.

IT-> Ideation support systems, visualizing concepts, idea generation

IT-> maybe discuss if computers can be creative? Or skip because not part of the research?

It-> Could be seen as replacing the brain wherease machines replace the body? Is that dualism?
\end{verbatim}

\section{Provisional section}