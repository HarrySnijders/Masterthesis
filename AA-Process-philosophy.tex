
%=== indicates the definition of words and concepts
\chapter{Process Philosophy (What is process Philosophy)}

%\begin{flushright} 
% But in the real world it is more important that a proposition be interesting than that it be true.
% (A.N. Whitehead)
%\end{flushright} 


%one of the principles on which a belief or theory is based:
%It is a tenet of contemporary psychology that an individual's mental health is supported by having good social networks.
%An idea or theory on which a statement or action is based:
%[+ that] They had started with the premise that all men are created equal.
%The research project is based on the premise stated earlier.

\subsection{What is the common western view of reaity?}

\subsubsection{Substance philosophy?}
The common world view in the western world, the substance paradigm, is an heritage of the substance philosophy of classical Greek antiquity.
In the classical substance philosophy the primairy units of reality, called 'substances', the things in the world the world are composed of ever lasting inert atoms. The only change possible on these substances was the alteration in their position in space and time. The properties of substances never changed.
\cite{rescher1996process,seibt-2013-sep,rescher-2012-sep}.


%begin
\cite[ch 5]{rapp1990whitehad}
p72
This view of nature is an aggregation of a convoluted historical process through the combined efforts of many great scientist. For example Descartes gave us the mechanical conception of nature.

In this atomistic view reality is constituted of a merely mechanical conception of nature (Descartes), as  aggregation of parts and change is interpreted as a mere reorganisation of passive elements ( PR 208-209.)
end
%===Petitio Principii: (circular reasoning, circular argument, begging the question) in general, the fallacy of assuming as a premiss a statement which has the same meaning as the conclusion



\subsubsection{What is the problem with the common western philosophical view of reality?}
%For, on the one hand, it appears that the conceptual contents of the relevant scientific terms cannot, without problematic distortions, be analyzed in terms of the categories of substance metaphysics.
%Among the various cases in point for either one or both of these claims are (i) quantum physics, (ii) self-organization, and, most recently, (iii) embodied cognition.
%Creativity, the generation of novelty, conflicts with this idea of endurance and is therefore more difficult to explain in a world that exists of fixed and ever lasting building blocks.
\cite{seibt-2013-sep}

%\begin
\cite[ch 5]{rapp1990whitehad}
The Creativity of Church Teaching 1983 Gerald Thomas Floyd 

%===Concreteness is an aspect of communication that means being specific, definite, and vivid rather than vague and general. A concrete communication uses specific facts and figures.

%===Immanence refers to those philosophical and metaphysical theories of divine presence in which the divine encompasses or is manifested in the material world. Immanence is usually applied in monotheistic, pantheistic, pandeistic, or panentheistic faiths to suggest that the spiritual world permeates the mundane.

p72
Creativity or novelty is not found as a categorie in natural science.
%
%===discursive knowledge The dichotomy of discursive and intuitive knowledge requires comparing and contrasting in order to flesh out an essay on the relevant merits of the subject. Intuitive thought is naked, unmediated apprehension, immediately given (to experience), whereas discursive thought is mediated and articulated instead within language. Intuition is non-inferential awareness of abstract or concrete truths. 
%=discursive - proceeding to a conclusion by reason or argument rather than intuition

%=nomological relating to or denoting principles that resemble laws, especially those laws of nature which are neither logically necessary nor theoretically explicable, but just are so.

%===Platonism = the theory that numbers or other abstract objects are objective, timeless entities, independent of the physical world and of the symbols used to represent them.
\end
Friedrich Rapp states that in the natural science there is not a category for novelty or creativity 
\cite{rapp1990whitehead}.
%\todo{zoek ook een meer recente link die dat bevestigd, bv https://www.researchgate.net/post/Creativity_in_natural_sciences (zoek natural sciences category creativity)}

p73
One seeks in vain for the categories of creativity, of novelty, or of creative advance in the conceptual system of natural sciences.

NS is based on experimental methods and mathimatical description.

Rapp shows how modern science is build on the substance paradigm and mentions four points in relation to the epistemological an methodological methods sciences rest on for gaining discursive knowledge.
First the experimental procedure, the relevant object of examination is isolated from its natural environment and placed in a laboratory equipped with technological artifacts for te examination of the object. The technical equipment is however build on the very same knowledge that is found with this specific experimental procedure.

Second the the Analytical method. The aim of the scientific method is not an comprehensive understanding of phenomena but rather a conscious choice is made for a theoretical conceptual analysis which is made concrete by a experimental set up.

Third The mechanistic mode of thought. The model for the conceptual understanding is provided by the functional aspects of a mechanical systems and not by living beings that, by purpose, are organized in a certain way or value or goal directed human acting. The physical world is explained as existing of elements with exclusive material properties.

Fourth the mathmatical rendering of science that leads to general laws and axioms that can be used for deduction to more specifiation. The mathematical system enables exact predictions that are practical for technical application

p77
With the theoritical framework modern science offers there is no room for categories to elaborate on the origin of novelty and creativive advance. It is only possible to discus these phenomenon from other, already known concepts.
In this way modern science creates an reductionist view of novelty and creativity. It is not possible to explain every detail of novelty because the concepts are missing. If they would be there already we would not deal with creative novelty.

Because of the experimental method of isolation of phenomena and the abstaraction caused by of mathematical description, the very different sensation novelty and creativy exibit under different circumstances are disregarded.

p79 scientific ethod fruitfull, maar vergeleken met wat? De periode voor deze methode? Wat is de invloed van de wereld bevolking en de tijd geinvesteerd in wetenschap?

%=== duly =  in the correct way or at the correct time; as expected:

\todo{evt: Merits and Limits of Applying the Scientific Method to Human Society}
\todo{zie scientific revolution, vak Roberts}

\todo{kan ik hier voorbeelden aanhalen uit de boeken gelezen van phil en psych?}
\todo{Paul/Kaufman pp komt niet voor in de index en whitehead ook niet
	libgen heeft het boek niet
	google op title en whitehead of "process philosophy" levert niets op}


%===accident = without intending to, or without being intended

%===monad = An ultimate atom, or simple, unextended point; something ultimate and indivisible.

%===reproach = to criticize someone, especially for not being successful or not doing what is expected

%===mitigate = to make something less harmful, unpleasant, or bad

%===prehension = PHILOSOPHY an interaction of a subject with an event or entity which involves perception but not necessarily cognition. 




%===infinite regress is a series of an infinitely cascading propositions, where the validity of one depends on the validity of the one which follows and/or proceeds it. Viciously circular infinite regressions, are propositions which reintroduce their own proposition in the solution.

%===Causa sui is the Latin name for a self-caused cause, one that is not the result of prior events.

%===ex post facto = with retrospective action or force.

%===modality = a particular mode in which something exists or is experienced or expressed.

%===exegesis = critical explanation or interpretation of a text, especially of scripture.

%===corroboration = evidence which confirms or supports a statement, theory, or finding; confirmation.

Western metaphysics has long been obsessed with describing reality as an assembly of static individuals whose dynamic features are either taken to be mere appearances or ontologically secondary and derivative. 

The scientifical method is tuned for this worldview and the 
Knowledge comes from science
Scientifical method very fruitfull but fails to describe phenomena like creativity
Logic, descartes object-subject, see Rapp in ch 5
\cite[chapter 5]{rapp1990whitehad}
labratory, reductionist, isolating and testing, not as a combination (interwined verbeek)
paradigma, limits seeing
\cite[ch 5]{rapp1990whitehad}

\todo{stukje over kuhn p79 over dat een paradigma blind maakt voor andere zienswijzes}

\subsection{What is the process philosophical view of reality?}
{rescher1996process}.
There is not one process philosophy view but there are 
different views.
In the field of process philosophy Alfred North Whitehead and 
Charles Hartshorn are seen as the most important contributers 
to the contempory view of process philosophy. 
But thes ar by far not the only scholars. Especially in the 
North America's, process philosophie had and has a large 
community of philosophers. 
\cite{rescher1996process}

\subsubsection{Seibt.What is the process philosophical view of reality?}
%Process philosophy, or process theology, or simply process thought, is a tradition in philosophy that is in progress. Different scholars contribute different views but there are some commonalities.

%Quantum mechanics has caused some problems in the substance paradigma.

See Reschner introduction en SEP art \cite{Rescher-2012-sep}
Seibt SEP entry \cite{Seibt-2013-sep}

Process Philosophy is a methapisics that is based on the idea that the world is not static but in a state of flux. Reality consists in modes of becoming and types of occurrences. 
The things the world is build of is not an everlasting substance but are the representations of processes that interact with each other.  

However there are different views among process philosophers how this world is realized  most process philosophers share the following view.

For replacement of the descriptive concepts of sustance methaphisics, what an basic entity is, basic categories are introduced with focus on what a basic entity does.
Functionalities of dynamic entities are explained with the label of \textit{process}. 

A process is to understand in different ways. 
What holds for all processes is that they occur — that they are somehow or other intimately connected not only to temporal extension but also to the directionality or passage of time.

Some processes are to understand as our as our common understanding of a processes, a temporally structured sequences of stages of an occurrence, with each such stage being numerically and qualitatively different from any other. 

But some processes are different. For example they are non-development occurences like activities or non-spatiotemporal happenings that realize themselves in a developmental fashion and thereby constitute the directionality of time. 

Everything we experience is made out of all kind of processes for example physical, biological or cognitive, Micro processes aggregate to macro processes, complex dynamic organisations, for more complex behavior. 

In this way the perhaps most powerful argument for process philosophy is its wide descriptive or explanatory scope. For example the human body and the human mind are both explained by the same principle of being representations of processes, this in contrast with the substance philosophy.

The temporally stable and reliably recurrent aspects of reality, the basic principle of substance methaphisics, are in process philosophy explained as the regular behavior of dynamic organizations of processes.


\todo{add from reschner sep and introduction}


\subsubsection{What solution offers process philosophy?}
\paragraph{seibt}
One of the goals of pp was to overcome the problems problems that substance philosophy marginalizes or sidesteps al together. Therefore process philosphers start asking questions like those related to varieties of becoming, developments and the mergency of novel conditions \cite{seibt-2013-sep}.

These overlap my question of creativity as technological ideation.


\paragraph{ POTENTIAL-ACTUAL}
creative activity (transforming potentiality into actuality)

\paragraph{dualism}
???With the process as a universal building block process 
philosophy tries to overcome the object subject dualism.

\paragraph{science adapted processist views}
But there are other domains and topics of science that, as processists stress, directly imply a process-based metaphysics. For, on the one hand, it appears that the conceptual contents of the relevant scientific terms cannot, without problematic distortions, be analyzed in terms of the categories of substance metaphysics. On the other hand, the researchers working in these areas have already adopted a largely processist perspective in their informal glosses of mathematical descriptions and in their heuristic approach to the domain. Among the various cases in point for either one or both of these claims are (i) quantum physics, (ii) self-organization, and, most recently, (iii) embodied cognition.

\subparagraph{self-organisation, emergence waar kmenn entities vandaan, causation_
seibt alleen voor creativity:
(ii) Self-organization: Process metaphysics has traditionally been motivated by the fact that it seems to give the best explanation of the phenomena of emergence, originally understood as an integral feature of evolution. Since the development of scientific theories of “self-organization,” “chaos,” and “complexity” have begun to alter our understanding of evolutionary change, there is a new need for a metaphysics that can accommodate all sorts of phenomena where dynamic organizations exert causal constraints. While older, speculative, process metaphysics embraced the idea of purposes and creativity in nature, and allowed for the explanatory category of a ‘self-realizing’ or ‘self-engendering’ entity (in various terminological guises), present-day analytical processists confine themselves to arguing that “downward causation” becomes perfectly intelligible once physicalism has been divorced from the assumptions of the substance paradigm, and most especially from the principle that causal powers cannot be attributed to dynamic organizations.[10]
(10. See in particular Bickhard 2000 and Campbell/Bickhard 2011, as well as Wimsatt 1997, where emergence and reducibility are defined in terms of types of interactions with a physical system.)
Campbell, R., Bickhard, M., 2011, “Physicalism, Emergence, and Downward Causation,” Axiomathes, 21(1): 33–56.

\subparagraph{embodied, ideation, de invloed van de omgeving}
seibt
(iii) Embodied cognition: The turn to “embodied cognition” in cognitive science provides another strong motivation for the turn to process in metaphysics. The standard model of cognition as the computation of symbolic representations fits well with the assumptions of substance metaphysics and suggested a pleasing analogy to classical atomism: mental operations effect relational change of cognitive atoms. But the first rivals to the standard model, connectionism and the so-called “Dynamic Hypothesis” (Tim Van Gelder), were constructed along largely process-ontological lines, replacing the classical conception of cognitions as discrete abstract objects that represent concrete things outside the head with a dynamic conception of cognitions as modes of functionings of a neural net or of process organizations. Recent results in embodied cognition research seem to tip the balance further into the direction of a process-based philosophy of mind, since they suggest that the bodily interaction of an organism plays a constitutive role in cognition. Some proponents of embodied cognition or “interactivism” insist that the new focus on organism-environment interactions makes any talk about representations obsolete, while others argue for a naturalist account of emergent representational processes and emergent normativity (Mark Bickhard). A key notion for the ‘embodiment thesis’ is the concept of “structural coupling,” a phase in the co-temporaneous development of two systems (e.g., organism and environment) where mutual dynamic dependencies unfold across system boundaries. Critics argue that the embodiment thesis might only hold for some form of cognition, but whatever the scope of the thesis might be, the fact remains that a more detailed description of the notion of structural coupling requires a process-ontological framework.

\paragraph{recursive def}
The whole individual person is continuously ‘in the making’ or being constituted, yet it also continuously influences which components (e.g., experiences, feelings, actions) enter into the constitution of the whole and in which ways these components occur. Such circular dependencies between a whole and its parts cannot be accommodated within a theory of individuals that is committed to the basic constructional principles of the substance paradigm, especially the claim that concrete individuals are fully determinate. Relationships of mutual constitution are legitimate theoretical tools within process ontologies where entangled recursive definitions are not in conflict with basic tenets about individual entities. By extending the dynamic dependencies among the component processes of a self to include aspects of the person's physical and social context, a process account of persons can formulate in a differentiated and scientifically informed fashion various claims about the formative role of our environments.[12]


\paragraph{Unifying Claims}

What unifies contemporary process-philosophical research more than any other aspect, however, is its metaphilosophical aim to revise long-standing theoretical habits. Given its current role as a rival to the dominant substance-geared paradigm of Western metaphysics, process philosophy has the overarching task of establishing the following three claims:

(Claim 1) The basic assumptions of the ‘substance paradigm’ (i.e., a metaphysics based on static entities such as substances, objects, states of affairs, or instantaneous stages) are dispensable theoretical presuppositions rather than laws of thought.
(Claim 2) Process-based theories perform just as well or better than substance-based theories in application to the familiar philosophical topics identified within the substance paradigm.
(Claim 3) There are other important philosophical topics that can only be addressed within a process metaphysics.

\subsection{Why Whitehead?}

The acceptance of a basic categorie of processes brings with it a limitation in novelty because agregate processes inheret characteristics of the basic processes. Compare this with the evolution theory where new species depend on the genetic structure of their predecessor.
???If I accepte the reduction into inifinite view, there is no limitation in what is possible.

\begin{quotation}
	seibt sep
	However, within that broad framework, process philosophers debate about how such a world of processes is to be construed, how it relates to the human mind (which is another process) and how the dynamic nature of reality relates to our scientific theories. In consequence, process philosophers also differ in their view on the role of philosophy itself and in their choice of theoretical style
	
\end{quotation}

The best way to show that the core assumptions of the substance paradigm can be dispensed with is surely just to do it. For process philosophy, as for any attempt at theory revision, the proof of the pudding is in the eating. Since Whitehead's process metaphysics is terminologically somewhat difficult to digest at first try, contemporary processists increasingly take non-Whiteheadian routes into process philosophy and proceed from linguistic ruminations and a critical review of
the traditional philosophical menu.

p84 
whiteheads notion van creativity is heel algemeen ivm zijn strefen van een verklaring voor allse. Bergson en Hegel concept van creativty liggen dichter bij het event zelf maar zijn mogelijk niet algemeen toepasbaar.

p84
His philosophy offers one of the few serious endeavors fo formulate a comprehensive  system aimed ad harmononizing the thoroughness and universality of philosophical questioning with the state of knoeledge attainend by modern science \cite[p84]{rapp1990whitehead}

seibt, kan als kritiek gezien worden whitehead valt hier onder enk ik:
While older, speculative, process metaphysics embraced the idea of purposes and creativity in nature, and allowed for the explanatory category of a ‘self-realizing’ or ‘self-engendering’ entity (in various terminological guises)
\subsubsection{What is his model?}

Voorbeeld met de begrippen van Whitehead:
Process
Actueal-entity
Creativity
Concrenesse
Prehension
Misschien ook de categorien van zijnden.

\cite{whitehead1929process}
Whitehead builds his theory on three universal categories are 'creativity', 'many', 'one' and his concrete elements are 'actual entity', 'prehension', 'nexus'.


Whitehead replaces the mechanical substance paradigm with te idea of a of 
the universal relatedness and reciprocal prehension of real occasions, factors that are expressed in the concrete elements of 'actual entity', 'prehension', 'nexus'.


%===entity a thing with distinct and independent existence.

%===prehension PHILOSOPHY an interaction of a subject with an event or entity which involves perception but not necessarily cognition.
%=== nexus = a connection or series of connections linking two or more things.

%=== conception = %synonyms:	inception of pregnancy, conceiving, fertilization, impregnation, insemination; rarefecundation = the forming or devising of a plan or idea.

%===occasion = 1. a particular event, or the time at which it takes place. 2.reason; cause.

\subsubsection{How to use Whitehead?}
moeilijk te begrijpen, volledige werk -> volg stengers
Nadeel atomist, laat begin buiten beschouwing?
Nadeel God, reposotory functie, belemert niet de creativiteit, Sommige stellen God is niet nodig
\subsubsection{What is creativity in the view of Whitehead?}	

\paragraph{Strengers}
Stengers: wat is creativitteit volgens whithead?

p72
When Whitehead became a metaphysician,
conscious experience became a creature of passage, which itself has
become creativity.



\subsection{How does process philosophy help to understand creativity?}
\subsection{Disadventages of Whiteheads view?}
-Whiteheads concept of creativitym because of its universal and fundamental significance creativity ccomes close to the traditional concept of God.

\subsection{Conclusion}


\section{Conclusion}
