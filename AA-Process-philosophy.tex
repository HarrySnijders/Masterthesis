
%=== indicates the definition of words and concepts
\chapter{Process Philosophy (What is process Philosophy)}

%\begin{flushright} 
% But in the real world it is more important that a proposition be interesting than that it be true.
% (A.N. Whitehead)
%\end{flushright} 


%one of the principles on which a belief or theory is based:
%It is a tenet of contemporary psychology that an individual's mental health is supported by having good social networks.
%An idea or theory on which a statement or action is based:
%[+ that] They had started with the premise that all men are created equal.
%The research project is based on the premise stated earlier.

\section{Provisional Content}

\subsection{What is the common western view of reaity?}

\subsubsection{Substance philosophy?}
The common world view in the western world, the substance paradigm, is an heritage of the substance philosophy of classical Greek antiquity.
In the classical substance philosophy the primairy units of reality, called 'substances', the things in the world the world are composed of ever lasting inert atoms. The only change possible on these substances was the alteration in their position in space and time. The properties of substances never changed.
\cite{rescher1996process,seibt-2013-sep, rescher-2012-sep}.

\begin
\cite[ch 5]{rapp1990whitehad}
p72
This view of nature is an aggregation of a convoluted historical process through the combined efforts of many great scientist. For example Descartes gave us the mechanical conception of nature.

In this atomistic view reality is constituted of a merely mechanical conception of nature (Descartes), as  aggregation of parts and change is interpreted as a mere reorganisation of passive elements ( PR 208-209.)




\end
%===Petitio Principii: (circular reasoning, circular argument, begging the question) in general, the fallacy of assuming as a premiss a statement which has the same meaning as the conclusion



\subsubsection{What is the problem with the common western philosophical view of reality?}
%For, on the one hand, it appears that the conceptual contents of the relevant scientific terms cannot, without problematic distortions, be analyzed in terms of the categories of substance metaphysics.
%Among the various cases in point for either one or both of these claims are (i) quantum physics, (ii) self-organization, and, most recently, (iii) embodied cognition.
%Creativity, the generation of novelty, conflicts with this idea of endurance and is therefore more difficult to explain in a world that exists of fixed and ever lasting building blocks.
\cite{seibt-2013-sep}

\begin
\cite[ch 5]{rapp1990whitehad}
The Creativity of Church Teaching 1983 Gerald Thomas Floyd 

%===Concreteness is an aspect of communication that means being specific, definite, and vivid rather than vague and general. A concrete communication uses specific facts and figures.

%===Immanence refers to those philosophical and metaphysical theories of divine presence in which the divine encompasses or is manifested in the material world. Immanence is usually applied in monotheistic, pantheistic, pandeistic, or panentheistic faiths to suggest that the spiritual world permeates the mundane.

p72
Creativity or novelty is not found as a categorie in natural science.

%===discursive knowledge The dichotomy of discursive and intuitive knowledge requires comparing and contrasting in order to flesh out an essay on the relevant merits of the subject. Intuitive thought is naked, unmediated apprehension, immediately given (to experience), whereas discursive thought is mediated and articulated instead within language. Intuition is non-inferential awareness of abstract or concrete truths. 
%=discursive - proceeding to a conclusion by reason or argument rather than intuition

%=nomological relating to or denoting principles that resemble laws, especially those laws of nature which are neither logically necessary nor theoretically explicable, but just are so.

%===Platonism = the theory that numbers or other abstract objects are objective, timeless entities, independent of the physical world and of the symbols used to represent them.

\end
Friedrich Rapp states that in the natural science there is not a category for novelty or creativity \cite{rapp1990whitehead}.
\todo{zoek ook een meer recente link die dat bevestigd, bv https://www.researchgate.net/post/Creativity_in_natural_sciences (zoek natural sciences category creativity)}

p73
One seeks in vain for the categories of creativity, of novelty, or of creative advance in the conceptual system of natural sciences.

NS is based on experimental methods and mathimatical description.

Rapp shows how modern science is build on the substance paradigm and mentions four points in relation to the epistemological an methodological methods sciences rest on for gaining discursive knowledge.
First the experimental procedure, the relevant object of examination is isolated from its natural environment and placed in a laboratory equipped with technological artifacts for te examination of the object. The technical equipment is however build on the very same knowledge that is found with this specific experimental procedure.

Second the the Analytical method. The aim of the scientific method is not an comprehensive understanding of phenomena but rather a conscious choice is made for a theoretical conceptual analysis which is made concrete by a experimental set up.

Third The mechanistic mode of thought. The model for the conceptual understanding is provided by the functional aspects of a mechanical systems and not by living beings that, by purpose, are organized in a certain way or value or goal directed human acting. The physical world is explained as existing of elements with exclusive material properties.

Fourth the mathmatical rendering of science that leads to general laws and axioms that can be used for deduction to more specifiation. The mathematical system enables exact predictions that are practical for technical application

p77
With the theoritical framework modern science offers there is no room for categories to elaborate on the origin of novelty and creativive advance. It is only possible to discus these phenomenon from other, already known concepts.
In this way modern science creates an reductionist view of novelty and creativity. It is not possible to explain every detail of novelty because the concepts are missing. If they would be there already we would not deal with creative novelty.

Because of the experimental method of isolation of phenomena and the abstaraction caused by of mathematical description, the very different sensation novelty and creativy exibit under different circumstances are disregarded.

p79 scientific ethod fruitfull, maar vergeleken met wat? De periode voor deze methode? Wat is de invloed van de wereld bevolking en de tijd geinvesteerd in wetenschap?

%=== duly =  in the correct way or at the correct time; as expected:

\todo{evt: Merits and Limits of Applying the Scientific Method to Human Society}
\todo{zie scientific revolution, vak Roberts}

\todo{kan ik hier voorbeelden aanhalen uit de boeken gelezen van phil en psych?}
\todo{Paul/Kaufman pp komt niet voor in de index en whitehead ook niet
	libgen heeft het boek niet
	google op title en whitehead of "process philosophy" levert niets op}


%===accident = without intending to, or without being intended

%===monad = An ultimate atom, or simple, unextended point; something ultimate and indivisible.

%===reproach = to criticize someone, especially for not being successful or not doing what is expected

%===mitigate = to make something less harmful, unpleasant, or bad

%===prehension = PHILOSOPHY an interaction of a subject with an event or entity which involves perception but not necessarily cognition. 




%===infinite regress is a series of an infinitely cascading propositions, where the validity of one depends on the validity of the one which follows and/or proceeds it. Viciously circular infinite regressions, are propositions which reintroduce their own proposition in the solution.

%===Causa sui is the Latin name for a self-caused cause, one that is not the result of prior events.

%===ex post facto = with retrospective action or force.

%===modality = a particular mode in which something exists or is experienced or expressed.

%===exegesis = critical explanation or interpretation of a text, especially of scripture.

%===corroboration = evidence which confirms or supports a statement, theory, or finding; confirmation.



De wetenschappelijke methode is gabasserd op het paradigma van de matery fysica.
Rap heeft daar een stuk over geschreven.
In de materie fysica en afgeleide logica is geen plaats voor feiten die niet op een duidelijke oorzaak gevolg gebasseerd zijn zoals creativiteit en bijvoorbeeld ervaringen of gevoelens.
Ik denk dat de sociale wetenschappen daar beter op de plaats zijn omdat bv het gedrag van een maatschappij ook niet is te beredeneren op basis van een rechtstreekse relatie oorzaak en gevolg.

Western metaphysics has long been obsessed with describing reality as an assembly of static individuals whose dynamic features are either taken to be mere appearances or ontologically secondary and derivative. 

The scientifical method is tuned for this worldview and the 
Knowledge comes from science
Scientifical method very fruitfull but fails to describe phenomena like creativity
Logic, descartes object-subject, see Rapp in ch 5
\cite[chapter 5]{rapp1990whitehad}
labratory, reductionist, isolating and testing, not as a combination (interwined verbeek)
paradigma, limits seeing
\cite[ch 5]{rapp1990whitehad}

Modern science, by contrast, seem to \cite[chapter 5]{rapp1990whitehad}


The static view on the world is reflected in the current scientific method.
\todo{Friedrich Rapp creativity and modern science}
fruitfull (oh ja, wat als we anders hadden gekeken? Dat zou betekenen dat we de meest effieciente weg volgen?)

Scientific method, take subject of research in isolation-> taken out of the rich natural environment-> environment should be included in the subject of research because it will influence it behaviour.

Creativity, the generation of novelty, conflicts with this idea of endurance and is therefore more difficult to explain in a world that exists of fixed and ever lasting building blocks.
\todo{see Freidrich Rapp ch 5 in witeheads metaphysics of creativity about how modern sience is not capable of dealing with principles like creativity because of the scientific method}
In a process view of the world, change and creativity is a basic principle. The world is no longer constructed out of matter but consists of processes. Micro processes, related in myriad ways with other micro processes, build up to macro processes, that because of their recurred characteristic construct the world as we experience it. In this world view every thing is explained as a process, also we humans. 
\cite[ch 5]{rapp1990whitehead}
the atomistic view of a mere mechanical aggregation of parts

Thus, for process philosophy to succeed at all, it must be possible to identify a familiar experience that can be used to anchor the basic intuition that sameness is dynamic. In fact, some process philosophers (e.g., Whitehead) argue that the traditional notion of substance (a time-invariant, necessarily located particular) precisely lacks any experiential grounding, while process philosophy can draw on the experience we are most intimately familiar with, namely, the way in which we experience ourselves.


\subsection{What is the process philosophical view of reality?}

%Twentieth century physics has thus turned the tables on classical atomism. Instead of very small things (atoms) combining to produce standard processes (windstorms and such), modern physics envisions very small processes (quantum phenomena) combining in their modus operandi to produce standard things (ordinary macro-objects). The quantum view of reality has accordingly led to the unravelling of that classical atomism that has, from the start, been paradigmatic for substance metaphysics.

%However, another variant line of thought was also current from the earliest times onward. After all, the concentration on perduring physical things as existents in nature slights the equally good claims of another ontological category, namely processes, events, occurrences — items better indicated by verbs than nouns. And, clearly, storms and heat-waves are every bit as real as dogs and oranges.

See Reschner introduction en SEP art \cite{Rescher-2012-sep}
Seibt SEP entry \cite{Seibt-2013-sep}


Process philosophy, or process theology, or simply process thought, is a tradition in philosophy that is in progress. Different scholars contribute different views but there are some commonalities.

In the traditional western metaphysics, started with Aristotle and build on ideas of Descartes for dualism and Newton for a mechanical world view, reality is based on a description of entities that have a persistent character. Change is represented as an alteration of the attributes of these persisting entities and is of a secundary ontology. The primairy ontology is the entity that is made out of of ever lasting substance.

I this methaphisics ???


This metaphisic is based on a cause and effect principle that.

its substance, The basic principle is persistance, reality as we experience it, is an reality that is already exsiting.

In the methaphysics of process philosophie], change is the basic principle. The reality as we experience it is a reality that is becoming, being is becoming.
The continuously going on and coming about of reality is explained by the working of processes. The temporally stable and recurrend aspects what the substance philosophy explains by the concept of matter is in process philosophy explained with the regular behavior of a dynamic system that is the result ot the interaction of processes.

Processes can be grouped to macro process. Processe are related to other existing processes (in hoeverre?).
In the atomistic view of pp the reduction of processes is finit and ends at the categorie of basic processes. It is however also possible to see processes as construction of processes in to infinity, and deny the existance of basic categories of processes.

The acceptance of a basic categorie of processes brings with it a limitation in novelty because agregate processes inheret characteristics of the basic processes. Compare this with the evolution theory where new species depend on the genetic structure of their predecessor.
???If I accepte the reduction into inifinite view, there is no limitation in what is possible.

creative activity (transforming potentiality into actuality)

Voorbeeld met de begrippen van Whitehead:
Process
Actueal-entity
Creativity
Concrenesse
Prehension
Misschien ook de categorien van zijnden.


???With the process as a universal building block process philosophy tries to overcome the object subject dualism.

and in this way the body is a groupexist of the combination of ther processes

Because everything in pp is expained with processe
With the explanation of y is explained with processesm




The modern scientific method is 
in this metaphysics is presented as 


has long been based on a description of reallity existing of 
is based on entities
Process philosophy is a metaphysical endeavour that  explanation of the phenomena in the world based on the developmental nature of reality where as the dominant, western, explenation of the world is based on a static reality. 
Process philosophy share the idea that to understand the world and answer the basic philosophical questions it is best to understand the world as an ever changing reality, is not what is, but what is becoming.

There is not one process philosophy view but there are different views.
In the field of process philosophy Alfred North Whitehead and Charles Hartshorn are seen as the most important contributers to the contempory view of process philosophy. 
But thes ar by far not the only scholars. Especially in the North America's, process philosophie had and has a large community of philosophers. 

Nichlas Rescher has writen a clear introduction into PP \cite{rescher1996process}.

Process philosophers try to find one principle of explanation for al questions and in this the western  substance philosophy has failed, (Descartes gave us dualism).
With one and get rid of the object-subject dichotomy. For this they define one basic principle that lies behind all the entities of reality. 
In the substance philosophy not everything is to explain from the basic unit matter. In science this is neglected.
Newton mechanica -> quentum physics.

becoming and changing over static being. The dominant western worl view is that of a world consisting of entities that consist of matter that is constructed from small building blocks. The idea is that these blocks exist for ever.

\subsubsection{process philosophy?}
Twentieth century physics has thus turned the tables on classical atomism. Instead of very small things (atoms) combining to produce standard processes (windstorms and such), modern physics envisions very small processes (quantum phenomena) combining in their modus operandi to produce standard things (ordinary macro-objects). The quantum view of reality has accordingly led to the unravelling of that classical atomism that has, from the start, been paradigmatic for substance metaphysics.
\subsubsection{What are the Basic Doctrines?}
Third, the “measurement problem” presents a particular difficulty for substance metaphysics, since the latter rests on the assumption that all individuals are fully determinate independently of their interaction context. In contrast, process metaphysics endorses the principle that ‘interaction is determination.’[9]

(ii) Self-organization: Process metaphysics has traditionally been motivated by the fact that it seems to give the best explanation of the phenomena of emergence, originally understood as an integral feature of evolution. Since the development of scientific theories of “self-organization,” “chaos,” and “complexity” have begun to alter our understanding of evolutionary change, there is a new need for a metaphysics that can accommodate all sorts of phenomena where dynamic organizations exert causal constraints. While older, speculative, process metaphysics embraced the idea of purposes and creativity in nature, and allowed for the explanatory category of a ‘self-realizing’ or ‘self-engendering’ entity (in various terminological guises), present-day analytical processists confine themselves to arguing that “downward causation” becomes perfectly intelligible once physicalism has been divorced from the assumptions of the substance paradigm, and most especially from the principle that causal powers cannot be attributed to dynamic organizations.[10]

(iii) Embodied cognition: The turn to “embodied cognition” in cognitive science provides another strong motivation for the turn to process in metaphysics. The standard model of cognition as the computation of symbolic representations fits well with the assumptions of substance metaphysics and suggested a pleasing analogy to classical atomism: mental operations effect relational change of cognitive atoms. But the first rivals to the standard model, connectionism and the so-called “Dynamic Hypothesis” (Tim Van Gelder), were constructed along largely process-ontological lines, replacing the classical conception of cognitions as discrete abstract objects that represent concrete things outside the head with a dynamic conception of cognitions as modes of functionings of a neural net or of process organizations. Recent results in embodied cognition research seem to tip the balance further into the direction of a process-based philosophy of mind, since they suggest that the bodily interaction of an organism plays a constitutive role in cognition. Some proponents of embodied cognition or “interactivism” insist that the new focus on organism-environment interactions makes any talk about representations obsolete, while others argue for a naturalist account of emergent representational processes and emergent normativity (Mark Bickhard). A key notion for the ‘embodiment thesis’ is the concept of “structural coupling,” a phase in the co-temporaneous development of two systems (e.g., organism and environment) where mutual dynamic dependencies unfold across system boundaries. Critics argue that the embodiment thesis might only hold for some form of cognition, but whatever the scope of the thesis might be, the fact remains that a more detailed description of the notion of structural coupling requires a process-ontological framework.

\subsubsection{What solution offers process philosophy?}
seibt:
(iii) Embodied cognition: The turn to “embodied cognition” in cognitive science provides another strong motivation for the turn to process in metaphysics. The standard model of cognition as the computation of symbolic representations fits well with the assumptions of substance metaphysics and suggested a pleasing analogy to classical atomism: mental operations effect relational change of cognitive atoms. But the first rivals to the standard model, connectionism and the so-called “Dynamic Hypothesis” (Tim Van Gelder), were constructed along largely process-ontological lines, replacing the classical conception of cognitions as discrete abstract objects that represent concrete things outside the head with a dynamic conception of cognitions as modes of functionings of a neural net or of process organizations. Recent results in embodied cognition research seem to tip the balance further into the direction of a process-based philosophy of mind, since they suggest that the bodily interaction of an organism plays a constitutive role in cognition. Some proponents of embodied cognition or “interactivism” insist that the new focus on organism-environment interactions makes any talk about representations obsolete, while others argue for a naturalist account of emergent representational processes and emergent normativity (Mark Bickhard). A key notion for the ‘embodiment thesis’ is the concept of “structural coupling,” a phase in the co-temporaneous development of two systems (e.g., organism and environment) where mutual dynamic dependencies unfold across system boundaries. Critics argue that the embodiment thesis might only hold for some form of cognition, but whatever the scope of the thesis might be, the fact remains that a more detailed description of the notion of structural coupling requires a process-ontological framework.

The whole individual person is continuously ‘in the making’ or being constituted, yet it also continuously influences which components (e.g., experiences, feelings, actions) enter into the constitution of the whole and in which ways these components occur. Such circular dependencies between a whole and its parts cannot be accommodated within a theory of individuals that is committed to the basic constructional principles of the substance paradigm, especially the claim that concrete individuals are fully determinate. Relationships of mutual constitution are legitimate theoretical tools within process ontologies where entangled recursive definitions are not in conflict with basic tenets about individual entities. By extending the dynamic dependencies among the component processes of a self to include aspects of the person's physical and social context, a process account of persons can formulate in a differentiated and scientifically informed fashion various claims about the formative role of our environments.[12]


But there are other domains and topics of science that, as processists stress, directly imply a process-based metaphysics. For, on the one hand, it appears that the conceptual contents of the relevant scientific terms cannot, without problematic distortions, be analyzed in terms of the categories of substance metaphysics. On the other hand, the researchers working in these areas have already adopted a largely processist perspective in their informal glosses of mathematical descriptions and in their heuristic approach to the domain. Among the various cases in point for either one or both of these claims are (i) quantum physics, (ii) self-organization, and, most recently, (iii) embodied cognition.

(i) Quantum-physical processes: When Whitehead turned from mathematics to philosophy, he was quite aware that recent developments in physics (the demise of classical atomism in the face of quantum theory and relativity theory) had thrown out our old common-sense vision of the order of the universe. Quantum physics brought on the dematerialization of physical matter — matter in the small could no longer be conceptualized as a Rutherfordian planetary system of particle-like objects. The entities described by the mathematical formalism seemed to fit the picture of a collection of fluctuating processes organized into apparently stable structures by statistical regularities — i.e., by regularities of comportment at the level of aggregate phenomena. During the early decades of the twentieth century process philosophers were excited by the evidence that physics had turned the tables on that core refuge of substance metaphysics, classical atomism. Instead of very small things (atoms) combining to produce standard processes (avalanches, snowstorms) modern physics envisions very small processes (quantum phenomena) combining to produce standard things (ordinary macro-objects) as a result of an as yet not understood modus operandi that could, nevertheless, be mathematically described. So-called enduring “things” in this picture come about through the emergence of stabilities in statistical fluctuations, as a stability wave in a surging sea of process, metaphorically speaking.
\subsubsection{Unifying Claims}

What unifies contemporary process-philosophical research more than any other aspect, however, is its metaphilosophical aim to revise long-standing theoretical habits. Given its current role as a rival to the dominant substance-geared paradigm of Western metaphysics, process philosophy has the overarching task of establishing the following three claims:

(Claim 1) The basic assumptions of the ‘substance paradigm’ (i.e., a metaphysics based on static entities such as substances, objects, states of affairs, or instantaneous stages) are dispensable theoretical presuppositions rather than laws of thought.
(Claim 2) Process-based theories perform just as well or better than substance-based theories in application to the familiar philosophical topics identified within the substance paradigm.
(Claim 3) There are other important philosophical topics that can only be addressed within a process metaphysics.

\subsection{Why Whitehead?}
The best way to show that the core assumptions of the substance paradigm can be dispensed with is surely just to do it. For process philosophy, as for any attempt at theory revision, the proof of the pudding is in the eating. Since Whitehead's process metaphysics is terminologically somewhat difficult to digest at first try, contemporary processists increasingly take non-Whiteheadian routes into process philosophy and proceed from linguistic ruminations and a critical review of
 the traditional philosophical menu.

p84 
whiteheads notion van creativity is heel algemeen ivm zijn strefen van een verklaring voor allse. Bergson en Hegel concept van creativty liggen dichter bij het event zelf maar zijn mogelijk niet algemeen toepasbaar.
 
p84
His philosophy offers one of the few serious endeavors fo formulate a comprehensive  system aimed ad harmononizing the thoroughness and universality of philosophical questioning with the state of knoeledge attainend by modern science \cite[p84]{rapp1990whitehead}
 
\subsubsection{What is his model?}
\cite{whitehead1929process}
Whitehead builds his theory on three universal categories are 'creativity', 'many', 'one' and his concrete elements are 'actual entity', 'prehension', 'nexus'.


Whitehead replaces the mechanical substance paradigm with te idea of a of 
the universal relatedness and reciprocal prehension of real occasions, factors that are expressed in the concrete elements of 'actual entity', 'prehension', 'nexus'.


%===entity a thing with distinct and independent existence.

%===prehension PHILOSOPHY an interaction of a subject with an event or entity which involves perception but not necessarily cognition.
%=== nexus = a connection or series of connections linking two or more things.

%=== conception = %synonyms:	inception of pregnancy, conceiving, fertilization, impregnation, insemination; rarefecundation = the forming or devising of a plan or idea.

%===occasion = 1. a particular event, or the time at which it takes place. 2.reason; cause.

\subsubsection{How to use Whitehead?}
	moeilijk te begrijpen, volledige werk -> volg stengers
	Nadeel atomist, laat begin buiten beschouwing?
	Nadeel God, reposotory functie, belemert niet de creativiteit, Sommige stellen God is niet nodig
\subsubsection{What is creativity in the view of Whitehead?}	

\subsection{How does process philosophy help to understand creativity?}
\subsection{Disadventages of Whiteheads view?}
-Whiteheads concept of creativitym because of its universal and fundamental significance creativity ccomes close to the traditional concept of God.

\subsection{Conclusion}

\section{Conclusion}
