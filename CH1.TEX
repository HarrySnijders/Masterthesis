\chapter{Introduction}

Versie 27-juli-2015.\\



De \textit{provisional section} is een hulp tijdens het schrijven. Daar begint het eigenlijke hoofdstuk. Voor deze sectie staan algemene opmerkingen en ideen voor het hoodstuk.\\


Ik maak gebruk van het TODO pakket van latex om herinneringen toe te voegen voor mijzelfdie me te binnen schieten tijdens schrijven bijvoorbeeld.\\


Deze week heb ik de introductie in pp gelezen van Rescher en een artikel van Cloots over de vraag naar het ultime. Vervolgens heb ik de structuur van de thesis nog eens goed bekeken en ben overgestapt op een draft versie waar allelei in staat ook samenvattingen, en een writing versie die de uiteindelijke thesis moet worden.

En toen eindelijk maar aan het schrijven. Ik heb er toch voor gekozen om met de introductie te beginnen om nog eens scherp te krijgen wat ik nou eigenlijk aan het onerzoeken ben. Ik verwacht dat ik aan de verschillende secties in dit hoofdstuk delen ga toevoegen als ik met de verschillende subvragen bezig ga.
Voor maandag is alleen de introduction van belang. Het is een eerste draft, ik ben vooral benieuwd naar of je mijn redenering kunt volgen en overtuigend vindt waar dat van toepassing is, of er iets mist en of ik structurele taalfouten maak.\\

%A fundamental difference between a natural science like physics and an artificial science such as computer science relates to the age old philosophical distinction between is and ought.
%Natural science onderzoekt wat er is, computer science onderzoekt hoe het zou moeten. Misschien is ditte gebruiken.

Working definition:
Creativity is the production of novelty with a structure that did not exist before.
All existing entities in reality have a structure, whether it is a idea or an artifact.

\todo{SSn 22-7
Je engels niet slecht, maar je gebruikt teveel spreektaal en letterlijke vertalingen van het nederlands. Verder is het niet geheel duidelijk van de introduction waarom ik als lezer geinteresseerd moet zijn in je studie en hoe jouw studie in de literatuur past en waarom jij dit wilt onderzoeken. Een paar details:
}
\section{NEW The importance of creativity}
Steven Shaviro
p72 73
What is the meaning, and what is the import, of our belief in creativity
today?
How does the new enter into the world? 
And how does the valuation
of the new enter into thought? 

Deleuze explicitly invokes Nietzsche’s call for a
“revaluation of all values,” and for the continual “creation of new values”
(Deleuze 1994, 136). And Whitehead and Deleuze alike are inspired by
Bergson’s insistence that “life . . . is invention, is unceasing creation” (Bergson
2005, 27). But the real turning point comes a century before Bergson and
Nietzsche, in Kant’s “Copernican revolution” in philosophy. Kant himself
does not explicitly value the new, but he makes such a valuation (or revaluation)
thinkable for the first time. He does this by shifting the focus of philosophy
from questions of essence (“what is it?”) to questions of manner (“how
is it possible?”).1 Kant rejects the quest for an absolute determination of being:
this is an unfulfillable, and indeed a meaningless, task. Instead, he seeks
to define the necessary conditions—or what today we would call the structural
presuppositions—for the existence of whatever there is, in all its variety
and mutability. That is to say, Kant warns us that we cannot think beyond the
conditions, or limits of thought, that he establishes. But he also tells us that,
once these conditions are given, the contents of appearance cannot be further
prescribed. The ways in which things appear are limited, but appearances
themselves are not. They cannot be known in advance, but must be encountered
in the course of experience. This means that experience is always able to
surprise us. Our categories are never definitive or all-inclusive. Kant’s argument
against metaphysical dogmatism, which both Whitehead and Deleuze
endorse, entails that being always remains open. “The whole is neither given
nor giveable . . . because it is the Open, and because its nature is to change
constantly, or to give rise to something new, in short, to endure” (Deleuze
1986, 9). “Creative advance into novelty” (Whitehead 1929/1978, 222) is always
possible, always about to happen.



\section{Provisional section head}
\todo{waar mogelijk nog referenties toevoegen}
\todo{toevoegen een verwijzing dat creativiteit in filosofie een ondergeschoven kindje is terwijl de psychology het belang sinds 1950 de komst van J.P. Guilford als voorzitter van de APA}


(Grotendeels overgenomen van het proposal)\\

Our society is saturated with all kinds of technological objects that shape our every day lives in many ways. With modern ICT artifacts like smartphones and computers we stay in contact with other people in quite different ways then our ancestors did without these artifacts.  But artifacts, like smartphones and computers, are more then just tools for communication, they influence our perception of the world and influence our behaviour \cite{Verbeek2005,verbeek2011moralizing}. Artifacts are not neutral technological objects that function only as a tool to serve an end.

\todo{-- introduction; provisional section
	*sprong van cultuur naar humans naar creativity is te simplistisch en niet genoed justified, en leuk dat 'I see it' maar niet genoeg onderbouwd met bewijs van de literatuur.
	* moet er ook niet een doel en plan in de provisional section?
	*ik denk dat je een provisional section globaal moet houden dus niet te veel kleine dingetjes gaan beschrijven dan wordt deze section te chaotiech en moeilijk te volgen.
}
In the past, technology pessimist thought that the development of society and culture followed the development of technology and therefore technological development was hard or impossible to influence. 
But contemporary scholars argue that technological development did not follow a straight path as the most efficient way to the end it was designed for. Technological development, they argued, was depending on social and cultural factors \cite{Bijker1989}.

If technology development is depending on social and cultural factors and therefore on humans (individual, as a group or as an society al large) then humans are able to steer the development of technology and are responsible for the effects of the technological on themselves and the environment. \todo{bron vermelding krijg ik van Rdb} Therefore it becomes important to understand the powers at work in he development process and possibilities to control or at least influence this process.  
No matter how important social and cultural factors are, at the end, for the idea of new technologies, or the design  of new artifacts, creativity is needed because in the way I see it, creativity is lies at the hart of the technological novelty.

Creativity is a word used for description of many different concepts in the domain of novelty but there are some characteristics most scholars in the field degree on that is that creativity is related to the generation of ideas or artifacts that are both original and valuable. With original is meant that the novelty produced did not exist in a similar way in the world before. A newer version of a smartphone with minor changes will not be considered very original. The first smartphones however could be seen as an original new artifact. 
The value of novelty is less obvious. It is often used, Boden does for example, to exclude novelty produced by fools from creative novelty. But what to say about novelty that is used against people? In the case of technology, the artifact produced should have value for its users.
\todo{hier mist nog wat, misschien verder ingaan op de waarde van novelty, zie ook Berry Gauts, wil ik ook nog wat zeggen over het surprise effect, dat iets nieuws niet voor de hand mag liggen anders is wordt het niet als creatief beschouwd of doe ik dat pas later. Zie o.a. Boden, Fergusson}

%Different models exist representing the creative process, the ones I have seen so far are the computational model \cite{Boden2004} and the psychological GenePlore model \cite{finke1996imagery}.

Creativity is possible on different levels \cite{Ferguson2010}, groups or individuals can be creative. Also different kinds of agents can be creative such as systems, animals or humans (Ibid.). I will however, only consider technology that is made by humans and finds its origin  in human creativity. I will also focus on the creativity of a single human,

\section{Creativity}

\todo{-introduction; creativity * je hebt je keuzes hier niet genoeg justified waarom je focust op creativity en technology en op agent etc.
*je herhaalt jezelf in de definitie hier.
*voorbeelden zijn beter als je ze van de literatuure gebruikt ipv zelf verzinnen.
*ik denk dat je inderdaad moet toespitsen op de software ontwikkeling, omdat mij het gebied van TI algemeen te breed en complex lijkt.}

Creativity is a very wide spread concept and a highly appreciated value. It is used to classify a person or an artifact or an idea.  It is also used to denote the process of creation where novelty is produced. Creativity can be a an experience, a agent might think he is being creative in his practice.
\todo{bronnen vermelden}

%new and value
%Scholars have different ideas about what creativity is but most agree that creative ideas and artifacts are those that are new and have a value. Boden differs two types of new. New can be that it either has not existed before in human history, called historical new or it is new to the creative agent, called psychological new \cite{Boden2004}. The value clause is added to exclude random novelty from being considered creative. For example a random sequence of characters might be the first one of its kind but it has no value and is therefore not considered as a creative product.


\paragraph{Novelty, agents and ideas}
\todo{stukje over TA en upstream?}
% alleen geinteresseerd in nieuwe ideen
I am interested in the origin of new technology, created by men. Whether the agent that comes up with new technology is considered a creative agent or not seems not to be of great importance. However an agent that is potential seen as creative agent might generate other ideas than an agent that never will be considered creative because the ideas she generates are to smalls steps forward and to much alike other ideas. I will focus on the generation of novelty that has an potential of being considered as creative but will leave the agent out.
\todo{citeer het artikel over de genius en alledaagse creativiteit om problemen op te lossen}

The outcome of an creative process might be an artifact or an idea. in the case of technology, the production of the artifact is accompanied with extra limitations,  technology has to work to have value.
 It has, among others, a tool function. 
\todo{wat is technology eigenlijk en waar dient het voor, is daar iets over te vinden?}
Software for example has to be designed according strict rules else it will not execute on the hardware it is designed for. An technological artifact is very much depending on its specific environment and the current technological possibilities. 
Like all artifacts, a technological artifact has to be manufactured and because it is more or less complex this requires a design of some type first.
\todo{see Dasgupta}
A design is an answer to an idea, where the idea can be a problem to solve or a new functionality required.
The environment is often included in the design process. For example in the case software where the hardware it has to run often puts limitations to the software to develop and is therefore included in the design.
It is very easy to imagine that an idea for a new technology is not yet possible to realize, to make a design and build the artifacs to realize the idea. 
Like for example he idea of a submarine with crew that is shrunken to a microscopic size and injected into the blood stream of a nearly assassinated diplomat to save his life (IMDB, the fantastic voyage). It was before 1966 that the idea is posed but it still is not possible to shrink objects. 
Coining the idea was possible, maybe the story is based on earlier ideas about micro objects that could be injected into the body for repair jobs,  but building the technology is not yet possible. 

\todo{moet ik hier nog iets zeggen over kunst en science, omdat veel literatuur die ik gezien heb juist daar over gaat}
%These roles are much less strict in the domain or art where Many artifacts in the domain of art and literature,

Since I am interested in the origin of new technology and not the realisation of new technology, and the fact that in the domain of technology an artifact is always preceded by an idea and a design to accomplish that idea, I will not have to consider the creation of new artifacts for understanding the origin of new technology but it is sufficient to focus on the genesis of new ideas.

\todo{nog iets over innovatief gebruik van niewe technology en user innovation? Of het verschil tussen innovatie en orginaiteit?}


%hoe om te gaan met waarde?
%twee fasen process, ideeen genereren fase twee ideeen beoordelen.
\todo{misschien opnemen Creativity SCOC factor. Social Construction of Creativity. Het idee dat of iets als creatief beoordeeld wordt wordt bepaald door de maatschappij
	}

\paragraph{Novelty and value}
I have said something about novelty, the agent, artifacts and ideas but what about value? How important is it for the TI that the ideas produced have a value. 

Creativity is related to a value and for deciding if something has a value some kind of judgement is necessary. In order to be valued as an creative action or outcome, the subject of creativity has to be compared to its environment before it can be granted with the predicate creative. It does not matter who performs the valuation, the idea has to be known first before it can be the subject of valuation.This is even the case if the idea does not leave the human brain.
In fact some psychological theories of creative processes in the brain are based on this two phase principle like the Geneplore model \cite{finke1996imagery} and the BVSR (Blind Variation Reductive Selection) model \cite{simonton2003human}.
In the GenePlore model novelty is generated in one or more cycles of the following two phases process. In the first phase new ideas are generated and in the second phase these ideas are validated for their usefulness. It is possible te enter a new first phase and generate new ideas based on the outcome of the second phase, these on their turn are validated and so on. 
BVSR is based on the evolution theory. The first phase exist of generating new ideas based on the blind variation of existing ideas. In the second phase the ideas are validated and selected according their value.
\todo{wat is het verschil tussen GenPlore en BVRS}

First there is the novelty produced and then this novelty is the subject of a validation process using different criteria to compute its value.
This can only be accomplished after the idea is posed. It does not to relate to how novelty is created but to how the novelty is appreciated. 
The outcome of the valuation will be that the novelty is considered being not creative or creative on a sliding scale depending on how different the idea is in the light of already existing ideas.
\todo{boden surprise? Examples?}
Im interest lies at the first phase. But again, as I said with the agent, I will focus on the generation of new ideas that have a potential of being merited as creative ideas.

\paragraph{Technological ideation}
%technological ideation
Above I showed why my interest lies not at the agent, the artifact and te validation part of creativity. I am first of al interested in the coming into existence of the new idea.  
Referencing for this focus to creativity might lead to confusion because creativity is for denoting so many different aspects of novelty. I am interested in the very beginning of new technology, the idea for a new technology, and therefore understand creativity as technological ideation (TI). 
With understanding Technological Ideation I hope to find the source where new technology comes from and what the influence of ICT is on this source.
Because the common notion related to technological ideation is creativity, I will still use creativity to find a link in existing literature. 
\todo{In eerste instantie focus ik op de invloed van softare op  TI in het algemeen. Het kan echter nodig blijken om TI toe te spitsen op een beperkerter domein, dan kies ik mogelijk wederom voor software ontwikkeling}


\section{An other world view}
The common world view in the western world is that of a world constructed out of matter an Aristotelian heritage. Matter consist of small, standard, building blocks that last for ever. 
Creativity, the generation of novelty, conflicts with this idea of endurance and is therefore more difficult to explain in a world that exists of fixed and ever lasting building blocks.
In a process view of the world, change and creativity is a basic principle. The world is no longer constructed out of matter but consists of processes. Micro processes, related in myriad ways with other micro processes, build up to macro processes, that because of their recurred characteristic construct the world as we experience it. In this world view every thing is explained as a process, also we humans. 

Change is a basic principle in process philosophy. There is not one version of process philosophy, different scholars have different description which contribute to the overall idea of process philosophy.
On of the goals of process philosophy is to overcome the limitations of the substance paradigm and cause a paradigm view.
American philosophy turned out as a fertile source for process philosophy with scholars as C.S. Peirce, John Dewey and William James. The best known process philosopher from the 20th century with the most comprehensive description of Process philosophy was A.N. Whitehead. For Whitehead creativity is the absolute principle of existence. It is his process philosophy I like to use for this case.
\todo{Bergson, Hartshorn}
\todo{twijfels zijn er ook, zijn atomistisch idee van basis processen staat me wat tegen, liever ga ik uit van het oneindig opdelen  van processen, maar dat zien we verderop in het hoofdstuk over proces philosophie}

\todo{moet hier nog meer bij, bv welke auteurs, dat er verschilende stromingen zijn}
\todo{misschien refereren naar de presentatie van Mieke Boon over anders kijken. Met PP kijk ik anders dan met de gangbare vaste stof metaphysica}


\section{What type of technology}
\todo{--introduction; what type of technology
*justification for software choice...persoonlijke ervaringen zijn niet genoeg, je moet hier ook iets van de literatuur bij halen; bv dat er nog niets is gedaan mbt software (uniqueness)}
Technology is a very broad domain with very different disciplines. 
I could either research from a top down position the process of technological ideation (TI) or perform the research bottom up.
Top down means that I do not choose for a specific technology but take the field of technology as a whole. Besides that this could turn a out to be a to big endeavour, I also think that I could oversee details that are important for my case. Therefore I choose a bottom up  approach and choose for limited technological field.
I choose as a technology ICT and more specific the domain of software. I have two reasons for this choices. First because I am a software engineer of origin which grants me with specific knowledge in the field of software development. Second because software, how it operates with its concept of processes and the products it brings to the fore like for example virtual reality reassembles the world view of process philosophy.


\section{Problem statement}
%-------------------------------------------------------------
\todo{dit is de, grotendeels oorspronkelijke, omschrijving uit het thesis proposal, deze houd ik al schrijvende aan de verschillende secties bij en gebruik ik tijdens het schrijven als een rode draad. }
\todo{niet te veel naar kijken, vooral aan het eind bijwerken}

Philosophers are increasingly interested and looking at the relations between humans and technology. 
In the field of ethicists new technology are assessed for their influence on the environment and their impact on just distribution of goods and fair distribution of the good life,  anthropological philosophers study how humans perceive the world and how humans are shaped both by technology and social philosophers study the influence of the society on technological development.
There is however remarkable little attention from philosophers what lies at the very hart of new technology, creativity \cite{gaut2010philosophy}. How does a human get her first idea for a new technology, and what than is the impact of technology on her creativity?

A contemporary field of technology where innovations cycles have become very small and new technologies are created at a very high rate is ICT (information and communication technology).

I propose therefore to ask the following question:
\textit{(Q) What is the impact of ICT software technology on creativity?}
In order to answer this question we have to understand what human creativity is. 

Therefore I propose following sub questions:
\textit{(Q1) What is (human) creativity} and  
\textit{(Q1) What is IT-software} and  
\textit{(Q1) What is Proess Philosophy} and  
\todo{aanvullen, dit zijn waarschijlijk de verschillende hoofdstukken}

\section{Conclusion}

\section{Outline of the thesis}
