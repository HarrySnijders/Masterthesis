\chapter{The concept of creativity}

-Laat de verschillende concepten in filosofie zien.
-Laat de verschillende verklaringen, uitleg over de werking in de psychology zien.
-Leg uit wat het begrip creativiteit betekend in de PP en schets een mogelijke werking. Gebruik daarvoor de verschillende PP auteurs, zie boek rescher.

\section{Provisional section head}
\section{Whitehead creativity}
\section{Halewood: Whitehead creativity}
Zie de gemarkeerde delen in de text.

Whiteheads concept of creativity might be better understood when we know what his intention was.
\begin{quotation}
	One of Whitehead’s abiding concerns was to
develop a scheme of thought which would be able to encompass and account for new
things, ideas, entities and processes being generated within existence.
\end{quotation}

\begin{quotation}
	Whitehead,
therefore, set himself the daunting task of elaborating a scheme which could genuinely
describe how new things come to be. This is the role he assigns to creativity.
\end{quotation}

\begin{quotation}
This is one of the first mistakes
that seems to have been made in the adoption and promulgation of Whitehead’s term.
It has been reduced in scope and made a characteristic of human endeavour.
\end{quotation}

\begin{quotation}
	The second chapter of this work is titled ‘The Categoreal
Scheme’ and in it Whitehead sets out the Categories that he will elaborate throughout
the rest of the text.
\end{quotation}

\begin{quotation}
Or, to put it another way, there are only instances of creativity, rather than creativity in
itself. Creativity is not an ethereal resource which can simply be tapped in to by gifted
humans. It is not an infinite realm of inspiration. Whitehead goes on: ‘creativity is
always found under conditions, and described as conditioned’ (PR 31).
\end{quotation}


\begin{quotation}

	This seems to be an example of one of the moments when Whitehead resorts to the
terminology of medieval scholastic philosophy; when he talks of an “ultimate which is
actual in virtue of its accidents”. An accident is distinct from an essence. Accidents are
variable qualities as opposed to the fixed qualities which comprise an essence. For
example, if we were to agree that a stone had an essence, then whether it were grey or
white, or hot or cold would be its accidents (its accidental qualities). What is interest in
this, for Whitehead and for us, is that he describes creativity as actual in virtue of its
accidents. That is, it only exists in the manifestations of creativity in the world; it does
not have an existence separate from this.
\end{quotation}

\begin{quotation}
We should not ask “how can we be creative?” or “how do we maximize creativity?” or
“how can we tap into creativity?” Creativity, according to Whitehead is a universal, in
his very specific sense of the word, but this is not to imply that facts, things or people
can be placed within or draw upon creativity. 

Rather, it is the other way around. We
need to explain how the abstract emerges from the concrete. 

This is the core of
Whitehead’s philosophy. Obviously, Whitehead is using the terms abstract and
concrete here in a very specific way.

it might be worth remembering that Whitehead is always at pains to account for both
the process and the reality. He also wants to account for both becoming and being (in
the sense of hard material things about us
\end{quotation}

\begin{quotation}
“General” potentiality functions as an abstract condition which provides a metaphysical
positioning and consistency to Whitehead’s argument. “Real” potentiality refers to
Whitehead’s insistence that creativity is only to be found through and in those occasions
of becoming which populate the world.
\end{quotation}

Creativiteit bestaat dus in wat het voortbrengt. Het is de ultime categorie, en daarom is het nuet te beschrijven, er is niets weermee het beschreven kan worden. Het wordt alleen zichtbaar in wat het voortbrengt.
Zo geredeneerd gaat mijn casus geen resultaat leveren. De invloed van IT op creativity bestaat simpelweg niet. 
Of het moet zijn dat creativity creativity is only to be found through and in those occasions
of becoming which populate the world, real potentially.

Betekend dat many becomes one, dat wat bestaat beinvloed wat er nieuw komt.
Ik weet niet of ik in Whitehead meer ga vinden hoe het dan beincvloed. Misschien naar Bergson kijken..

>>>Hier verder met stephen c pepper zie reschner


Redenatuie:



\section{Conclusion}